
\section{Leviers d'action environnementale}
\label{section:lev_act_env}
Dans la section précédente, nous avons détaillés les facteurs qui favorisent ou limitent l'action environnementale. Nous présentons maintenant les différents leviers d'action identifiés parmi les cas étudiés. (1) Le premier levier repose sur l'innovation, l'adaptation de l'activité de l'entreprise et la mise en place de fonctionnements ou de produits alternatifs.  (2)  Les entreprises de l'échantillon cherchent ensuite à informer, à sensibiliser le public sur les enjeux environnementaux, et à diffuser les innovations. (3) Enfin, certaines s'appuient sur l'action militante et le lobbyisme pour agir à plus grande échelle. 


    \subsection{Innover sur le plan environnemental}
    
        Au sein des organisations étudiées, l'innovation environnementale s'observe à différents niveaux. Elle peut prendre la forme de simples ajustements, d'adoptions de "bonnes pratiques" communément répandues par ailleurs. Il s'agit donc d'innovations seulement du point de vue de l'organisation. A l'inverse, les innovations peuvent avoir une portée plus importante et constituer une réelle nouveauté du point pour le marché ou pour la société en générale. 
        
        Dans cette section, nous nous appuyons sur les résultats de ce chapitre pour formuler une série de propositions pouvant être testées par la suite. 
    
        \subsubsection{Limiter son impact}
        
            Plusieurs organisations, comme Air PACA, la Tour du Valat, ou encore Provence Forêt mettent en oeuvre des innovations qui visent à limiter l'impact environnemental de leur activité. Ces innovations prennent la forme ou de changements d'ampleur plus ou moins importante au quotidien, telle que la réduction des consommations de fournitures, la mise en place d'un système de recyclage. Les trois organisations citées ont également réorganisé les modes de déplacement (téléconférences préférées aux déplacements chez Tour du Valat dans la mesure du possible, adoption de véhicules électriques chez Air PACA, et optimisation des déplacements des techniciens de Provence Forêt). Il s'agit là d'adoption de pratiques déjà existantes par ailleurs. Ces changements ne remettent pas fondamentalement en cause l'activité ; ils visent uniquement à la rendre moins dommageable. 
            
            Ce type d'innovation semble répondre plutôt à un besoin de mise en cohérence avec les valeurs exprimées par les organisations, ainsi qu'aux attentes sociétales exprimées en faveur de l'écologie. Les organisations qui expriment de fortes valeurs environnementales cherchent ainsi à agir en cohérence avec leur discours. Les pratiques écologiques répondent alors un besoin de légitimité. 
            \begin{quotation}
                \citit{L’idée c’est bien ça, c’est de dire : notre mission c’est de servir l’environnement mais est-ce que nous on met en oeuvre ce qu’on préconise, et est-ce que dans notre quotidien on est cohérent avec les valeurs qu’on défend.} (Directeur, Fondation Tour du Valat) \\
                \citit{ça fait parti des choses importantes qu’on fait en étant un acteur de l’environnement que d’aligner ce que l’on fait avec ce que l’on est censés promouvoir.} (Directeur, Air PACA)
            \end{quotation}
            Provence Forêt est dans une situation particulière à cet égard, l'activité de coupe des arbres ayant une image très négatives, perçue comme une atteinte à l'environnement. Les efforts fait par l'organisation pour limiter son impact et adopter une approche durable de la gestion des forêts vise à rééquilibrer cette perception et à favoriser l'adhésion du public. 
            
            Enfin, ces innovations s'accompagnent souvent d'avantages économiques: la réduction des consommations de fourniture ou de carburant s'accompagnant généralement d'une baisse des coûts correspondant. La dimension monétaire est donc aussi un facteur motivant ces actions. 

            \begin{hyp}
                \label{prop:A}
                La limitation de l'impact environnemental est motivée par des facteurs externes ou de compétitivité.
            \end{hyp}
            
            Les cas étudiées montrent que ces innovations peuvent être organisationnelles (réorganisation des modes de transports), technologiques (adoption de véhicules électriques ou encore sociales (encouragement du co-voiturage chez Tour du Valat).
            
            \begin{hyp}
            \label{prop:B}
                La limitation de l'impact environnemental s'appuie sur des éco-innovations organisationnelles, sociales et technologiques.
            \end{hyp}
        
        \subsubsection{Développer des alternatives}
            Une démarche d'innovation environnementale plus pro-active voir stratégique est adoptée par des \oess. Il ne s'agit plus de se conformer aux "bonnes pratiques environnementales", mais bien d'en développer et promouvoir de nouvelles. Ces innovations ont généralement une portée plus grande, et s'appuie sur des nouvelles technologies, ou des évolutions des modes d'organisation ayant un fort impact. Elles sont plutôt observées dans les organisations du panel dont l'activité est explicitement liée à l'environnement. Air PACA, dont le métier est la surveillance de la qualité de l'air, ce type d'innovation est stratégique pour suivre les évolutions du secteur. L'association adopte de nouveaux outils de mesure, et s'inscrit dans une forte dynamique de digitalisation et de mise à disposition des données. Pour la Tour du Valat, de telles innovations s'inscrivent pleinement dans la mission même de la fondation. Elle se présente ainsi comme un laboratoire d'innovation autour de la préservation des zones humides. Non seulement l'activité de recherche de la fondation est ce qui fait sa réputation, et qui assure donc sa pérennité, mais elle répond avant tout à une volonté portée dès la création de \cit{réconcilier l'homme avec la nature}. Les recherches sont donc animées par l'engagement des salariés et des dirigeants. De manière similaire, l'association Bzzz a été fondée dans une démarche écologiste. Elle promeut une démarche nouvelle, à l'écart des pures logiques de rendement et de rentabilité, symbolisée par son activité \cit{d'apiculture alternative}. La démarche va en réalité plus loin puisqu'elle concerne également la dimension humaine de la gestion. Elle s'inscrit explicitement dans une approche militante, et cherche à avoir un impact au delà de son activité, en diffusant ses méthodes alternatives le plus largement, et en agissant auprès des pouvoirs publics. Enfin, l'association Enfants et Loisir a considérablement fait évoluer ses modes d'accueil et d'accompagnement des enfants, à l'initiative d'une personne animée par des convictions environnementales. 
            
            Ce sont donc la dimension d'engagement des personnes, ainsi que l'enjeu stratégique des questions environnementales dans l'organisation qui motivent ce type d'éco-innovations. 
    
            \begin{hyp}
            \label{prop:C}
                Le développement de fonctionnements alternatifs est motivé par l'engagement personnel et collectif et la dimension stratégique donnée à l'action environnementale. 
            \end{hyp}
            
            Les cas étudiées montrent que ces modes d'actions s'appuient souvent sur des éco-innovations technologiques et institutionnelles. Air PACA s'appuie par exemple de plus en plus sur des capteurs individuels, moins coûteux que les instruments de mesure classiques, et qui peuvent être installés chez des particuliers. Bzzz a développé la Bzzz-mobile, miellerie-savonnerie itinérante aménagée dans un camion, afin de pouvoir être mise à disposition d'apiculteurs ou bien d'être présentée au public. La Tour du Valat s'appuie sur de nouveaux procédés pour faire fonctionner sa chaudière à partir des déchets verts produits en Camargue. Ces organisations permettent aussi de faire évoluer les pratiques au niveau institutionnel, en lien ou non avec les évolutions technologiques citées précemment. Alors que l'ingestion des cartouches en grenaille de plomb était néfaste pour les canards, la Tour du Valat, en partenariat avec les chasseurs, a développé des cartouches en grenaille de fer et acier, dont l'efficacité a été prouvée. L'association Bzzz mène également des actions de lobbyisme, obtenant ainsi l'annulation d'un décret qui rendait presque impossible l'installation de ruches à Marseille, et la réouverture d'une miellerie appartenant à la mairie et laissée à l'abandon.
            
            \begin{hyp}
            \label{prop:D}
                Le développement de fonctionnements alternatifs s'appuie sur des éco-innovations technologiques et institutionnelles.
            \end{hyp}

    \subsection{Promouvoir les bonnes pratiques environnementales}
    
       Les organisations les plus actives en matière d'action environnementale s'efforcent d'augmenter la portée de leurs actions en les diffusant aurpès d'autres acteurs. Cela peut prendre la forme d'une action d'information et de sensibilisation, d'une démarche de partage des connaissance, jusqu'à la mise en oeuvre d'une dynamique sociétale visant à associer de nombreuses parties prenantes au processus.
       
        \subsubsection{Sensibiliser le public}
        Des actions de sensibilisation à l'action environnementale sont observées chez Bzzz, Air PACA, et Enfants et Loisirs. Bzzz cherche à mobiliser le public en étant présents sur différentes manifestations autour de thématiques environnementales. La Bzzz Mobile, miellerie aménagée dans un camion leur permet de montrer leurs processus. L'association organise également des animations et des formations pour présenter leurs pratiques au public. La sensibilisation est un des quatre volets de l'action d'Air PACA. Elle suit la mission de surveillance et s'inscrit dans une logique d'intérêt général : le rôle de l'association est de communiquer de manière transparente les résultats et les mesures effectués. Au contraire, l'information du public ne fait pas partie de la mission de la crèche Enfants et Loisirs. Pourtant, dans le cadre de sa démarche écolo-crèche, l'association organise des évènements visant à sensibiliser les habitants de la commune aux enjeux environnementaux. L'organisation vise à encourager l'application des bonnes pratiques au delà du cadre de la crèche, en invitant les familles à imiter les comportements bénéfiques chez eux. 
        
        Ces démarches de sensibilisation et d'information sont pro-actives ; elles ne sont pas contraintes par des facteurs extérieurs ou par une nécessité économiques, mais correspondent à une volonté interne de promouvoir une prise en compte plus large de l'environnement. Ceci s'inscrit dans le cadre de la mission de l'organisation ou provient également de l'engagement personnel des membres de l'organisation. Pour Air PACA, l'information revêt une dimension stratégique puisqu'il s'agit de la raison d'être de l'organisation. Au contraire, cette activité parait plus éloigné du rôle initial de la crèche Enfants et Loisirs. C'est un engagement interne qui est à l'origine des actions de sensibilisation. Finalement, Bzzz combine à la fois une mission environnementale et un très fort engagement des membres.
        
            \begin{hyp}
            \label{prop:E}
                Les actions d'informations et de sensibilisations sont motivées essentiellement par les facteurs engagement et stratégie.
            \end{hyp}
            
            Les exemples traités font ressortir deux types d'éco-innovations mobilisées pour l'information et la sensibilisation. Ce sont d'une part des innovations organisationnelles, autour de la mise en place de nouvelles activités, telles que l'organisation ou la participation à des évènements en lien avec les problématiques environnementales. D'autre part, les organisations s'appuient sur des technologies spécifiques. Bzzz mobilise sa miellerie mobile pour parler de son approche de l'apiculture, et Air PACA investi dans les outils numériques pour toucher un public plus large et pouvoir partager des données en temps réel. 
            
            %%% Innovation orga : mise en place d'actions, développement d'activités spécifiques
             %%% Innovation techno : Bzzz Mobile, data (air paca)
            \begin{hyp}
            \label{prop:F}
                Les actions d'informations et de sensibilisations  s'appuient essentiellement sur des éco-innovations organisationnelles mais aussi sur des innovations technologiques.
            \end{hyp}
            
        \subsubsection{Diffusion des innovations}
            La diffusion de l'action environnementale passe également par un partage des innovations développées par l'organisation. Il ne s'agit plus seulement de sensibiliser, mais bien de mettre à disposition des acteurs de l'économie des outils pour agir. L'idée est bien là de permettre un changement d'échelle et d'aboutir à un impact environnementale significatif. Provence Forêt et Enfants et Loisirs bénéficient de ce type d'action. La coopérative forestière fait partie d'un groupe de coopératives régionales du même secteur, qu'elle a rejoint en vue de la certification PEFC, et dont elle bénéfice des innovations.
            \begin{quotation}
                \citit{On n’a pas assez d’argent pour mettre en place des choses, mais on a la chance dans le groupe de coopération forestière, c’est que les autres coopératives mettent vraiment l’accent sur les innovations et donc il y a beaucoup de recherche-développement} (Responsable Environnement, Provence Forêt)
            \end{quotation}
            De la même manière, l'évolution de l'association Enfants et Loisirs vers des pratiques respectueuses de l'environnement a été accompagnée par l'association EcoloCrèche, qui a développé de nombreux procédés et qui les essaime dans de nombreuses crèches. 
            
            Afin d'essaimer son modèle, et ses techniques apicoles alternatives, l'association Bzzz opte pour la transparence. L'ensemble de ses documents (statuts, comptes rendu de réunions, documents de travail...) sont rendus disponibles sur leur site internet. La fondation Tour du Valat s'inscrit également pleinement dans une logique de diffusion des techniques. Elle s'efforce ainsi d'exporter les innovations développées en Camargue à d'autres  écosystème similaires en France ou à l'étranger où elles peuvent s'appliquer. \\
            
            Comme pour les actions de sensibilisation, le partage des innovations s'explique par l'engagement personnel et collectif envers l'environnement, et par la dimension stratégique de l'innovation environnementale dans les organisations (Pour la Tour du Valat et Air PACA, cette démarche s'inscrit pleinement dans leur activité). Un élément qui apparaît déterminant dans ces actions de diffusion est la logique collective et partenariale. Pour le directeur de la Tour du Valat, le recours à des partenaires est indispensable pour avoir un impact global: \citit{on a une mission qui est quand même très ambitieuse, à la fois en termes de couverture géographique puis de ce que ça suppose comme changement, et donc il est évident qu’avec nos forces vives on n’est pas la maille.}. L'organisation cherche constamment de nouveaux partenariats avec des acteurs aussi bien publics que privés, du secteur lucratif ou non lucratif. L'association Bzzz s'inscrit également dans une démarche collective, et préfère voir des partenaires chez ceux qui pourraient se poser en concurrents. 
            
            Comme précédemment (cf. proposition \ref{prop:E}), les motivations sont de deux ordres. D'une part, elles peuvent résulter de la mission de l'organisation, et revêtir une dimension stratégique. D'autre part, l'engagement, les convictions et les valeurs sont un moteur de la diffusion des innovations. En effet, une forte conscience des enjeux environnementaux amène à rechercher une action plus générale qu'une simple limitation de l'impact de l'organisation. 

            \begin{hyp}
            \label{prop:G}
                Les actions de diffusion des innovations sont motivées essentiellement par les facteurs engagement et stratégie. Elles sont favorisées par les partenariats et le collectif.
            \end{hyp}

            % Pas de sens de dire "quel type d'EI on met en place pour diffuser les EI". ==> Proposition pas très pertinente et difficile à justifier. ==> OUT. 
            %\begin{hyp}
            %\label{prop:H}
            %    Les actions de partage des connaissances s'appuient essentiellement sur des éco-innovations technologiques et organisationnelles.
            %\end{hyp}
            
        \subsubsection{Créer des synergies, initier une dynamique sociétale}
        
        
        \todo[inline]{A voir... peut être pas pertinent ici car difficile à illustrer avec les cas étudiés, mais intéressant de parler de cette dynamique dans la discussion ? Car beaucoup d'auteurs y voient la un des rôles de l'ESS. ON risque aussi d'être très redondant avec les sections précédentes car c'est proche de l'action collective}
            \begin{hyp}
            \label{prop:I}
                La création de synergies et l'initiation d'une dynamique sociétale sont motivées essentiellement par les facteurs engagement et stratégie. Elles sont favorisées par les partenariats et le collectif.
            \end{hyp}
            \begin{hyp}
            \label{prop:J}
                 La création de synergies et l'initiation d'une dynamique sociétale s'appuient essentiellement sur des éco-innovations sociales et institutionnelles.
            \end{hyp}
        
    \subsection{Action militante et lobbyisme}
        
        Nous avons vu que l'innovation constituait un premier moyen d'action environnementale. Puis nous avons montré que les organisations cherchaient à diffuser les pratiques pour permettre un changement d'échelle. La dernière catégorie de leviers d'action identifiée est l'action militante.
        
        Celle-ci est particulièrement présente chez l'association Bzzz, qui se présente comme une association engagée. L'étude de ce cas fait apparaître plusieurs exemples d'actions auprès des acteurs politiques locaux. Un premier exemple est l'installation de ruche en milieu urbain à Marseille, conduisant in fine à l'invalidation d'un décret municipal interdisant de telles installations à moins de 300 mètres des habitations. Une autre illustration est l'obtention de la mise à disposition d'apiculteurs par la Mairie d'une miellerie laissée inutilisée dans un des parcs de la ville. L'association se positionne donc parfois en opposition, mais aussi comme force de proposition auprès des responsables publics:
        \begin{quotation}
            \citit{De plus, avec Filière Paysanne, SOS Longchamps etThomas, articuleur, nous avons
            rédigé en commun nos propositions pour le développement agricole, alimentaire
            et économique du bassin marseillais et de sa région que nous avons soumis aux
            candidats aux élections municipales marseillaises.} (Site internet)
        \end{quotation}
    
        
        C'est avant tout l'engagement personnel et les convictions individuelles qui poussent à actionner ce levier d'action. Pour l'un des co-directeurs de Bzzz, la possibilité d'avoir un positionnement engagé est même une des raisons pour prendre part à l'association. Il explique ainsi : \citit{Moi je suis assez militant dans mes engagements, l’association Bzzz me paraissait et me paraît toujours engagée}. De même, les actions de la Tour du Valat s'inscrivent dans les trace d'un homme qui souhaitait \citit{réconcilier l'homme avec la nature}. 
        
        La majorité des actions militantes menées par l'association sont collectives. Bzzz s'associe autant que possible à d'autres organisations, et prend part à des actions collectives, comme des manifestations. La Tour du Valat s'appuie sur des partenariats institutionnels avec des acteurs publics ou des associations pour atteindre une portée et une légitimité suffisante pour agir. Un exemple frappant, pour une organisation à mission environnementale, est le partenariat avec les chasseurs, et l'activité de chasse menée en interne. Ceci traduit la volonté de la fondation d'avoir une démarche de co-construction plutôt que d'opposition. Une telle approche se retrouve également chez Bzzz, dont les abeilles sont menacées par les exploitations agricoles conventionnels voisines. L'association mène donc un travail de sensibilisation afin d'amener les agriculteurs à évoluer vers des pratiques plus respectueuses de la nature.  
       
        \begin{hyp}
        \label{prop:K}
            L'action militante et de lobbyisme des \eess est motivée par l'engagement et les convictions, et est favorisée par l'action collective et partenariale. 
        \end{hyp}
        
        Comme le montrent les exemples cités, l'action militante est à mettre en lien à des éco-innovations institutionnelles. Elle vise en effet à obtenir des changements à l'échelle d'un territoire, donc bien au delà du périmètre de l'organisation. Les évolutions obtenues bénéficient à l'ensemble de la collectivité, et non pas seulement aux intérêts privés de l'organisation. 
        
        L'exemple de la mise au point de cartouches en fer et acier par la Tour du Valat a ceci de particulier qu'une innovation technologique a donné lieu à une innovation institutionnelle. Une fois l'efficacité de ces munitions prouvée, et l'acceptation des chasseurs obtenue, la fondation a obtenu que les cartouches en plomb soient rendues illégales, réduisant ainsi le nombre de cas de saturnisme. 
        
        \begin{hyp}
        \label{prop:L}
            L'action militante et de lobbyisme des \eess est en lien avec la mise en place d'éco-innovations institutionnelles ou systémiques
        \end{hyp}
        
        L'action environnementale dans l'\ess répond à différentes motivations, et prend des formes multiples. Nous avons formulé douze propositions (encadré \ref{encadre:propositions} qui établissent un lien entre les leviers d'actions des \oess, leurs motivations, et le type d'innovations mises en place. Nous clôturons ce chapitre en proposant une synthèse des enseignements de cette étude de cas. 
        
        
        
        
        
        
        
        
    
   \begin{encadre}
        \fbox{
            \begin{minipage}{0.9\textwidth}
                \caption{Synthèse des propositions}
                \label{encadre:propositions}
               \textbf{Proposition \ref{prop:A}} - La limitation de l'impact environnemental est motivée par des facteurs externes ou de compétitivité.\\
               \textbf{Proposition \ref{prop:B}} - La limitation de l'impact environnemental s'appuie sur des éco-innovations organisationnelles, sociales et technologiques.\\
               \textbf{Proposition \ref{prop:C}} - Le développement de fonctionnements alternatifs est motivé par l'engagement personnel et collectif et la dimension stratégique donnée à l'action environnementale. \\
               \textbf{Proposition \ref{prop:D}} - Le développement de fonctionnements alternatifs s'appuie sur des éco-innovations technologiques et institutionnelles.\\
               \textbf{Proposition \ref{prop:E}} - Les actions d'informations et de sensibilisations sont motivées essentiellement par les facteurs engagement et stratégie.\\
               \textbf{Proposition \ref{prop:F}} - Les actions d'informations et de sensibilisations  s'appuient essentiellement sur des éco-innovations organisationnelles mais aussi sur des innovations technologiques.\\
               \textbf{Proposition \ref{prop:G}} - Les actions de partage des connaissances sont motivées essentiellement par les facteurs engagement et stratégie. Elles sont favorisées par les partenariats et le collectif.\\
               %\textbf{Proposition \ref{prop:H}} - Les actions de partage des connaissances s'appuient essentiellement sur des éco-innovations technologiques et organisationnelles.\\
               \textbf{Proposition \ref{prop:I}} - La création de synergies et l'initiation d'une dynamique sociétale sont motivées essentiellement par les facteurs engagement et stratégie. Elles sont favorisées par les partenariats et le collectif.\\
               \textbf{Proposition \ref{prop:J}} - La création de synergies et l'initiation d'une dynamique sociétale s'appuient essentiellement sur des éco-innovations sociales et institutionnelles.\\
               \textbf{Proposition \ref{prop:K}} - L'action militante et de lobbyisme des \eess est motivée par l'engagement et les convictions, et est favorisée par l'action collective et partenariale. \\
               \textbf{Proposition \ref{prop:L}} - L'action militante et de lobbyisme des \eess est en lien avec la mise en place d'éco-innovations institutionnelles ou systémiques.\\
            \end{minipage}
       }
    \end{encadre}
        