% !TEX root = /Admin/main.tex


\section{Déterminants de l'action environnementale}
\label{section:det_act_env}

Les entretiens réalisés ont permis de mettre en lumière des facteurs qui favorisent l'action environnementale. Il est intéressant de constater que certains de ces atouts peuvent parfois avoir des effets inverses et constituer des barrières à sa mise en oeuvre.

Nous voyons que l'action environnementale répond à des éléments externes, de contexte, à des convictions personnelles des personnes impliquées dans l'organisation, mais aussi dans certains cas à une dimension stratégique et à des aspects économiques. La présence de ces déterminants dans les organisations étudiées est détaillée dans le tableau \ref{table:determinantsEI}.

Dans l'étude de la littérature, nous avons présenté plusieurs typologies de déterminants de l'action environnementale. Cependant, des auteurs \parencite[par exemple][]{dart2010green} suggèrent que d'autres facteurs expliquent l'attitude des \oess  vis-à-vis de l'environnement. Pour cette raison, le codage est réalisé de manière exploratoire, en laissant la place à de nouvelles catégories. Les catégories de déterminants présentées ici ne sont donc pas issues de la littérature, mais bien de l'étude des cas. Nous discutons ensuite de la convergence avec les typologies basées sur le secteur privé lucratif.

\begin{table}[]

    \caption{Synthèse des déterminants de l'\ei dans les 7 organisations étudiées}
    \label{table:determinantsEI}
    \begin{tabularx}{\linewidth}{|L|L|c|c|c|c|c|c|c|}
    \hline
        Catégorie & Sous-catégorie & AP & AMS & GA &  EL & TdV & PF & BZ \\ \hline

        Facteurs externes & Contexte sociétal et technologique &
        + & + & & & & + &
        \\ \hline

        Facteurs externes & Légitimité et statuts de l'ESS &
        +/- &  & & & +/- &  &
        \\ \hline

        Facteurs externes & Contexte législatif &
        - &  & & & &  +  &
        \\ \hline

        Engagement personnel et collectif & Engagement individuel &
        - &  & & +/- & +/- &  +/-  & +
        \\ \hline

        Engagement personnel et collectif & Volonté collective &
        +/- &  &  & +/- & + & + & +
        \\ \hline

        Dimension stratégique & Place stratégique de l'action environnementale &
        + & +/- & +/- & + & + & + & +
        \\ \hline

        Dimension stratégique & Remise en cause des logiques de marché &
        + &  &  & +  & + &  & +
        \\ \hline

        Dimension économique & Financement &
        +/- & - & - &  & - & - & +/-
        \\ \hline

        Dimension économique & Compétitivité &
        & + & - &  &  &  &
        \\ \hline

        \multicolumn{2}{|l|}{Partenariats et collectif} &
        + & + &  & + & + & + & +
        \\ \hline

    \end{tabularx}
    \begin{footnotesize}
         + : effet positif \\
        - : effet négatif \\
        +/- : effet parfois positif, parfois négatif
    \end{footnotesize}

\end{table}

    \subsection{Facteurs externes}
    Nous présentons tout d'abord les facteurs externes qui impactent l'action environnementale dans les \oess.
        \subsubsection{Contexte sociétal et technologique}
        Trois répondants estiment que leur action environnementale s'inscrit dans un contexte sociétal plus général et en lien avec \citit{la montée en puissance de la conscience environnementale des gens} (Responsable Environnement, Coopérative Provence Forêt). Ceci peut se traduire par des demandes explicites ou bien simplement créer un contexte propice à l'action environnementale.  Le directeur d’Air PACA perçoit ainsi \citit{ des attentes aujourd’hui qui sont de plus en plus précises et nombreuses} et qui prennent pour son association la forme de demande d'interventions plus ou moins formelles. Pour l'un des co-président de l'association Bzzz, on est ainsi \citit{en train de surfer sur un moment d’actualité, un moment sociétal plutôt favorable et porteur}.

        Le contexte technologique est également un facteur d'innovation. Le secteur de la qualité de l'air par exemple, sur lequel se positionne Air PACA et qui a été longtemps laissé de côté, fait l'objet d'un intérêt croissant et de nombreuses recherches voient le jour. De nouveaux appareils de mesure se développent et de nouveaux modes d'exploitation des données émergent. L'association se positionne auprès de la communauté scientifique pour suivre ces évolutions.

        \subsubsection{Légitimité et statuts de l'\ess}
        L'appartenance à l'\ess est un facteur ambivalent sur l'action environnementale. Pour certains répondants, elle constitue un atout à travers l'indépendance qu'elle peut offrir. Cependant, cette indépendance connaît des limites et les statuts de l'\ess se heurtent parfois à un manque de reconnaissance et de légitimité.

        Pour le Directeur d'Air PACA, association agréée par l'État et qui remplit une mission de service public, comme pour celui de la fondation Tour du Valat qui intervient fréquemment auprès d'acteurs publics, leurs statuts respectifs assurent une forment d'indépendance et de liberté d'action. Malgré le contrôle exercé sur elle par l'État, Air PACA garde l'initiative des actions qu'elle mène. Cette indépendance s'accompagne également d'une plus grande réactivité, comme le rapporte le Directeur de la Tour du Valat. Une fondation échappe ainsi à la lourdeur administrative d'un organisme public, tout en oeuvrant pour l'intérêt général. Cependant, le constat inverse est fait par la Directrice de l'association Enfants et Loisirs. Certains projets de rénovation utilisant des processus écologiques n'ont pas pu être réalisés, les locaux étant la propriété de la commune. Ce qui est un atout (bénéficier de locaux mis à disposition gratuitement) s'accompagne d'une perte de contrôle qui impacte l'action environnementale. Air PACA s'est heurté à des difficulté similaires lors de la mise en place du processus de gestion des déchets, la Mairie n'offrant pas spontanément la possibilité de collecter le papier à recycler. Enfin la Tour du Valat comme Air PACA mentionnent des difficultés à exister institutionnellement, du fait même de leur statut d'organisme non lucratif. L'agrément public donne à Air PACA une légitimité supplémentaire par rapport à d'autres associations, mais lui donne moins de poids qu'une agence d'État. Elle doit ainsi justifier de son rôle d'intérêt général à échéance régulière pour obtenir les financements dont elle a besoin. Il en va de même pour la Tour du Valat donc l'action n'est pas automatiquement reconnue, malgré la reconnaissance scientifique dont elle bénéficie au niveau international.
        \begin{quotation}
            \citit{Je passe une énergie incroyable à exister institutionnellement, à nouer des relations avec les ministères, avec les institutions régionales, locales etc. sans quoi on n’a pas d’existence par nous-mêmes, on n’a pas d’existence institutionnelle par nous-mêmes, on ne l’a que par les liens qu’on arrive à entretenir} (Directeur, Fondation Tour du Valat)
        \end{quotation}
        Les partenariats noués avec différents organismes sont un moyen de pallier ces difficultés (voir section \ref{av_part}).

        \subsubsection{Contexte législatif}
        L'action environnementale s'inscrit également dans le contexte des lois et réglementations en vigueur. C'est particulièrement le cas pour Provence Forêt qui fait face à un contexte législatif complexe, les zones protégées (de type Natura 2000) étant nombreuses dans la région et pouvant se superposer. Pour la coopérative, la mise en place des procédures en vue de la certification ISO 14001 est un moyen de mieux respecter cette réglementation. Il y a donc une adéquation entre le respect de l'environnement et la prise en compte des contraintes législatives.
        \begin{quotation}
            \citit{On pourrait se dire que c’est simple de suivre la réglementation mais en fait elle est tellement compliquée que grâce à ISO 14001, il y a une procédure qu’on a, ça nous permet de passer tout en revue et de ne pas se tromper} (Responsable Environnement, Provence Forêt).
        \end{quotation}

        La législation, susceptible d'évoluer, peut à l'inverse constituer une contrainte. Air PACA, dont le conseil d'administration donne une place à des représentants de l'État, est dépendante de ces changements, souvent impulsés par l'Union Européenne. Le statut, le financement et la mission même de l'association sont soumis à la législation et l'action environnementale peut être interrompue ou modifiée sur une simple décision de l'État, ce qui fait peser une certaine incertitude sur les projets à moyen et long terme.

    \subsection{L'engagement personnel et collectif}
    Au-delà des facteurs externes, il ressort des entretiens le rôle essentiel de l'engagement des parties prenantes internes à l'organisation, aussi bien au sein de la gouvernance que dans les équipes salariées et bénévoles. La prise en compte de l'environnement est également perçue comme compatible et cohérente avec la mission sociale des organisations de l'\ess.

        \subsubsection{Dimension individuelle}
        Plusieurs répondants ont souligné le rôle de l'engagement personnel de personnes au sein de l'organisation pour impulser des innovations environnementale. Chez Enfants et Loisirs, le projet écolo-crèche qui structure aujourd'hui l'activité a été initialement portée par une directrice sensible à l'écologie. Le \cit{devoir} d'agir pour l'environnement et d'éduquer les enfants à l'écologie est désormais fortement ancré et inscrit dans les documents de l'association. Cette dynamique environnementale est également partagée par les équipes et les parents membres y ont été sensibilisés. Les convictions personnelles des salariés sont également un moteur de l'innovation environnementale à la Tour du Valat. Comme le note le directeur, \citit{les gens qui bossent à la Tour du Valat, la grande majorité ne sont pas là par hasard}, et ainsi sont volontaires pour s'impliquer dans les projets, voire les porter.

        A l'inverse, les comportements humains peuvent également constituer un frein à la mise en oeuvre des éco-innovations. Celles-ci s'accompagnent fréquemment d'une surcharge de travail pour les salariés et exigent une évolution des comportements, parfois mal comprise. Par exemple chez Air PACA, les économies demandées sur les consommations de papier ou la mise en place du recyclage ne correspondent pas toujours aux habitudes individuelles des salariés. Cependant, dans la plupart des cas, ces barrières ont pu être dépassées. L'enjeu est alors le maintien des bonnes pratiques dans le temps, l'engouement et la motivation pouvant s'estomper progressivement.

        \subsubsection{Dimension collective}

        Les valeurs partagées par les membres des organisations peuvent donner forme à une culture commune, favorable à l'action environnementale. Pour plusieurs répondants, cette dynamique entre en résonance avec les principes de l'ESS ou même du développement durable. En effet, certaines organisations parlent plutôt d'un ancrage dans le développement durable (Tour du Valat, AMS Environnement) ou l'économie circulaire (Bzzz) que \cit{d'économie sociale et solidaire}.

        La dimension collective de l'\ess est particulièrement incarnée par l'ouverture des organes de décision à des parties prenantes plus ou moins variées, dans lesquels la voix de chaque personne a le même poids. Le positionnement de ces organes vis-à-vis de l'action environnementale est déterminant. Chez Provence Forêt, la volonté d'adopter un fonctionnement durable et d'adopter des pratiques responsables est fortement portée par le conseil d'administration, c'est pourquoi elle occupe une place centrale dans l'activité. Cependant, l'hétérogénéité des parties prenantes ayant part aux décisions conduit également à des désaccords. Ceci est parfaitement illustré par le fonctionnement d'Air PACA, dont la gouvernance est composée à fois par des industriels, des acteurs publics et des associations de défense de l'environnement. Alors que certains souhaitent valoriser l'innovation autour de la qualité de l'air, d'autres poussent à un retour aux missions d'origine de l'association.

    \subsection{Dimension stratégique}
        \subsubsection{Place stratégique de l'action environnementale}
        Un élément déterminant de l'action environnementale, quasi-tautologique, est la place qu'occupe l'environnement dans la mission de l'organisation. Il va de soi que des organisations comme Air PACA ou la Tour du Valat sont plus sensibles aux enjeux environnementaux, dans la mesure où ceux-ci sont directement liés à leur activité. L'innovation environnementale sert directement l'organisation.
        Cependant, l'action environnementale peut se justifier même lorsque le lien est moins évident. La mission d'une crèche n'a a priori aucun lien avec l'écologie, cependant l'association Enfants et Loisirs y a trouvé une véritable cohérence avec sa mission de soin et de pédagogie. Réduire l'utilisation de produits chimique est bénéfique pour l'environnement, mais aussi pour la santé des enfants. Les activités autour du recyclage ou de la réutilisation peuvent également avoir un véritable intérêt pédagogique. Alors que certaines actions mises en place n'ont pas pu être maintenues dans la durée, celles qui avaient une réelle cohérence avec le coeur de l'activité, c'est-à-dire qui étaient bénéfiques pour les enfants, ont perduré et rencontré une forte adhésion du personnel et des parents. De la même manière, la priorité d'AMS environnement est l'insertion de publics en difficulté. La dimension environnementale présente un intérêt car elle est valorisante, alors qu'un regard négatif est souvent porté sur l'insertion. Elle constitue ainsi un facteur motivant pour les bénéficiaires.

        Néanmoins, l'action environnementale a tendance à être laissée de côté lorsqu'elle ne trouve pas sa place dans l'activité de l'entreprise. Certaines organisations sont tournées en premier lieu vers une mission économique ou sociale. Pour le groupe Arborescence, qui revendique un fonctionnement similaire à celui d'une entreprise classique, l'action environnementale ne peut se justifier que par des opportunités de développement. C'est pourquoi le groupe s'est agrandi d'une société spécialisée dans le lavage écologique d'automobiles. Cependant, la démarche environnementale n'est pas allée plus loin, car les opportunités liées à l'action environnementale sont trop peu nombreuses et constituent une niche déjà très occupée par d'autres acteurs du secteur. Le même constat est partagé par AMS Environnement: si la direction estime que l'écologie pourrait être une direction future pour l'organisation, elle présente encore trop peu d'opportunités économiques :
        \begin{quotation}
            \citit{Le problème c’est qu’aujourd’hui y’a pas suffisamment d’activités où on peut revaloriser nos 30 \% de financements. Il y a peu de besoins écologiques aujourd’hui. Il faut les trouver, il faut les négocier, il faut se battre, il faut être en concurrence avec d’autres structures d’insertion.} (Directeur, Association AMS Environnement).
        \end{quotation}

        \subsubsection{Remise en cause des logiques de marché}
        \label{subsec:contestation_marche}

        Les statuts particuliers de l'\ess permettent aux organisations de se soustraire des logiques purement utilitaristes, voire de remettre en question les schémas de production dominants. Plusieurs des cas illustrent l'effet sur l'action environnementale à des degrés différents. Tout d'abord, les statuts permettent à certaines organisations d'échapper aux mécanismes de concurrence, comme c'est le cas pour Enfants et Loisirs (seule organisation agréée pour l'activité de crèche dans la commune) ou Air PACA (qui bénéficie d'une forme de délégation de service public exclusive). Une forme de concurrence peut apparaître sous des formes plus subtiles, mais de façon moins nettement moins forte que dans le cas des deux structures d'insertion (AMS Environnement et le Groupe Arborescence) ou encore de la coopérative Provence Forêt. Il est donc plus aisé pour les entreprises qui échappent aux pures logiques de compétition de consacrer plus d'efforts à des actions environnementales, y compris lorsque celles-ci ne sont pas rentables ou moins rentables que d'autres investissements. Chez Enfants et Loisir, un certain surcoût a été accepté par les parents et les financeurs publics, comme dans le cas du recours à une alimentation biologique. L'absence de concurrence directe est également bénéfique pour la diffusion des informations relatives à l'action environnementale, comme l'illustre le cas d'Air PACA, qui rend publique et en temps réel une quantité d'information sur la qualité de l'air, utilisable par des acteurs privés à but lucratif. Ce partage d'information, bénéfique pour la société, n'est pas dommageable pour Air PACA puisqu'il s'agit de sa mission même, là où une structure à but lucratif aurait tout intérêt à garder la propriété de ces données.

        Un cas plus marqué de remise en question des logiques concurrentielles est celui de Bzzz. La mission de l'association est justement de proposer un mode de fonctionnement alternatif, faisant passer la dimension écologique et humaine avant la dimension économique. Chez Bzzz, cet aspect est décliné à de multiples niveaux, comme par exemple sur le plan managérial.
        \begin{quotation}
            \citit{On a une forme horizontale avec une logique de collaboration avec notre force de travail qui sont nos salariés. Le conseil collégial est patron de ses salariés et en même temps je pense qu’on est plutôt des patrons… des patrons différents.}  (Co-président, Association Bzzz).
        \end{quotation}
        Cela se manifeste également au niveau des mode de production : dans l'optique de \citit{ne pas stresser les abeilles [...] ne pas traiter avec des produits chimiques, ne pas voler la totalité du miel}, l'association se limite à 50 ruches, quand une exploitation "normale" en compte 200 par ETP.
        Au niveau de la distribution, l'organisation favorise la logique de circuits-courts, limite le recours aux intermédiaires, allant jusqu'à proposer de \citit{sortir du système économique et monétaire} tout en prônant \citit{la bienveillance, la transmission, la communication non-violente, la lenteur} (Site internet).

        L'\ess semble ainsi en mesure de proposer des alternatives au système économique, alternatives qui peuvent s'avérer bénéfiques pour l'environnement. Cette démarche n'est cependant pas systématique. Certaines organisations choisissent au contraire de s'inscrire pleinement dans les schémas économiques classiques, comme c'est le cas pour le Groupe Arborescence. De plus, les contraintes économiques sont omniprésentes et limitent la marge de manoeuvre des \oess.

    \subsection{Dimension économique}
        \subsubsection{Question du financement}
        \label{paragraphe:det_fin}

        L'étude des cas montre que le financement des \oess est une difficulté permanente. Certains dispositifs dédiés à ces structures peuvent néanmoins être mobilisés.

        Six des sept cas étudiés présentent des difficultés à financer leur activité, ce qui conduit à une incertitude sur l'avenir de l'organisation. La question du financement concerne tous les statuts. Pour plusieurs répondants, l'accès aux financements constitue une charge de travail importante et réduit le temps consacré à l'activité, comme l'illustrent les verbatims suivants.

        \begin{quotation}
            \citit{Chacun doit trouver de l’argent dans son domaine, y compris des chercheurs dont le métier n’est pas de chercher l’argent, c’est de chercher, mais néanmoins ici le chercheur il va aussi chercher de  l’argent. Et c’est vrai que trouver 35 \% du budget global c’est une charge, c’est une pression sur les personnes qui est importante } (Directeur, Fondation Tour du Valat). \\
            \citit{Et en fait on se rend compte que c’est toujours impossible, il nous faudra toujours une subvention pour survivre. Et aujourd’hui c’est le gros problème c’est est-ce qu’on va avoir la subvention en 2018 : on ne sait pas ! } (Responsable Environnement, Provence Forêt). \\
            \citit{Ça me dérange un peu, c’est-à-dire que moi j’ai toujours l’impression [...] de passer un temps de rédaction de dossiers, de toujours montrer patte blanche alors qu’on essaye d’être le plus transparent possible. C’est des choses qui prennent énormément de temps, énormément d’énergie.} (Co-président, Association Bzzz). \\
        \end{quotation}

        En outre, l'orientation des organisations vers des problématiques sociales spécifiques impacte négativement la génération de revenus et rend les \oess moins compétitives, comme le remarque le directeur d'AMS Environnement : \citit{du moment vous mettez une plus-value humaine, on ne peut pas être moins chers qu’une autre structure.}

        En revanche, l'économie sociale bénéficie de sources de financements dédiées, souvent orientées vers l'innovation et qui peuvent soutenir l'action environnementale. Les organisations peuvent avoir recours à des dons (Association Bzzz) ou à différents types de subventions ou financements publics (ensemble des organisations du panel). Bzzz recourt également au financement participatif (crowd-funding) pour financer certains projets et répond à des appels à projets consacrés à l'innovation sociale et environnementale. Ces alternatives aux modalités classiques permettent de diversifier les sources de financement, mais demandent aussi un investissement humain important.

        Finalement, la difficulté à financer l'activité conduit les organisations à repenser leurs modèles, parfois au détriment de l'action environnementale :
        \begin{quotation}
            \citit{Régulièrement on fait des arbitrages entre nos ambitions et puis après si on voit qu’on n’a pas les moyens, pas les aides pour nos ambitions, et bien on révise à la baisse nos ambitions tout en essayant d’être cohérents et qu’à terme le retour sur investissement soit pertinent.} (Directeur, Fondation Tour du Valat). \\
            \citit{Nous on a un problème d’argent, on n’a pas assez d’argent pour mettre en place des choses [...]. Nous on a pas mal d’idées mais en fait, très honnêtement, on n’arrive pas à mettre en place.} (Responsable Environnement, Provence Forêt).
        \end{quotation}



        \subsubsection{Recherche de compétitivité}

        La visée sociale plutôt que financière de l'\ess peut constituer un facteur favorisant l'action environnementale, comme indiqué dans la section \ref{subsec:contestation_marche}. Pour autant, certaines organisations s'inscrivent pleinement dans les logiques de l'économie dominante. Le gain de rentabilité potentiel encourage alors l'action environnementale. Les dirigeants constatent que certaines pratiques ont à la fois un coût \citit{environnemental et économique} (Directeur, Association Air PACA). À un niveau plus stratégique, l'écologie ouvre aussi des perspectives de développement et de différenciation par rapport aux concurrents qui intéressent les \oess. Le directeur d'AMS Environnement constate ainsi :
        \begin{quotation}
            \citit{L’écologie c’est une stratégie, oui c’est une stratégie pour durer, parce que je pense que à échéance, on sera obligé d’admettre qu’il va falloir beaucoup de projets écologiques [..]. Notre vision peut être intéressante et stratégiquement on a choisi d’aller au milieu de gens qui réfléchissent sur l’environnement.}
        \end{quotation}

        La limite à l'effet de la recherche de compétitivité sur l'action environnementale se présente dès lors que les dirigeants ne décèlent pas d'opportunités liées à l'écologie. C'est par exemple la raison pour laquelle le groupe Arborescence ne s'oriente pas dans cette direction :
        \begin{quotation}
            \citit{Aujourd’hui, c’est un petit peu les niches, sur les activités de réemploi et tout ça. [...] Il n’y a pas un objectif précis d’aller dans ce domaine d’activité. D’abord, on n’a pas les moyens, et puis il y a quand même pas mal de structures qui y sont déjà aujourd’hui.}
        \end{quotation}

    \subsection{Rôle des partenariats et du collectif}
        \label{av_part}

    La dimension collective des organisations joue un rôle particulier pour l'action environnementale. Elle n'agit pas nécessairement comme un déterminant, ni comme un frein, mais plutôt comme un facilitateur. Plusieurs situations rencontrées par les entreprises de l'échantillon illustrent la contribution d'une approche collective à l'innovation en matière d'environnement.

    Pour certaines organisations, une approche collective est un moyen d'avoir une vision plus large de l'activité environnementale et de lui donner une cohérence générale.
    \begin{quotation}
        \citit{On est dans un métier où les choses vont vite, nous on s’occupe de l’air mais d’autres vont s’occuper de l’eau, des sols, donc on a tout un réseautage qui est nécessaire pour donner une vision d'ensemble de ces questions. } (Directeur, Air Paca).
    \end{quotation}
    De la même façon, s'inscrire dans un cadre collectif permet de mettre en pratique les innovations développées, de faire le lien avec l'action et de diffuser les connaissances. Pour la Tour du Valat par exemple, les partenariats sont indispensables pour insuffler des changements de pratiques à l'échelle internationale. En agissant seule, l'organisation n'a pas la capacité d'être présente sur tous les projets autour des zones humides. Fonctionner en partenariat avec d'autres organisations plutôt que de les voir comme des compétiteurs permet de donner plus de poids aux \eess. C'est en agissant de manière conjointe avec d'autres organisations militantes que Bzzz a pu obtenir des changements de politiques locales en matière d'environnement. \\

    L'action collective facilite également l'accès à des ressources matérielles, humaines ou financières, qui peuvent parfois manquer aux organisations. Des activités de sensibilisation menées par l'association Enfants et Loisirs ont été rendues possible par l'engagement des parents des enfants accueillis par la crèche. L'association Bzzz s'est appuyée sur plusieurs associations partenaires pour achever son projet de miellerie mobile. Ainsi, la Caravanade a fourni des outils et un accompagnement technique.

    Les organisations qui s'inscrivent dans ce type de démarche peuvent aussi bénéficier d'une entraide et d'une mise en commun des connaissances. En passant la certification PEFC dans le cadre d'un groupe de coopératives, Provence Forêt a ainsi profité de l'expérience des autres sociétés engagées dans la démarche. En outre, se sont créées des synergies, chaque coopérative bénéficiant des retours des autres. En outre, alors que Provence Forêt dispose de peu de moyens pour innover, elle peut s'appuyer sur ce qui est fait dans d'autres organisations de ce groupe. \\

    Le collectif se manifeste aussi au niveau de la gouvernance. AMS Environnement souligne ainsi l'intérêt d'avoir des personnes spécialistes des questions environnementales dans son Conseil d'administration, qui peuvent apporter leur expertise et accompagner l'organisation dans une démarche d'innovation. C'est également le cas pour la Tour du Valat qui, en faisant intervenir des élus dans sa gouvernance, se dote d'un soutien institutionnel qui renforce ses possibilités d'agir pour mener à bien sa mission environnementale. \\

    Enfin, pour Provence Forêt, la Tour du Valat ou l'association Bzzz, l'action partenariale constitue aussi une opportunité de réduire les coûts, voire d'obtenir certains financements. La Tour du Valat s'appuie en majorité sur son capital ainsi que sur la participation des autres fondations de son créateur. Toutefois, ces revenus sont insuffisants et la structure doit diversifier les financements. Les conventions de partenariats peuvent apporter une source de revenus complémentaires, tout en permettant la diffusion des actions environnementales de la fondation. Bzzz s'appuie sur la participation du public pour financer certains projets, en recourant à des financements participatifs (crowd-funding). Là encore, l'obtention de revenus permet en même temps de faire connaître l'organisation, ses activités et le modèle alternatif qu'elle propose, tout en permettant aux personnes de s'impliquer dans l'association. De façon plus directe, Bzzz reçoit également des financements de partenaires associatifs. La \cit{Bzzz-mobile} a ainsi été en partie financée par un don des Fonds Epicuriens.

\section{Discussion}
% \reynaud{À mon avis cette partie est un peu plus faible vous auriez dû lire des livres par exemple Mile et Huberman il vous permettrait davantage de justifier votre démarche je pense que la démarche est bonne je la trouve simplement insuffisamment justifiée.

% Pour conclure pourriez-vous dessiner le modèle de Batsal Roth en montrant les spécificités de l' ESS avec un encadré pour la compétitivité un autre pour la responsabilité...}

\label{section:conclu_quali}
    L'étude de cas présentée dans ce chapitre met en lumière la grande diversité des approches de l'action environnementale dans les \oess. Elles réagissent à différents déterminants et adoptent de stratégies environnementales plus ou moins pro-actives. L'étude aboutit à huit propositions, soulignant le lien entre les identités organisationnelles et l'action environnementale. \\

    La section finale de ce chapitre est consacrée à discuter des résultats de l'étude en regard des modèles développés dans la littérature. Nous montrons que les leviers d'action environnementale mobilisés et l'ampleur des efforts fournis s'expliquent par les motivations initiales des entreprises. Nous discutons également d'un élément essentiel pour comprendre l'action environnementale dans les \eess : le positionnement des organisations dans une posture concurrentielle. Alors qu'existe une tendance sociétale forte pour l'\ess à s'inscrire dans le marché et à se montrer compétitive, cette dynamique est plutôt défavorable à une action environnementale forte.


    \subsection{Déterminants et identité organisationnelle}

        Le modèle de \textcite{bansal2000why} offre une grille de lecture intéressante de nos résultats. Les auteurs distinguent trois déterminants de l'action environnementale : la compétitivité, la légitimité et la responsabilité. Le facteur \textbf{compétitivité} désigne la profitabilité attendue de l'action environnementale. les entreprises qui s'inscrivent dans cette démarche cherchent à obtenir des avantages compétitifs, à travers la valeur ajoutée apportée à leurs produits ou à travers de possibles réductions de coûts. Ce facteur ressort dans plusieurs cas étudiés. Il est notamment présent, assez logiquement, dans les organisations ayant une forte identité utilitariste, celles qui se comparent explicitement à des entreprises classiques. Pour celles-ci, la recherche de performance est naturelle, non seulement dans l'action environnementale, mais dans tous les aspects de leur activité. Cependant, la compétitivité environnementale peut aussi être contrainte et répondre aux pressions financières et concurrentielles que rencontrent l'\ess... et qui façonnent aussi l'identité organisationnelle. Les organisations les plus utilitaristes du panel sont aussi celles qui ressentent une concurrence très importante.  En outre, l'effet de ce facteur est limité, conformément à ce que suggèrent \textcite{dart2010green}. Les répondants en recherche de compétitivité identifient peu d'opportunités dans l'action environnementale. Les organisations qui recherchent véritablement un gain de compétitivité en matière d'écologie sont celles qui sont déjà engagées dans l'action environnementale, souvent en lien avec leur coeur de métier.

        \begin{hyp}
        \label{prop:DET1a}
            La recherche de compétitivité est favorable à l'action environnementale dans l'\ess.
        \end{hyp}

        \begin{hyp}
        \label{prop:DET1b}
            La présence d'une identité utilitariste modère positivement l'effet de la recherche de compétitivité sur l'action environnementale.
        \end{hyp}

        \begin{hyp}
        \label{prop:DET1c}
            La difficulté à identifier des opportunités de développement dans l'action environnementale modère négativement l'effet du facteur compétitivité sur l'action environnementale.
        \end{hyp}

        La \textbf{légitimité} est définie par \textcite{suchman1995managing} comme \citit{une perception ou une supposition généralisée que les actions d’une entité sont désirables, convenables ou appropriées au sein d’un système socialement construit de normes, de valeurs, de croyances ou de définitions} (p. 574). Dans l'approche de \textcite{bansal2000why}, le facteur légitimité vise à maintenir la pérennité de l'entreprise en assurant la conformité avec la législation applicable et en maintenant une image en adéquation avec les attentes des parties prenantes concernées. Les résultats vérifient l'effet de la recherche de légitimité sur l'action environnementale dans les \eess. Cette dimension correspond aux facteurs externes identifiés à travers l'étude de cas. Le souci des questions d'image et de perception de l'entreprise par les parties prenantes renvoient à l'identité utilitariste. L'action environnementale vise à pérenniser l'entreprise et à maintenir son activité en répondant aux attentes des clients et des autres parties prenantes. Par ailleurs, plusieurs organisations marquées par une identité fonctionnelle liée à l'environnement soulignent le besoin de cohérence entre leur engagement et leurs actions. En effet, la légitimité d'une organisation à mission environnementale pourrait être contestée si elle même n'adoptait pas des pratiques respectueuses de la nature.

        \begin{hyp}
        \label{prop:DET2a}
            La recherche de légitimité est favorable à l'action environnementale.
        \end{hyp}

        \begin{hyp}
        \label{prop:DET2b}
            L'identité utilitariste modère positivement la relation entre légitimité et action environnementale.
        \end{hyp}

        \begin{hyp}
        \label{prop:DET2c}
            L'identité fonctionnelle modère positivement la relation entre légitimité et action environnementale si la mission est liée à la préservation de l'environnement.
        \end{hyp}

        Enfin, Bansal et Roth identifient une dimension de \textbf{responsabilité} qui désigne une démarche moins pragmatique, mais motivée plutôt par des valeurs et un souci de l'intérêt général. La responsabilité correspond aussi au devoir moral de l'organisation vis-à-vis de la société.
        Ce facteur renvoie directement à la notion d'engagement personnel et collectif mis en évidence par l'étude de cas. Il est à mettre en lien avec l'identité normative des organisations. L'action environnementale des \oess est ici motivée par des convictions portées par des dirigeants ou bien collectivement partagées par l'ensemble des équipes. Le facteur responsabilité a un effet d'autant plus important lorsqu'il est à l'origine même de la création de l'organisation. L'action environnementale n'est alors plus un devoir dont l'accomplissement procure une satisfaction, mais répond à une volonté profonde des personnes. La force de ce déterminant tient dans son caractère incontournable : les organisations centrées sur des valeurs écologiques se montrent très éco-innovantes, y compris lorsqu'elle font face à des barrières importantes. Ceci va dans le sens des résultats de \textcite{marin2015smes}. A partir d'une analyse typologique, les auteurs observent en effet que certaines entreprises (\citit{Revealed barriers cluster}) estiment faire face à des obstacles conséquents à l'éco-innovation, mais s'appuient sur leur expérience pour maintenir des investissements importants. Notre étude montre que l'attachement individuel et collectif à l'environnement est un élément important pour mettre en place des éco-innovations malgré les difficultés rencontrées. L'identité normative est donc un élément favorable à l'action environnementale, à condition que l'écologie soit identifiée parmi les valeurs de l'organisation. \\

        \begin{hyp}
        \label{prop:DET3a}
            L'identité normative est favorable à l'action environnementale si et seulement si l'écologie est identifiée comme une valeur dans l'organisation.
        \end{hyp}

        % \todo[inline]{See recent publications of Bansal}

        Notre étude met également en évidence le rôle de la coopération et du collectif dans l'action environnementale. Peu évoqué dans la littérature sur l'éco-innovation, ce facteur est central dans l'ESS, en particulier, par définition, dans le mouvement coopératif et mutualiste. La gestion démocratique, l'ouverture de la gouvernance à des parties prenantes variées et l'action collective sont des valeurs centrales des \oess. Pour certains auteurs, c'est même la dimension démocratique de l'\ess qui constitue sa réelle spécificité (plutôt que l'aspect non lucratif) \parencite{mccambridge2004underestimating}. Petrella et Richez suggèrent ainsi que le principal enjeu de l'entrepreneuriat social, courant récent de l'ESS, est d'intégrer cette dimension :
        \begin{quotation}
            \citit{ Indeed, if social entrepreneurship is seen as a private innovative solution to new societal challenges unmet by the state nor the market through an original way of combining resources, no alternative model is emerging. But, if social entrepreneurship is led by participative and democratic governance processes that imply a diversity of stakeholders and resources, it can be seen as a building block for an alternative model. } \parencite{petrella2014social}
        \end{quotation}

        La dimension collective est présente dans la plupart des cas étudiés et est fortement valorisée par la majorité des répondants. Elle permet aux organisations d'accéder plus facilement à des financements, à des compétences et à des ressources matérielles et humaines nécessaires à l'action environnementale. Elle joue également un rôle déterminant dans la diffusion de l'innovation environnementale. A travers des partenariats, les organisations obtiennent la légitimité pour promouvoir leurs actions et peuvent accéder à une audience plus importante. L'ESS, historiquement organisée autour d'une approche collective, peut donc jouer un rôle significatif dans la perspective d'un changement d'échelle de l'action environnementale et dans l'optique d'avoir un impact conséquent dans la lutte contre le réchauffement climatique et la préservation de la biodiversité.

         \begin{hyp}
        \label{prop:DET4a}
            L'identité collective est favorable à l'action environnementale si et seulement si les parties prenantes sont sensibles à l'écologie.
        \end{hyp}

\begin{landscape}
    \begin{table}[h]
    \caption{Déterminants de l'action environnementale et identité organisationnelle dans l'\ess}
        \label{table:syntheseDeterminantsIdentite}
        \centering \small
        \begin{tabularx}{\linewidth}{|X|X|X|K{0.5\linewidth}|}
            \hline
             \textbf{Déterminants (modèle de Bansal et Roth)} & \textbf{Facteurs identifiés dans l'étude} & \textbf{Identités associées}  & \textbf{Spécificités de l'\ess}\\ \hline

            Légitimité
            &  Facteurs externes
            & - Utilitariste \newline - fonctionnelle
            & Le facteur légitimité se manifeste par l'attention portée aux parties prenantes. Elle traduit d'une part un souci d'image, d'autre part une volonté d'afficher une cohérence entre l'activité, les valeurs prônées et les actes.
            \\ \hline

            Responsabilité
            & Engagement personnel et collectif
            & - Normative \newline - collective
            & La notion de responsabilité, de rôle éthique à jouer et même de devoir envers l'environnement est présente dans l'ESS. Mais cette dimension doit être complétée par une démarche volontaire, pro-active, répondant à un engagement fort, parfois militant.
            \\ \hline

            Compétitivité
            & Dimension stratégique et dimension économique
            & - Fonctionnelle \newline - utilitariste \newline - normative
            & Le facteur compétitivité dans l'\ess revêt un double aspect. D'un côté, il correspond à l'approche concurrentielle classique, la volonté de croissance et de développement ou simplement de pérennité économique. De l'autre, il répond à une contrainte, face à des difficultés financières. Confrontés à une réduction de leurs financements habituels, les organisations n'ont d'autres choix que de jouer le jeu de la concurrence, au sein de l'\ess mais aussi face au secteur marchand.
            \\ \hline

            & Dimension collective
            & - Collective
            & Ce facteur est nouveau par rapport au modèle. Les cas mettent en évidence l'importance de la coopération pour l'action environnementale. Elle permet notamment de pallier un manque de ressources humaines ou financières.
            \\ \hline

        \end{tabularx}
    \end{table}
\end{landscape}

    \subsection{Impact sur l'éco-innovation}
        Les différents mécanismes discutés dans la section précédente aboutissent à la réalisation d'innovations environnementales dans les organisations étudiées. Les différents types d'éco-innovations identifiés dans la littérature sont représentés dans les \oess. Les organisations développent ou adoptent de nouvelles technologies ou de nouveau procédés afin de réduire leur impact environnemental. Elles mettent également en oeuvre des changements organisationnels et encouragent aussi des changements de comportements des salariés ou des publics cibles (éco-innovations sociales). Certaines innovations dépassent le cadre de l'organisation et prennent la forme d'éco-innovations institutionnelles \parencite{rennings2000redefining}. Si elles sont plus délicates à identifier, on peut toutefois relever des exemples d'innovations \cit{green system} \parencite{arundel2009measuring} qui reposent sur le développement de systèmes de production alternatifs. L'\ess agit donc sur l'ensemble des volets de l'éco-innovation.\\

        Parmi les nombreux exemples d'innovations observés, certains revêtent une portée apparemment plus significative que d'autres. Il est dès lors pertinent de se demander si l'\ess contribue vraiment à l'\ei en apportant des idées nouvelles, répondant ainsi à l'ambition de mouvement de transformation de la société, ou si elle se contente de suivre un mouvement général. \textcite{johannessen2001innovation}, parlant de l'innovation en général, se demandent ce qui est nouveau, à quel point cela est nouveau, et pour qui cela est nouveau. En effet, les exemples d'innovations radicalement nouvelles sont plus rares que ceux d'évolutions progressives, adaptées de procédés existants. Ce qui constitue un changement profond dans une organisation peut être déjà appliquée depuis longtemps dans d'autres structures. Pour les auteurs, ces trois questions aboutissent en réalité à un construit unidimensionnel, qui porte sur la radicalité de l'innovation. Les \oess étudiées mettant en place les innovations les plus radicales sont celles qui sont animées par un engagement individuel et collectif, et notamment lorsque cet engagement est inscrit dans la mission de l'organisation. Par innovations plus radicales, on entend ici celles qui sont nouvelles non seulement à leur échelle, mais aussi d'un point de vue plus général ou du point de vue d'autres organisations, voire qui aboutissent à une remise en cause du système. A l'inverse, les \eess qui ne sont pas prioritairement animées par la cause environnementale, mais par d'autres enjeux (par exemples sociaux), sont plus susceptibles de se tourner vers des innovations moins radicales. Les expériences menées ailleurs, leur adoption par de nombreuses organisations voire l'existence de services publics (par exemple pour le recyclage) permettent une mise en oeuvre facilitée, nécessitant d'y consacrer moins d'efforts et de ressources. \\

        Finalement, un élément émergent de l'étude de cas est la volonté de sensibilisation aux enjeux environnementaux, du partage des connaissances et de la diffusion de l'éco-innovation. A nouveau, cette démarche de diffusion de l'action environnementale est particulièrement saillante chez les organisations fortement sensibles à l'écologie du point de vue des valeurs. Ce phénomène montre que les organisations ne souhaitent pas seulement limiter leur propre impact sur l'environnement, mais aussi généraliser les bonnes pratiques afin d'obtenir des effets globaux. Ceci prend la forme de sensibilisation du grand public et des acteurs concernés, de partenariats permettant de reproduire des innovations développées localement dans d'autres contextes ou sur d'autres territoires. Les \oess, grâce à leur dimension ouverte, la possibilité de faire coexister des parties prenantes variées, ainsi que leur proximité avec la société civile, ont la capacité de créer des synergies et d'insuffler un mouvement généralisé autour des enjeux écologiques. Cependant, cette vision d'une ESS ancrée dans la société, ou \cit{encastrée} selon l'expression de Polanyi, est menacée par la domination écrasante des logiques capitalistes et concurrentielles.

    \subsection{Impact de la concurrence}

        La place de l'\ess dans l'économie fait l'objet d'un véritable débat dans la sphère académique, mais plus largement dans la sphère économique. Doit-elle s'inscrire dans le cadre de l'économie de marché et en accepter le principe de concurrence, ou bien constituer une alternative au modèle dominant ? Notre étude confronte des approches très différentes, parfois choisies, mais parfois contraintes. Les \eess font face à des situations variées. Certaines sont de fait en concurrence directe avec le secteur privé lucratif et n'ont pas d'autre choix que d'être concurrentielles. Les deux cas d'entreprises du secteur de l'insertion, réputé difficile et concurrentiel, en sont d'excellents exemples. Au contraire, certaines organisations, notamment sous format associatif, échappent à cette concurrence en raison de leur statut, par exemple parce qu'elles bénéficient d'un agrément qui leur confère un certain monopole local. Enfin, des organisations choisissent de s'extraire des logiques de marché, au risque de devoir affronter des difficultés supplémentaires et parfois de limiter leurs ambitions. La tendance générale, relevée par plusieurs auteurs, va dans le sens de l'adoption des logiques de marché dans l'ESS. Or ceci est un sujet d'inquiétude pour les organisations, qui voient dans ce mouvement une remise en question de leur capacité d'action, notamment en matière environnementale. \\

        Il apparaît que les \oess qui s'inscrivent dans des logiques de compétitions sont moins éco-innovantes que celles qui échappent à la concurrence. En effet, confrontées à la pression du marché, les entreprises privilégient les éco-innovations qui contribuent à leur positionnement stratégique, mais  y renoncent dès lors qu'elles ne perçoivent pas d'opportunité de développement. Or, les répondants concernés voient plutôt dans l'écologie une économie de niche ou bien des opportunités futures encore peu accessibles.
        Les résultats sont conformes à ceux obtenus par \textcite{mathieu2015les}. Pour les auteurs, les organisations s'inscrivant dans un référentiel financier adoptent plus souvent des stratégies adaptatives, c'est-à-dire plutôt un positionnement de suiveur, quand les entreprises ayant un référentiel durable s'inscrivent dans une stratégie proactive. L'étude menée dans le présent chapitre aboutit à un constat similaire : les \eess dont l'action est motivée par l'engagement individuel et collectif sont plus éco-innovantes que les organisations motivées par la recherche de compétitivité, et sont plus aptes à surmonter les barrières rencontrées. Ceci amène à s'interroger sur l'orientation générale de l'\ess et son rapprochement avec le système capitaliste. Si cette dynamique est jugée plus efficiente en termes de création de valeur par les tenants du libéralisme économique, peut-on espérer la même efficacité en matière environnementale ? \\


        %==)> ce lien dépasse en réalité les questions de financement !!! Mpeme les orgas en difficultés parviennent à éco-innover si elles échappent à ces logiques. Mais celles qui sont sous cette pression ne peuvent pas.

        % \section*{idées en vrac}



        %     Chang et Chen (2013) identifient un lien entre l’identité organisationnelle environnementale et l’éco-innovation

        %     Chang et Chen (2013) : la performance en termes d’éco-innovation des entreprises ayant une identité organisationnelle environnementale s’explique en partie par le gain de légitimité environnementale qu’elles en retirent.
