% !TEX root = /Admin/main.tex


\section*{Introduction du chapitre}


Le premier volet de notre problématique concerne la communication et la rhétorique environnementale. L'étude présentée dans le chapitre \ref{chapitre:twitter} montre que différentes stratégies de communication sont adoptées par les \oess et que les problèmes environnementaux peuvent être perçus comme des menaces mais aussi comme des opportunités de développement économique et d'innovation. Au-delà du discours, comment agissent les entreprises de l'\ess face à ces enjeux ? \\

 Cette seconde étude cherche à caractériser les déterminants de l'action environnementale. Elle prend la forme d'une recherche qualitative et porte sur sept cas d'\eess, étudiés sur la base d'entretiens semi-directifs, ainsi que sur des documents relatifs à ces organisations. L'étude a pour objectif de trouver des liens entre les caractéristiques des organisations et leur approche de l'action environnementale. Afin de mieux appréhender les spécificités des organisations, nous mobilisons le concept d'identité organisationnelle. Nous avons identifié dans la littérature quatre identités distinctes : \\
 \begin{itemize}
     \item L'identité utilitariste, ou la propension à adopter des pratiques de gestion de l'économie classique et à favoriser des revenus marchands.
     \item L'identité normative, qui se manifeste par l'adhésion au cadre institutionnel de l'ESS, à ses principes et à ses valeurs.
     \item L'identité collective, qui peut se matérialiser dans la gouvernance, dans la stratégie partenariale ou dans l'ouverture des activités à de parties prenantes diverses.
     \item L'identité fonctionnelle, c'est-à-dire l'accent mis sur la mission sociale ou environnementale de l'organisation, indépendamment des modalités pratique de poursuite de cette mission.\\
 \end{itemize}

 Les \eess sont souvent caractérisées par leur hybridité, c'est-à-dire que plusieurs identités organisationnelles, parfois contradictoires en apparence, peuvent cohabiter. Le recours au concept d'identité plutôt que le recours à une catégorisation exclusive telle que l'approche \cit{Marché VS Valeurs} permet justement de rendre compte de la complexité des organisatons et des ambivalences qui peuvent exister en leur sein. \\

 Plusieurs modèles et typologies peuvent être exploités pour caractériser l'action environnementale et ses déterminants. Cependant, les réflexions de \textcite{dart2010green}
  soulèvent des interrogations quant à la validité de ces modèles pour l'ESS. En effet, ses organisations n'ont pas la même finalité que les entreprises du secteur marchand : elles poursuivent une mission d'intérêt général au lieu de rechercher la maximisation des profits. En conséquence, on peut s'attendre à ce que leurs motivations à mener une action environnementale et les modalités d'action soient également différentes. Pour cette raison, nous ne posons pas \textit{a priori} un cadre d'analyse strict de l'action environnementale. Les résultats correspondant sont donc des résultats extraits directement du terrain. Ils sont ensuite confrontés à un modèle afin de relever les aspects communs et les différences avec les entreprises classiques. \\


 Dans un premier temps, les sept cas d'études sont présentés successivement, à l'aide des quatre identités développées dans la littérature. Dans un second temps, nous présentons les déterminants et barrières à l'action environnementale dans ces organisations (section \ref{section:det_act_env}). Enfin, nous terminons par une synthèse des enseignements de cette étude de cas (section \ref{section:conclu_quali}).
