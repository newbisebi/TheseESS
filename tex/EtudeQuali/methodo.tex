Le premier volet de la thèse concerne l'étude du discours environnemental des \eess. Celle-ci est importante car la communication permet d'adresser les demandes des parties prenantes. En outre, elle semble causer un effet d'entraînement sur les actions concrètes de protection de l'environnement \parencite{frandsen2011rhetoric}. Ce sont ces actions environnementales qui sont l'objet de notre second volet. A travers l'étude qualitative de sept cas d'entreprises, nous cherchons à identifier les leviers d'action environnementale dans l'ESS, ainsi que les facteurs qui motivent ou font obstacle à cette action. Nous nous appuyons pour cela sur des entretiens semi-directifs complétés par des documents (cf. tableau \ref{table:casetudies}). \\

Les cas étudiés ont été sélectionnés parmi des organisations installées dans la région Provence-Alpes-Côte d'Azur, et reconnus par l'INSEE comme appartenant au champ de l'ESS tel que défini par la loi de 2014\footnote{Nous nous sommes appuyé pour cela sur la base de données SIREN mise à disposition par l'INSEE, et qui comporte un champ indiquant cette appartenance}. Les entreprises susceptibles d'avoir une action environnementale (même réduite) ont été privilégiées. En effet, l'étude d'un cas d'organisation sans aucune action environnementale, quoiqu'elle puisse nous renseigner sur les barrières rencontrées, apporte peu d'éléments sur les déterminants. Cependant, l'étude ne se limite pas à des organisations directement concernées par l'action environnementale : nous cherchons à déterminer si l'appartenance à l'économie sociale favorise ou non l'action environnementale, y compris pour des organisations ayant une mission distincte. Ainsi, deux organisations du panel ont une mission environnementale (la Tour du Valat agit pour la protection des zones humides, et Air PACA opère dans le secteur de la qualité de l'air). Trois entreprises ont un lien indirect avec l'environnement : Provence Forêt, coopérative forestière ; AMS Environnement, qui a une activité d'entretien des espaces verts et Bzzz qui produit du miel et produit dérivés du miel. Finalement, Arborescence (groupe d'insertion par le travail) et Enfants et Loisirs (crèche), n'ont pas une activité environnementale a priori. L'échantillon cherche également à refléter la variété des structures organisationnelles de l'ESS. Il se compose ainsi d'une fondation, de quatre associations (dont une structure d'insertion) et de deux coopératives (dont une SCIC). Enfin, nous avons également privilégié des organisations nous donnant accès à des documents internes et dont les dirigeantes ou dirigeants ont pu se rendre disponibles pour réaliser des entretiens. 


\begin{longtable}{
        |>{\setlength{\baselineskip}{0.75\baselineskip}}K{0.18\linewidth}
        |>{\setlength{\baselineskip}{0.75\baselineskip}}K{0.16\linewidth}
        |>{\setlength{\baselineskip}{0.75\baselineskip}}K{0.18\linewidth}
        |>{\setlength{\baselineskip}{0.75\baselineskip}}K{0.48\linewidth}
    |}

        \caption{Cas étudiés}
        \label{table:casetudies} \\
        \hline
        \textbf{Organisation} & \textbf{Type} & \textbf{Entretien} & \textbf{Documents} \\ \hline
        \endfirsthead

        \hline
        \textbf{Organisation} & \textbf{Type} & \textbf{Entretien} & \textbf{Documents} \\ \hline
        \endhead

        Tour du Valat	& Fondation	& Directeur Général
        & - Plan stratégique \newline - Brochure \newline - Rapports annuels
        \\ \hline

        AMS Environnement &	Association	&Directeur
        & - Vidéo sur l'insertion
        \\ \hline

        Enfants et Loisirs &	Association	& Directrice
        &	- Projet social \newline - Projet éducatif et pédagogique \newline - Statuts \newline - Site web - écolocrèche
        \\ \hline

        Arborescence &	Coopérative	& Président
        & - Plaquette
        \\ \hline

        Provence Forêt	& Coopérative 	& Responsable environnement
        & - Plaquette \newline - Statuts \newline - Politique de gestion durable \newline - Engagement des propriétaires forestiers \newline - Newsletters (2015 et 2016)
        \\ \hline

        Air PACA	& Association	& Directeur
        & - Bilan 2016
        \\ \hline

        Bzzz	& Association	& Co-président
        & - CR conseil collégial (4) \newline - newsletters (3) \newline - PV AG 2017 \newline - Rapports d'activité 2016 et 2017 \newline - Rapport financier 2017 \newline - Statuts \newline -Programme prévisionnel 2018 \newline - Site web (Actions militantes - la Bzzz Mobile - Méthodes apicoles - valeurs et partenariats) \newline - 2 vidéos de présentation \newline - Formulaire de candidature au financement "mon projet pour la planète"
        \\ \hline

\end{longtable}

    \textcite[][p. 29]{yin2009case} décrit l'étude de cas comme \citit{une voie méthodologique rigoureuse} s'appuyant sur des \citit{procédures formelles et explicites}. L'auteur écarte ainsi d'emblée la vision d'une méthode purement interprétative, qui ne pourrait être utilisée qu'à titre d'exemple. Au contraire, Yin affirme que l'étude de cas peut être utilisée aussi bien à des fins d'exploration que de validation. Dans les deux cas, chaque étape doit être rigoureusement menée et documentée afin d'éviter les biais de validation. Yin décrit donc les modalités de l'étude de cas réaliste, c'est à dire qui admet l'existence \citit{d'une réalité unique indépendante de l'observateur} (p. 42).
    L'étude de cas est particulièrement pertinente lorsque l'on souhaite étudier un phénomène de manière détaillée et dans le cadre du contexte réel dans lequel il se produit. Elle permet aussi d'approfondir la compréhension lorsque \citit{les frontières entre le phénomène et le contexte ne sont pas clairement évidentes} \parencite[][p. 41]{yin2009case}. \\

    Pour garantir la validité de l'étude, nous nous appuyons sur un guide d'entretien élaboré à partir de la littérature. Il permet de maintenir une consistance entre les différents entretiens et de permettre une comparaison fiable. Le contenu des entretiens est retranscrit et codé à l'aide du logiciel Nvivo. Dans cette section, nous précisons la démarche d'élaboration du guide d'entretien et le processus de codage, et expliquons comment le concept d'identité organisationnelle est adapté à l'analyse.


\subsection{Élaboration du guide d’entretien}

    Le recours à l'entretien semi-directif nécessite le recours à un guide d'entretien (annexe \ref{annexe:guide}). Celui-ci s'appuie sur des questions ouvertes dont la vocation n'est pas d'être traitées successivement, mais d'établir la liste des thématiques à aborder avec les répondants. \\

    Le guide d'entretien est constitué à partir de la littérature et est guidé par la problématique.

    Une première partie de l'entretien porte sur des questions générales de compréhension des mécanismes de l'entreprise. Elles s'appuient sur les dimensions organisationnelles identifiées dans la littérature, à savoir la dimension utilitariste, la dimension normative, la dimension collective et la dimension fonctionnelle. Concernant la dimension utilitariste, il est demandé au répondant de présenter le modèle économique de l'organisation pour comprendre la place de la dimension commerciale dans son fonctionnement. Les personnes interrogées s'expriment également sur les valeurs et la finalité de l'organisation, afin d'évaluer les dimensions normative et fonctionnelle. Des questions sur les partenariats et la participation des parties prenantes permettent de mieux cerner la dimension collective. Certaines organisation de l'\ess ne revendiquent pas leur appartenance à ce segment de l'économie, voire l'ignorent. C'est pourquoi les questions sont guidées par la littérature sur l'ESS sans l'évoquer explicitement. \\

    La seconde partie s'intéresse spécifiquement à l'action environnementale de l'organisation, dans une perspective d'innovation. Il est demandé au répondant quelle est la place de l'action environnementale, si elle existe, et ce qui l'anime. Le guide s'appuie sur la littérature pour évaluer les catégories de déterminants de l'éco-innovation identifiables, ainsi que les types d'actions menées. Le terme \cit{d'innovation} est parfois mal compris, voire mal accepté dans certaines organisations : il suggère ou bien la nécessité de parler de quelque chose de complètement nouveau, jamais vu ailleurs, ou peut être perçu comme appartenant au registre de l'entreprise capitaliste. Les questions sont donc posées de manière plus neutre, en évoquant simplement l'action menée à des fins environnementales. Le caractère innovant ou non est évoqué plus tardivement dans l'entretien afin de limiter ce biais d'interprétation. \\

    Enfin, la troisième et dernière section interroge directement le lien entre l'appartenance à l'\ess (cette fois ci explicitement évoquée) et l'action environnementale.


\subsection{Pertinence de l’étude documentaire}
    Les entretiens sont complétés par l'étude de documents de l'organisation : rapports annuels, brochures, comptes rendus de réunion ou pages des sites internet. Ceci poursuit plusieurs objectifs. L'étude documentaire permet d'ajouter des éléments et des exemples que le répondant n'a pas eu le temps d'évoquer au cours de l'entretien. Les rapports annuels, ainsi que les pages d'actualité des sites internet sont plus exhaustifs et présentent fréquemment en détail les innovations réalisées. En revanche, les motivations à l'action environnementale sont peu développées, et souvent de manière indirecte. Les aspects moraux ou éthiques peuvent aussi être sur-valorisés par rapport aux motivations financières. Les barrières à l'éco-innovation sont généralement éludées : les documents indiquent rarement les projets qui n'ont pas pu être réalisés. En outre, les documents officiels ont une vocation promotionnelle et donc très orientée. Cependant, cette limite est peu dommageable dans la mesure où l'on s'intéresse à des éléments positifs, valorisants, que l'entreprise n'a pas d'intérêt à dissimuler.


\subsection{Utilisation du codage dans l'analyse}
    Les entretiens sont retranscris, puis codés à l'aide du logiciel Nvivo. Comme le rappellent \textcite[][p.90]{miles2014qualitative}, le codage permet de \citit{réduire un large volume de données en un petit nombre d'unités analytique}.
    La grille de codage s'appuie sur les mêmes éléments théoriques qui structurent le guide d'entretien. Elle retient les quatre dimensions d'analyse de l'ESS, à savoir la dimension utilitariste, la dimension normative, la dimension collective et la dimension fonctionnelle. Les innovations identifiées sont catégorisées selon qu'elles sont de type technologique, social, organisationnel ou institutionnel \parencite{rennings2000redefining}. Les déterminants de l'action environnementale sont abordés de manière plus ouverte, laissant la place à un codage émergent. En effet, \textcite{dart2010green} estiment que les \oess ne réagissent pas aux mêmes leviers que les entreprises du secteur lucratif, mais pourraient répondre à d'autres incitations. La littérature étant principalement alimentée par l'étude d'entreprises classiques, le choix est fait de s'inscrire ici dans une perspective exploratoire. \\

    Le codage repose sur une unité d'analyse assez ouverte, pouvant être la phrase ou un ensemble de phrases. S'il peut paraître arbitraire voire peu rigoureux, ce choix est en réalité pragmatique. En effet, comme le rappellent \textcite[][p. 35]{ayache2011codage}, \citit{personne  ne  sait  exactement  pourquoi  et comment  un  mot  ou  une  phrase  peuvent  parfois  constituer  une  unité  de  sens,  et parfois n’être pas considérés en eux-mêmes  comme  des  unités  de  sens  et  être  alors noyés dans une unité de sens plus vaste}. Nous cherchons donc à utiliser un découpage flexible, qui permette de s'intéresser réellement au sens du contenu, plutôt que de donner l'illusion d'une \citit{subjectivité  éclairée  du  chercheur}. En outre, le codage est réalisé dans une perspective de classement, et non de réelle quantification. Le codage est multinomal \parencite{ayache2011codage}, afin de permettre des recoupements entre différentes idées exprimées dans un même passage. Nous rejoignons les auteurs en considérant le codage comme un instrument, et non une mesure de la validité de l'étude. Ceci est cohérent avec l'objectif de la thèse et la posture réaliste adoptée. En effet, cette dernière repose sur la formulation d'énoncés qui peuvent être confirmés ou réfutés. Or, l'étude se positionne sur la phase de formulation plutôt que de confirmation. Ainsi, il n'est pas nécessaire (ni d'ailleurs possible, au vu du design de notre étude de cas), d'aboutir à une validation statistique des résultats basée sur le codage. La prépondérance d'un thème par rapport à un autre nous renseigne sur l'identité organisationnelle du cas étudié, ou encore sur son comportement vis à vis de l'environnement. Elle permet d'effectuer une comparaison entre des entreprises selon les thèmes les plus appuyés, et d'identifier des concordances entre des thèmes qui apparaissent souvent ensemble. Mais le codage réalisé ici ne vise pas à donner une vision statistique de l'action environnementale dans l'\ess.


\begin{comment}
    \subsection{Opérationnalisation du concept d'identité organisationnelle}

    L'étude s'appuie sur le concept d'identité organisationnelle, présenté dans la littérature. Pour chaque cas, nous cherchons à identifier les identités les plus fortement marquées. Pour cela, nous nous appuyons sur les éléments des entretiens ou des documents qui renvoient à ces identités. Dans ce paragraphe, nous précisons les critères d'évaluation des identités.

    \textbf{Identité utilitariste}
    \begin{itemize}
        \item Les sources de revenus : les organisations ayant une forte identité utilitariste valorisent l'indépendance financière en cherchant à maximiser les ressources liées à leur activité. Elles cherchent donc à générer un chiffre d'affaire qui permette, autant que possible, de couvrir leurs coûts de fonctionnement. Inversement, les organisations ayant une faible identité utilitariste considèrent comme légitime le recours à des dons et des subventions. Les revenus de l'activité ne doivent pas nécessairement couvrir les coûts et les ressources externes sont perçues comme un financement du service d'intérêt général rendu.
        \item La détermination des prix : Les entreprises ayant une forte identité utilitariste adoptent des prix de marché
        Prix de marché VS tarification particulière ou prix \cit{juste},
        \item Professionnalisation des activités VS différentes formes d'auto-gestion,
        \item Ressources humaines salariés VS bénévoles,
        \item Non lucrativité VS lucrativité limitée. \\
    \end{itemize}

    \textbf{Identité normative}
\end{comment}
