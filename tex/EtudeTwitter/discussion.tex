\section{Discussion}

    \subsection{Stratégies de communication}

        \subsubsection{Utilisation du réseau}
            %\todo[inline]{Rajouter la partie correspondante dans les résultats ou bien supprimer ce paragraphe}
           % Les résultats montrent une grande variété d’utilisation du réseau, ce qui confirme les observations de \textcite{guo2017speaking}. Twitter est particulièrement investi par Des associations et fondations qui s’adressent à un public très large qu’elles peuvent facilement atteindre sur Twitter. Ce sont les plus prolifiques en termes de nombre de tweets. Cependant, Twitter est principalement utilisé comme un média de diffusion d’informations plutôt que comme un moyen d’échange et d’interaction. Conformément à ce qui a été observé par \sources, la part des tweets consacrée à répondre à d’autres utilisateurs est faible. En revanche, les mentions, qui permettent d’interpeler, de référencer ou de citer un autre utilisateur, sont plus utilisées. Les organisations s’inscrivent donc bien dans une logique de réseau et d’interconnexion, mais la place pour l’échange et la discussion reste limitée. Ce constat est légèrement moins marqué pour les entreprises sociales qui consacrent une part un peu plus importante des tweets à répondre à d’autres utilisateurs (raison ?). Ces résultats rejoignent ceux de \textcite{guo2014tweeting}, qui observent trois modalités d’usage de Twitter. Il s’agit de diffuser de l’information, de créer une communauté et d’appeler l’audience à agir. Les résultats montrent que 68\% des tweets sont consacrés à diffuser une information alors que seulement 20\% visent à créer une communauté, et 12\% appellent à l’action. Une observation comparable est faite par \textcite{mariaux2017promouvoir} qui distinguent cette fois cinq catégories de tweets visant à promouvoir la transition écologique. Ils reposent sur une logique d’information, de mise en évidence d’alternatives, de dialogue, d’action ou de critique. A nouveau le registre prédominant est celui de l’information (35\%), qui rassemble des tweets à sens unique, qui n’attendent pas de réponse ou de réaction de l’audience. La catégorie de tweets visant à promouvoir des alternatives innovantes (20\%) regroupe aussi un contenu à sens unique. Les tweets consacrés à encourager un échange ente l’organisation et son audience représentent 20\% du volume. Cependant, une part importante de ce contenu invite à une rencontre en dehors du réseau, à l’occasion de forum, de conférences ou de portes ouvertes. Ils n’ouvrent pas réellement à une discussion au sein du réseau. \\

            Un enjeu important sur Twitter est la visibilité des contenus diffusés. Le réseau étant utilisé comme un outil d’information, son efficacité repose sur la capacité à atteindre une audience large. Dans le contexte du secteur non lucratif aux USA, \textcite{guo2017speaking} ont montré que le nombre d’abonnés et le nombre de tweets diffusés constituaient des facteurs fondamentaux pour atteindre l’audience la plus large possible. Les résultats de la présente étude, menée dans un contexte plus large, celui de l’ESS, confirment ces observations. Au-delà de la stratégie rhétorique, au-delà du type de contenus diffusés, ce sont le nombre d’abonnés et le volume des tweets qui permettent d’atteindre un public large. Or la popularité sur ce réseau est fortement liée à la popularité médiatique ou sociétale d’une personne ou d’une organisation. Une organisation qui bénéficie déjà d’une certaine renommée peut donc atteindre plus facilement une large audience sur Twitter. Ceci est confirmé par l’observation de notre échantillon pour lequel les organisations ayant le plus d’abonnés sont des associations et fondations connues du grand public. Au regard de la transition écologique, ces organisations jouent donc un rôle fondamental dans la diffusion des alternatives et dans la promotion de comportements et de modes de vie ou de production plus respectueux de l’environnement. Naturellement, toutes ces organisations n’ont pas vocation à promouvoir la cause environnementale. Des associations comme les Restos du Cœur, la Croix Rouge ou Oxfam France, qui ont toutes les trois plus de 100 000 abonnés sur Twitter, ont des missions orientées vers des problématiques sociales éloignées des préoccupations environnementales. Pourtant, elles ont, en raison de leur popularité, la capacité à porter et à promouvoir les questions d’écologie de façon plus efficace que des organisations moins connues dédiées à l’environnement. Dans une recherche publiée en 2014, \textcite{buchs2014role} montre que l’investissement dans une organisation environnementale conduit à modifier les comportement individuels vers des pratiques plus responsables. Cependant, elle souligne la portée limitée de ces seules organisations, et souhaite encourager les organisations du secteur non lucratif à porter la thématique de l’écologie, même si elle semble éloignée de leur mission première. La présente recherche conduit à soutenir ce propos : sur un réseau social comme Twitter, qui permet d’atteindre un public extrêmement large et divers, les entreprises de l’ESS bénéficiant d’une audience importante sont en capacité de promouvoir efficacement la transition écologique.   \\

            L’étude met également en évidence la façon dont les utilisateurs sont connectés sur le réseau social. Elle fait apparaître une forte interconnexion entre les organisations de l’ESS : bien que des communautés se distinguent, celles-ci sont le plus souvent très liées entre elles. Deux communautés font exception. Les coopératives agricoles forment un ensemble d’organisations très connectées entre elles, mais ayant peu de lien avec le reste de l’ESS. Elles s’identifient donc davantage par leur activité que par leur appartenance à l’ESS. Ceci peut s’expliquer par le fait que le choix du mode coopératif repose essentiellement sur les spécificités de leur activité : la production agricole et les conditions difficiles des agriculteurs en France imposent la mise en commun de certains moyens de production et de certains processus, afin de mieux résister à la concurrence internationale. Ces entreprises ne semblent pas identifier de proximité particulière avec l’ESS dans son ensemble, mais seulement avec les organisations qui partagent la même activité.
            De manière similaire, les mutuelles forment une communauté à part du reste de l’ESS. Les mutuelles ont un positionnement spécifique au sein de l’ESS, car elles sont exclusivement dédiées aux activités de l’assurance et de la santé. Elles montrent toutefois des liens plus forts avec le reste de l’ESS, notamment avec les associations professionnelles comme le CNCRES mais aussi avec des organisations du secteur caritatif donc les activités sont souvent proches et avec lesquelles des partenariats sont possibles.
            Une communauté regroupe les organisations dont la mission est liée avec l’environnement. Elle est toutefois très liée avec le reste de l’ESS, en particulier le mouvement caritatif, le mouvement des entreprises sociales et le mouvement coopératif. Ceci traduit le fait que la cause environnementale est prise en charge par différents types d’organisations et à travers différents statuts. Des associations et fondations vont se trouver plutôt dans une logique de protection, de réparation ou de sensibilisation (Green Peace, Fondation Nicolas Hulot…) quand des coopératives ou entreprises sociales vont chercher à développer des modes de production alternatifs et plus respectueux (Enercoop, Biocoop).\\

            Les six communautés identifiées (Entreprises agricoles, Mutuelles, Secteur caritatif, Action environnementale, Mouvement coopératif, Entrepreneuriat social) ne constituent pas une classification homogène qui permettrait d’aboutit à une typologie d’entreprises de l’ESS. Certains groupes sont regroupés autour d’un statut, d’autres selon une activité. Les interactions montrent que les organisations de l’ESS se rassemblent en fonction du statut, de l’activité mais aussi autour des organes fédérateurs de référence. L’ESS n’apparaît donc pas ici comme une économie segmentée, qui distinguerait nettement différentes catégories d’entreprises, mais plutôt comme une multiplicité de modèles économiques et d’activités interagissant entre elles.

        \subsubsection{Prise en compte de l’environnement}

            Il est intéressant de constater que les entreprises de l’ESS ayant adopté une logique de marché sont celles qui semblent avoir le mieux intégré le discours environnemental (coopératives, entreprises sociales). Ces entreprises, bien qu’elles intègrent les principes de l’ESS, s’appuient plus volontiers sur des pratiques commerciales et managériales du secteur lucratif. Ceci peut s’expliquer par des attentes plus fortes vis-à-vis de ces entreprises que pour le reste de l’ESS.  Comme le soulignent \textcite{dart2010green}, le secteur non lucratif bénéficie d’une image éthique et responsable qui minimise les attentes en matière de responsabilité sociale et environnementale. Déjà légitimées par leur action sociale, les organisations de ce secteur n’auraient plus besoin de justifier de leur responsabilité. Il est probable que les entreprises sociales et coopératives, en raison de la dimension commerciale de leur activité, ne bénéficient pas (ou pas autant) de cette légitimité a priori, et soient donc obligées de démontrer qu’elles s’inscrivent également dans une démarche éthique et responsable. En outre, en lien avec le développement de l’entrepreneuriat social, de nouveaux modes de financement dédiés à ce secteur apparaissent. Reconnaissant les principes de l’ESS, les investisseurs attendent un retour sur investissement beaucoup plus faible que pour des entreprises capitalistes. Cependant, les exigences en termes de performance sociale et environnementale sont élevées et les entreprises doivent démontrer leur contribution à la société, à travers des éléments concrets et des outils de mesure. Il est donc vraisemblable que les entreprises sociales passées par les circuits entrepreneuriaux dédiés à l’ESS (incubateurs, concours pour obtenir des aides, demandes de financements à des « Business Angels ») sont sensibles à toutes les facettes qui contribuent à créer de la valeur sociale, y compris au volet environnemental. Pour autant, il convient de s’interroger ici sur le lien entre la communication environnementale et la mise en œuvre d’actions concrètes. L’étude s’arrête à constater une différence dans le discours sur Twitter, c’est-à-dire dans la communication. Or, pour les raisons citées précédemment, les organisations de type entreprises sociales sont mieux au fait de la nécessité de valoriser leur impact et leur responsabilité environnementale. Les dernières années ont vu émerger des « startups sociales » portées par des entrepreneurs jeunes, issus de grandes écoles, très au fait de l’importance du marketing et de la communication et maitrisant très bien le fonctionnement des réseaux sociaux. On peut donc craindre que les résultats ne soient influencés par un « effet de communication » là où d’autres structures de l’ESS se concentreront sur leur activité et valorisent peut-être moins l’aspect environnemental dans leur discours. Ainsi, les organisations qui cherchent précisément à se différencier des entreprises de marché auront plus de peine à s’inscrire dans une forme de « culture de la communication ».  \\

            L’étude permet d’identifier plusieurs stratégies rhétoriques dans la promotion des thématiques environnementales. Deux stratégies s’opposent très nettement entre les organisations du secteur non lucratif et les organisations adoptant un positionnement proche des entreprises de marché. Ces dernières, à savoir les coopératives et entreprises sociales, communiquent à travers les cadres de l’innovation, de la mise en évidence d’alternatives et du développement économique. Pour elles, la transition écologique et le développement durable constituent une opportunité de développement, conformément à ce que propose \textcite{cretieneau2010economie}. Pour les associations et fondations, le discours repose sur la controverse, le débat d’idées et une prise de position militante. Elles peuvent avoir recours à un certain catastrophisme visant à alerter sur l’urgence du changement. Elles ont donc davantage un rôle de sensibilisation et d’information. Les mutuelles apparaissent un peu en marge de cette opposition. Pour elle, le discours environnemental est employé en lien avec leurs thématiques clés, à savoir le progrès social. L’environnement peut par exemple être abordé à travers son impact sur la santé et sur les bonnes pratiques à adopter. \\

            La stratégie adoptée par le secteur non lucratif, c’est-à-dire un discours engagé, militant, semble plus performant. Ces tweets souvent percutants sont largement partagés sur le réseau social. Pour autant, il n’est pas nécessairement pertinent pour toutes les organisations de l’adopter. Il correspond en effet à l’identité de certaines associations et fondations, dont la mission est d’interpeller, de critiquer et de militer. Cependant, pour des organisations s’inscrivant dans une logique commerciale, visant à se doter d’une identité « professionnelle » \parencite{dart2004being}, cette stratégie pourrait être contreproductive et conduire à délégitimer l’entreprise au regard de parties prenantes importantes pour elles (notamment les financeurs publics ou privés). C’est pourquoi elle est plutôt adoptée par des organisations comme Green Peace, qui revendique une totale indépendance et ne se finance qu’à travers des dons de particuliers. Les entreprises proches d’un fonctionnement de marché adoptent plutôt une stratégie politiquement plus neutre, s’appuyant sur leur dimension novatrice, économiquement performante, pour porter le discours environnemental. Elles courent ainsi le risque de passer à côté d’un « effet buzz », mais maintiennent une image innovante et constructive. Il faut toutefois souligner que si le cadre rhétorique des conflits et du débat obtient les meilleures performances, il n’est pas majoritaire, y compris au sein du secteur non lucratif. S’il est très utilisé par certaines organisations symboliques, \textcite{mariaux2017promouvoir} montent qu’une majorité de tweets adoptent plutôt un caractère factuel, s’appuyant sur des faits documentés plutôt que sur des opinions ou des prises de position. \\

            Enfin, l’impact des stratégies rhétoriques doit être relativisé. Le nombre d’abonnés d’une entreprise, c’est-à-dire sa popularité sur le réseau, semble plus important que le type de discours employé pour atteindre une large audience. Autrement dit, la portée du tweet dépend moins du contenu que de l’émetteur. Ceci étant, un statut provocateur, engagé, peut bénéficier de l’effet « buzz » et permettre in fine une augmentation du nombre d’abonnés. Ceci conduit à s’interroger sur le lien entre le contenu et le nombre d’abonnés : la popularité résulte-t-elle plutôt de la renommée de l’organisation en dehors du réseau, ou bien du type de contenus diffusés qui ont permis de susciter l’intérêt d’un grand nombre d’utilisateurs ? Les deux effets sont probablement combinés, dans la mesure où \textcite{guo2017speaking} ont également démontré l’importance du nombre de tweets sur la diffusion des contenus, ce que confirment nos résultats. On peut également s’interroger sur l’audience que l’on souhaite atteindre. Certains publics favorisent probablement des contenus engagés et sont plus intéressés par une prise de position vis-à-vis de l’actualité, quand d’autres recherchent sur Twitter une information plus neutre et moins subjective.


    \subsection{Contribution technique et méthodologique}

        Cette étude adopte des techniques récentes, encore peu utilisées dans les sciences humaines. Par son caractère original, elle vise à proposer ou développer des outils permettant à la science de gestion de profiter de l'essor des données disponibles sur internet.  Ce travail s'appuie sur plusieurs méthodes qui présentent chacune des intérêts pour la recherche, mais soulèvent aussi certains biais. \\

        L'utilisation d'un langage de programmation pour la recherche permet tout d'abord d'accéder à des volumes importants de données. En effet, la collecte de tweets, même en nombre plus réduit que notre échantillon, est peu aisée et extrêmement chronophage. L'automatisation garantit en outre la qualité et l'uniformité des données. En rendant systématique la collecte, il est aisé d'étudier des données de même nature, mais collectées sur des périmètres ou des périodes de temps différentes. Les scripts ou applications développés pour collecter les données sont en outre aisément transposables à une toute autre étude.
        % \todo[inline]{Si ça parait avant la fin de la thèse, on pourra citer la réutilisation des scripts pour l'étude de  Fabienne sur l'AI dans la radiologie}
        Si le volume des données est bien sûr un des enjeux, toutes les problématiques ne justifient pas d'analyser un million de tweets. Ainsi, comme le souligne Gregory Saxton dans une note de blog\footnote{\url{http://social-metrics.org/python-for-academic-research/}}, un autre intérêt de ces méthodes est l'originalité des données auxquelles elles donnent accès. Twitter est un exemple d'entreprise mettant à disposition des données structurées, mais de nombreuses organisations s'inscrivent dans une démarche d'\cit{Open Data} ( \cit{Données ouvertes}). En France, des données économiques produites par des organismes publics (par exemple l'INSEE) sont disponibles librement sur \url{https://www.data.gouv.fr/fr/}. Cette base de données très vaste comporte également des informations sur l'implantation des services publics ou leurs réalisations, sur l'ensemble du pays comme sur des territoires en particulier. De tels éléments pourraient par exemple apporter de la valeur aux recherches en management public. De manière plus restreinte, le réseau social professionnel Linkedin, dont les données sont protégées, a décidé d'ouvrir l'accès à une API dédiée aux universitaires, think-tanks et ONG\footnote{\url{https://engineering.linkedin.com/teams/data/projects/economic-graph-research/economic-graph-details}}. Ces données ont un potentiel considérable pour des recherches autour de la gestion des carrières, des processus de recrutement ou encore des réseaux professionnels.

        Les outils informatiques permettent également d'automatiser la collecte de données moins structurées. Les recherches en netnographie s'appuient sur les messages échangés sur des forums ou autres espaces de discussion en ligne. La collecte est souvent manuelle et demande un investissement temporel considérable. La technique du \cit{scraping}, qui consiste à collecter automatiquement le contenu des pages d'un site internet, lorsqu'il n'existe pas d'interface de type API, peut constituer un gain de temps et d'efficacité pour les chercheurs qui utilisent de telles méthodes. \\

        L'étude menée ici présente successivement plusieurs approches, plusieurs utilisations possibles des données. Celles-ci permettent d'apprécier le discours environnemental dans l'\ess à travers plusieurs points de vue. Elles sont également à mettre en lien avec la posture épistémologique du chercheur et la distance ou proximité qu'elle suppose d'avoir avec les données, ainsi qu'avec la finesse des résultats attendus en fonction de la problématique. \\

        Les premières approches s'appuient sur des données quantitatives, objectives. Le nombre d'abonnés, de retweets, de favoris, ou encore l'existence de liens à sens unique ou réciproques entre les utilisateurs sont factuels et ne sont pas le résultat de l'action ou de l'interprétation du chercheur. On peut dire avec confiance que la recherche n'a pas d'effet sur les données, la collecte étant transparente pour les utilisateurs du réseau social. La seule exception serait le cas d'un chercheur lui même très actif sur Twitter et ayant une capacité d'influence. Ces techniques sont donc particulièrement pertinentes dans une perspective positiviste, dans laquelle le chercheur doit éviter autant que possible d'agir sur les données. Il demeure bien sûr certains biais. Ils se présentent d'une part au niveau de la collecte, dans la définition du périmètre, le choix des utilisateurs à intégrer au panel et des mots clés à retenir, et d'autre part dans le choix des analyses réalisées. \\

        La classification non supervisée constitue également une approche dans laquelle le rôle du chercheur est limité. Toutefois, elle implique une part d'analyse dans le choix de l'algorithme et dans le choix du nombre de catégories à retenir ($n$). Le chercheur peut donc agir sur deux paramètres pour faire varier les résultats, jusqu'à l'obtention d'une classification ayant du sens. Une manière de valider que la classification retenue est la plus pertinente pourrait être de partager dans les résultats non seulement les catégories retenues, mais également les catégorisations proches (avec le même algorithme mais $n-1$ ou $n+1$ groupes, ou avec $n$ groupes mais un algorithme différent). Le lecteur peut ainsi être en mesure de juger de la pertinence du choix effectué. On peut également considérer qu'en limitant le rôle du chercheur dans la réalisation des catégories, on laisse cette responsabilité à un algorithme. Ceci soulève deux inquiétudes. D'une part, le processus peut se révéler opaque si le chercheur n'a pas les compétences algorithmiques pour comprendre ce que fait le programme. D'autre part, l'algorithme est bien sûr le résultat d'un travail produit par d'autres chercheurs, ingénieurs ou mathématiciens. On n'est donc pas tout à fait dans la situation idéale (dans une perspective positiviste) où le chercheur est complètement distancié de l'objet de recherche et l'outil de mesure parfaitement neutre. \\
        %    \cofeAm{0.2}{1}{180}{3cm}{-5cm}

        Les deux approches suivantes (classification supervisée et codage) impliquent une intervention directe du chercheur sur les données. L'intérêt de ces approches par rapport à la précédente est évident : la recherche est guidée par la problématique et aboutit à des résultats potentiellement plus précis. Le contenu n'est pas classé de façon émergente, mais au contraire selon la typologie attendue par le chercheur. Il y a toutefois une distinction importante entre les techniques mobilisées pour l'étude du sentiment et de l'objectivité et pour celle suivant la classification du cadrage rhétorique. Dans le premier cas, on classe un échantillon de tweets, puis on laisse à l'algorithme le soin \cit{d'apprendre} et déterminer lui-même les structures qui ont conduit à cette classification, pour ensuite la reproduire sur le reste du corpus. Le principal biais provient de la construction de ce corpus d'apprentissage : le classement du contenu par le chercheur peut être discutable. En outre, la force de l'algorithme dépend également du volume de données pré-codées. Ainsi, un niveau de confiance supérieur peut être atteint à condition d'effectuer un codage manuel sur un plus grand nombre de tweets. Dans le second cas, on donne au logiciel une règle pré-déterminée d'affectation. Il s'agit d'un processus automatisé, mais il n'y a plus d'intelligence machine en action. Le biais est donc d'autant plus important. \\
