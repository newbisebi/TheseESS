De nombreuses \oess\ ont adopté Twitter comme moyen d’expression et de promotions d’idées. La thématique de l’environnement, du changement climatique et plus généralement de la nécessaire transition écologique font partie des sujets discutés. \\

L’utilisation de cette source de données très vaste, à l’aide d’un langage informatique et combinée avec la méthode de l’analyse lexicale constitue une approche originale, encore peu utilisée dans la recherche en sciences de gestion. Twitter donne accès à des données à la fois quantitatives et qualitatives, qui peuvent être appréhendées de façon statique (approche retenue ici) ou bien longitudinale. Sur ce réseau, les contenus peuvent être évalués pour leur performance, qui donne une idée de l’ampleur de la diffusion d’un message et de l’audience atteinte. Les interactions publiques  entre les utilisateurs, qui prennent la forme de mentions, de réponses à des messages ou d’abonnements, sont documentés et peuvent donner lieu à des études à l’aide de l’approche par les réseaux. Les analyses portant sur des volumes considérables de données peuvent être complétées par des études qualitatives (par exemple \textcite{guo2014tweeting}, 2014 et \textcite{mariaux2017promouvoir}). Bien que l’utilisation de Twitter soulève un certain nombre de difficultés qui lui sont spécifiques, nous pensons qu’il peut constituer un matériau utilisable dans de nombreuses disciplines des sciences de gestion. \\

Les résultats mettent en évidence une \ess\ caractérisée par des pratiques diverses, mais avec pourtant une forte interconnexion. Des acteurs de référence (organisations renommées ou organismes représentatifs) jouent un rôle de rassemblement et contribuent à constituer un véritable réseau d’\oess.  L’étude met en évidence une opposition entre le secteur non lucratif, historiquement plus éloigné du secteur capitaliste, et des entreprises de l’\ess\ qui au contraire adoptent des pratiques de marché. Cette opposition se traduit par une différence d’approche des thématiques environnementales. Les premières adoptent un rôle de sensibilisation ou de contestation, alors que les suivantes ont une démarche de développement d’alternatives crédibles sur le plan économiques. En outre, les organisations adoptant des pratiques « business » semblent plus sensibilisées au sujet de l’écologie, et plus consciente de leur responsabilité environnementale. \\

Ces conclusions doivent être prises avec certaines précautions. L’approche qui caractérise les associations et fondations comme des entreprises éloignées du marché, et les coopératives et entreprises sociales comme proche du marché est simplificatrice. Elle conduit à masquer la grande diversité des organisations, qui ne se limite absolument pas à leur statut juridique. La deuxième partie de l’étude s’appuie sur une mesure plus précise de l’orientation au marché et a pour objectif de tester de manière quantitative une série d’hypothèses formulées à partir du travail présenté ici. 
