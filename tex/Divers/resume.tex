% SET SPECIFIC FONT
\usefont{T1}{phv}{m}{n}\selectfont{}
\renewcommand*{\arraystretch}{1.2}


\begin{center}
  \large \textbf{Les organisations de l'Économie Sociale et Solidaire face aux enjeux écologiques : stratégies de communication et d'action environnementale}
\end{center}

\vspace{1cm}

La protection de l'environnement naturel constitue un enjeu déterminant pour le futur de l'humanité. L'Économie Sociale et Solidaire (ESS), qui partage les principes du développement durable, est particulièrement bien placée pour mettre en oeuvre des alternatives de développement plus écologiques. Ce secteur composé des associations, des coopératives, des mutuelles, des fondations et des entreprises sociales bénéficie d'un préjugé positif puisque ses organisations agissent pour l'intérêt général. C'est pourquoi peu d'attention est donnée à l'action environnementale de ces organisations. Cette recherche a pour objet d'examiner les facteurs et les modalités de l'action environnementale dans cette économie hétérogène. La thèse appréhende les organisations de l'ESS sous l'angle de l'identité organisationnelle et s'intéresse d'une part à la communication environnementale, d'autre part aux actions concrètes. Une étude est menée pour chacun de ces deux volets de la recherche.\\

La première étude a pour terrain le réseau social Twitter et vise à identifier les stratégies rhétoriques adoptées par les organisation de l'ESS sur les sujets liés à l'environnement. Plus de 910 000 tweets ont été collectés et analysés à l'aide d'un programme codé dans le langage information Python. En recourant à une démarche \citit{Big-Data} et à l'aide des techniques d'exploration automatique de texte, nous identifions les principales thématiques du corpus et mettons en évidence l'opposition entre des stratégies militantes et des stratégies basées sur les opportunités de développement économique. \\

Cette première étude est complétée par l'étude de sept cas, sur la base d'entretiens menés auprès de dirigeantes et dirigeants d'organisations de l'ESS. Elle s'intéresse aux facteurs de l'action environnementale, en lien avec l'identité organisationnelle de l'ESS. Les résultats mettent en lumière la grande diversité d'approches au sein de l'ESS, mais aussi le rôle déterminant de l'engagement individuel, des convictions et des logiques collectives dans la réalisation d'actions environnementale. L'étude aboutit à la formulation de huit propositions de recherche qui synthétisent les relations identifiées. \\

Ce travail apporte plusieurs contributions à la littérature. Sur le plan méthodologique, nous développons l'approche de l'exploration automatique de texte, rarement utilisée dans les Sciences de Gestion. Sur le plan théorique, la thèse introduit la dimension collective en tant qu'identité organisationnelle de l'ESS. Nous adaptons ensuite un modèle d'action environnementale en identifiant un déterminant supplémentaire spécifique à ces organisations. Finalement, la recherche invite l'ESS à remettre au centre les questions d'écologi, et donne des pistes pour soutenir les organisations dans une démarche environnementale. \\

\vspace{1cm}


\textit{\textbf{Mots clés :} Action environnementale, Économie Sociale et Solidaire, Rhétorique environnementale, Identité organisationnelle, Big-Data, Exploration de texte}

