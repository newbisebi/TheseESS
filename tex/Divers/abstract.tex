\begin{center}
  \large \textbf{Social and Solidarity Economy organisations in the face of ecological challenges: strategies for communication and environmental action}
\end{center}

\vspace{1cm}

The protection of the natural environment is a key stake for the future of humanity.  The Social and Solidarity Economy (SSE), which shares the principles of sustainable development, is particularly well placed to implement more ecological development options. This sector, which is composed of associations, cooperatives, mutual firms, foundations and social enterprises benefits from a positive bias since its organisations act in the general interest. As a result, little attention is paid to the environmental action of these organisations.
The purpose of this research is to examine the factors and modalities of environmental action in this heterogeneous economy. The thesis looks at SSE entities from the perspective of organizational identity, and focuses on environmental communication on the one hand, and concrete actions on the other. A study is conducted for each of these aspects.\\

The first study focuses on the Twitter social network and aims at identifying the rhetorical strategies adopted by SSE organisations on environment-related topics. More than 910,000 tweets were collected and analyzed with the help of a program coded in the Python information language. By applying a "Big-Data" approach as well as text mining techniques, we identify the main themes of the corpus, and highlight the opposition between militant strategies and strategies based on economic development opportunities. \\

This first study is completed by the examination of seven cases, based on interviews with leaders of SSE organisations. It focuses on the factors of environmental action, in relation to the organizational identity of the SSE. The results highlight the great diversity of approaches within the SSE, but also the decisive role of individual commitment, convictions and collective logic in the implementation of environmental actions. The study resulted in the formulation of eight research proposals that describe the identified relationships.  \\

This work makes several contributions to the literature. On the methodological level, we develop the text mining techniques, rarely used in Management Sciences. On the theoretical level, the thesis introduces the collective dimension as an organisational identity of the SSE. We then adapt an environmental action model by identifying an additional determinant specific to these organizations. Finally, the research invites the SSE to put ecological issues back at the centre, and provides suggestions for supporting organisations in an environmental approach.\\ 

\vspace{1cm}

\textit{\textbf{Mots clés :} Environmental action, Social and Solidary Economy, Environmental rhetoric, Organisational identity, Big-Data, Text-mining}

% RESET FONT
\usefont{T1}{bch}{m}{n}\selectfont{}
\renewcommand*{\arraystretch}{1.5}
