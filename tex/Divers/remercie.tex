% !TEX root = /Admin/main.tex


%%%%% INTRO
L'achèvement de cette thèse ne sera pas, comme elle l'aurait pu, le début d'une carrière académique. Mais cela n'a pas d'importance, car ces années de travail représentent pour moi bien plus qu'une étape de ma vie professionnelle. C'est pourquoi je voudrais exprimer ma reconnaissance à celles et ceux qui ont contribué, de bien des façons, à sa réussite. \\

%%%%% Directrice thèse
Je souhaite en premier lieu remercier ma directrice de thèse, Emmanuelle Reynaud, pour son accompagnement durant ces années. Peu de doctorants peuvent affirmer n'avoir jamais attendu plus de quelques jours une réponse de leur superviseur. J'ai eu cette chance, même quand je vous assaillais de long mails et documents à relire.  Merci pour vos conseils précieux et vos idées pour orienter mon travail. Mais j'aimerais surtout vous remercier d'avoir su m'encourager à persévérer, toujours avec une grande bienveillance, dans les moments où la motivation faisait cruellement défaut. \\

%%%% Jury + invité
J'adresse également ma reconnaissance aux personnalités qui me font l'honneur de participer à mon jury de thèse. Merci aux Professeurs Sandrine Berger-Douce et Samuel Mercier d'avoir accepté d'être les rapporteurs de cette thèse. Je remercie également le Professeur Gille Paché pour ses précieux conseils et ses encouragements dans la finalisation de ce travail. Enfin, convaincu de l'intérêt des échanges entre le milieu académique et le monde de l'entreprise, c'est très chaleureusement que je remercie Monsieur Roman Bayon d'avoir accepté de partager son point de vue d'expert de l'informatique sur mon travail. \\

%%%%%% Répondants
Je tiens à remercier les dirigeants et dirigeantes d'organisations de l'ESS que j'ai rencontrés au cours de ma recherche. Merci d'avoir accepté de me consacrer du temps pour répondre à mes questions et m'aider à mieux comprendre ce qu'est réellement l'économie sociale. Ces moments où j'ai pu découvrir une partie de votre métier et de vos missions ont été parmi les plus passionnants de ce travail. \\

%%%%% CERGAM
Au delà de l'accompagnement académique, j'ai pu compter sur bien des soutiens au cours de mes années  à l'IAE. Un immense merci en particulier à Marie, Léonie et Fabienne, sans qui les contraintes administratives auraient eu raison de ma persévérance. Mais surtout, merci pour votre gentillesse et vos encouragements. \\

%%%%%% Amis
Un grand merci à mes collègues mais néanmoins (et surtout !) ami·e·s. Merci pour votre humour qui m'a aidé à garder la distance psychologique indispensable à ce travail, et pour votre soutien social qui m'a permis d'échapper aux affres du stress. Vos intelligences pas du tout artificielles et vos opinions de leader·euse·s ont rendu cette expérience en votre compagnie véritablement enrichissante. C'est grâce à vos précieux conseils et à l'héritage cognitif de mes prédécesseurs que je suis parvenu à achever ce travail.
C'est donc toujours marqué d'une certaine nostalgie que je relierai vos noms alors que vieillira cette thèse : Mohamed, Bénédicte, Fabien, Yasmine, Nada, Fabienne, Joseph, Aurélie, Kunjika, et bien d'autres encore. Ne vous laissez pas duper par le ton humoristique et les quelques clins d'oeil de ce paragraphe : c'est sincèrement que je vous remercie. Faute de rester votre collègue, j'espère avoir la chance de vous garder comme amis. \\


%%%%%% Famille
Du fond du coeur, je remercie mes proches, mes parents, mes frères et soeurs, et toutes les personnes qui m'ont accompagné depuis des années. Au delà du harcèlement en règle que vous avez collectivement organisé pour m'interdire toute désertion de cette longue entreprise, je vous suis reconnaissant pour tant de choses. Merci de m'avoir accompagné tout au long de mes études, mais aussi de ma vie en général, et de m'avoir permis d'en arriver là. Je n'ai jamais vraiment bien su l'exprimer, mais je mesure chaque jour la chance que j'ai eu d'être aussi bien entouré et d'avoir reçu autant de soutien dans tous les moments difficiles. \\

Enfin, à Gabrielle, qui a été à mes côtés la plus longue partie de mon doctorat. C'est toi qui m'a encouragé à entreprendre ce projet et qui m'a donné tant de motivation pour le poursuivre. Merci infiniment pour ton soutien, pour m'avoir rappelé chaque jour le sens de ce travail et surtout merci d'avoir cru en moi. \\

% A tous, une fois encore je vous remercie, et je vous souhaite une bonne lecture.


% \newpage


% \vspace*{0.2\textheight}

% \begin{addmargin}[0.3\linewidth]{0em}
%     \begin{flushright}
%     À Lucas, Julia, Chiara et Adèle, \bigbreak

%     \hfill
%     \end{flushright}

% \end{addmargin}
