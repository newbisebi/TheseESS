L'humanité connaît depuis deux siècles un essor formidable, sans précédent dans son histoire. Le développement des connaissances scientifiques et la mise au point  de technologies prodigieuses ont conduit à un allongement de l'espérance de vie, à de meilleures conditions de santé et à une augmentation importante du niveau de vie. Ce bond extraordinaire est cependant assorti d'un coût social et écologique qui menace aujourd'hui notre espèce. Alors que les nations s'enrichissent, la pauvreté augmente à travers le monde et les écarts se creusent \parencite{lange2018changing}.  La consommation de ressources naturelles excède largement la quantité que la planète est capable de regénérer et les dégâts causés à l'environnement sont considérables. L'alerte, déjà donnée il y a plusieurs décennies, est toujours aussi pressante et les actions menées ne sont pas encore à la hauteur des enjeux. Pour relever ce défi, l'humanité n'a d'autres choix que de remettre en question ses modes de vie et de développement. Ceci requiert inévitablement des changements individuels, mais aussi une transformation en profondeur des modes de production, accompagnée par des décisions politiques. \\

A travers cette recherche, nous avons souhaité mettre la lumière sur la contribution de l'Économie Sociale et Solidaire à cette transition. Alors que le système économique majoritaire dépend de la croissance et repose sur la course aux profits, les organisations de l'\ess s'inscrivent dans un tout autre paradigme. Au lieu d'opérer les yeux bandés, indifférentes à leur environnement, espérant quelques retombées positives pour la société, elles sont portées par la volonté de contribuer directement à l'intérêt général. C'est pourquoi l'\ess mérite une réelle attention de la part de celles et ceux qui cherchent des solutions et de nouveaux modèles permettant de protéger la planète... et l'humanité elle même. \\

L'objectif poursuivi dans cette thèse est très modeste au regard des enjeux. Elle vise à montrer de quelle manière les \oess peuvent prendre part à la nécessaire transition écologique et comment elles peuvent être soutenues dans cette démarche. Pour cela, nous avons opté en premier lieu pour une approche assez distante, en analysant la prise de parole des organisations sur un réseau social. Nous avons ainsi identifié différentes approches des nombreuses thématiques qui constituent la question environnementale. La problématique de la préservation de l'environnement soulève, naturellement, de fortes inquiétudes. De nombreuses organisations jouent un rôle de lanceurs d'alerte, rappelant sans cesse l'urgence et la nécessité d'agir maintenant et de manière percutante. D'autres organisations discutent davantage des solutions à mettre en oeuvre et identifient des opportunités, des perspectives positives. Dans un deuxième temps, en nous rapprochant du terrain, nous avons souhaité comprendre comment les questions environnementales façonnent les entreprises de l'ESS. Qu'est-ce qui les pousse à agir pour préserver la nature et quels moyens peuvent être mis en oeuvre ? Là encore, des disparités importantes émergent. L'action environnementale est parfois inexistante et le sujet écarté. Elle est quelquefois présente de manière minimale, par sentiment d'une obligation, d'une responsabilité écologique. Mais elle est également portée de manière très affirmée par certaines organisations, qui profitent de leurs spécificités pour se comporter en pionnières de l'action environnementale.

\subsection*{Contributions}
La thèse prétend, humblement, apporter sa contribution à la recherche scientifique. Du point de vue théorique, nous avons cherché à synthétiser les multiples approches de l'\ess à travers le concept d'identité organisationnelle. Nous complétons la littérature en proposant de considérer le caractère collectif d'organisations de l'\ess comme une identité à part entière. Nous introduisons également l'identité fonctionnelle, déjà observée par \textcite{young2001organizational, young2000alternative}, et qui caractérise les organisations concentrées sur leur mission plutôt que sur les modalités d'action. L'ajout de ces deux dimensions permet de mieux rendre compte de la diversité de l'ESS et de ne pas se limiter à une opposition entre des organisations reposant sur le marché versus des organisations resposant sur des valeurs. Cette perspective, en effet, nous semble limitée dans la mesure où le caractère utilitariste des organisations est parfois contraint. Il existe ainsi de grandes disparités entre des entreprises portées par la conviction que la rentabilité issue d'une activité marchande est le meilleur moyen d'atteindre leurs objectifs et des entreprises ayant adopté des pratiques de marché par contrainte et par souci de pérennité. \\

Sur le plan méthodologique, nous expérimentons une méthode encore peu mise en pratique dans le champ de la recherche en sciences de gestion, celle de l'exploration automatique de texte. Nous utilisons les technologies informatiques pour collecter et analyser automatiquement un volume considérable de données. A l'aide d'outils de classification ou d'algorithmes d'apprentissage machine, nous parvenons à extraire du sens d'un corpus trop important pour permettre un codage manuel et à en tirer des conclusions. Si l'immense masse de données disponible sur les serveurs répartis sur la planète est parfois désignée comme le nouvel or noir du \siecle{21}, c'est principalement à des fins marketing qu'elle est aujourd'hui exploitée. Nous pensons que la recherche scientifique peut, elle aussi, s'emparer de cette opportunité. Les outils nécessaires sont accessibles et nous semblent pouvoir donner lieu à des recherches passionnantes dans de multiples disciplines des sciences de gestion. \\

Enfin, sur le plan managérial et sur le plan sociétal, cette thèse invite à une réflexion approfondie sur la place et le rôle de l'\ess dans l'économie. Nous menons celle-ci en laissant de côté la neutralité et le détachement imposés par une posture post-positiviste, et que nous nous sommes efforcés de maintenir tout au long de la conduite de cette recherche. Nous suggérons de ne pas systématiser les logiques de concurrence et plaidons au contraire pour le maintien d'une économie solidaire engagée.


\subsection*{Sortir des logiques de concurrence et repenser la société}

L'ESS échappe de plus en plus difficilement à la concurrence qui fait rage dans la sphère économique. Loin de la concurrence pure et parfaite théorisée par les libéraux, il s'agit d'une course au profit qui se fait au détriment des populations pauvres et des nations les moins riches. Alors que l'ESS voudrait constituer une alternative à ce système déletère, d'aucuns aimeraient voir ses organisations s'inspirer davantage du système capitaliste, plus performant, plus efficace. Les subventions disparaissent peu à peu et les organisations sont contraintes de se regrouper, au détriment de la proximité et de l'ancrage territorial qu'elles pouvaient avoir. La légitimité même des mécanismes de redistribution, de réciprocité ou de solidarité sont contestés, au profit des seuls échanges marchands \parencite{eynaud2019mobiliser}. Toute activité qui n'est pas rentable doit être ré-évaluée et corrigée. Bien sûr, il n'est pas absurde parler d'utilité ou d'efficacité dans le contexte de l'ESS. Celle-ci ne doit pas se contenter d'être une économie de bonnes intentions dont, paraît-il, l'enfer est pavé. Les organisations, bien sûr, doivent s'efforcer de rendre le meilleur service possible à la société et d'utiliser au mieux les moyens qui leur sont apportés. Mais quel sens cela a-t-il si cela implique d'imposer de mauvaises conditions de travail aux salariés, de négliger le bien être des bénéficiaires ou encore de renoncer à des projets plus écologiques, mais moins rentables ? Pour sortir de la crise environnementale qui débute, il s'avère assez incongru de pousser les organisations à imiter le système qui en est la cause. \\

Quel compromis peut être trouvé ? La méfiance à l'égard des \oess, accusées parfois de gaspiller inutilement les fonds publics, peut aisément être contrebalancée par le développement des logiques collaboratives. Les modèles coopératifs comme les \scic, qui donnent voix au chapitre à des parties prenantes variées, en sont un bon exemple. Non seulement cela favorise une transparence et rend possible un contrôle des activités de l'organisation, mais cette démarche renforce aussi sa capacité à faire face aux difficultés. Notre recherche souligne en effet à quel point la dimension collective favorise l'action environnementale. Une autre voie est décrite par \textcite{eynaud2019mobiliser}, dans la continuité de la pensée de Polanyi, Weber ou Tocqueville. Elle propose de \citit{réhabiliter le projet solidaire de l'économie}, c'est-à-dire de re-légitimer les mécanismes de redistribution et de réciprocité au sein de l'économie. Sa réflexion conduit également à se rappeler que l'humanité n'est pas vouée qu'à calculer, mais qu'elle peut faire preuve d'une \citit{rationalité substantive}, la conduisant à s'attacher à d'autres valeurs comme le travail désintéressé ou la vie associative.



\transition
\citit{Notre maison brûle}. C'est par ces mots que le Président Chirac ouvre son discours au sommet mondial du développement durable à Johannesburg en 2002, 10 ans après le sommet de Rio qui posait les bases du développement durable. Au moment où nous achevons ce travail de longue haleine, notre maison, en Amazonie comme en Afrique subsaharienne, est littéralement prise par les flammes. Les réponses politiques à cet enjeu planétaire sont dérisoires et chaque année l'impact de l'humanité sur la planète se fait ressentir un peu plus. Parce qu'il serait  trop facile de céder au pessimisme et à la fatalité, nous concluons en soulignant qu'il existe cependant des éléments encourageants. Si les décisions internationales se font attendre, de nombreuses initiatives se développent au niveau local. De nouveaux modes de consommations émergent, relocalisant la production, et favorisant en même temps les liens entre habitants d'un même village ou d'un même quartier. Monnaies locales, agro-écologie ou covoiturage sont autant d'activités dans lesquelles l'\ess prend toute sa place et tout son sens. Plutôt que de vouloir la diluer dans la sphère des échanges marchands, nous proposons au contraire de la pousser à mettre en oeuvre cette \citit{double solidarité} évoquée par Philippe Eynaud : \citit{celle qui relie les hommes et la nature, et celle qui unit les hommes entre eux}. Il ne s'agit pas là d'une douce rêverie, mais au contraire d'un réel projet de société, un projet économique et un projet pour la planète. Car le véritable intérêt de l'Economie Sociale et Solidaire ne réside pas dans quelques différences structurelles, mais bien dans sa capacité à créer une dynamique porteuse d'un changement de société.
