% \begin{quotation}
%   \citit{Notre maison brûle et nous regardons ailleurs. La nature, mutilée, surexploitée, ne parvient plus à se reconstituer et nous refusons de l'admettre. L'humanité souffre. Elle souffre de mal-développement, au Nord comme au Sud, et nous sommes indifférents. La terre et l'humanité sont en péril et nous en sommes tous responsables.}
% \end{quotation}

% C'est sur ces mots que le Président Chirac ouvre son discours au sommet mondial du développement durable à Johannesburg en 2002, alors que l'écologie politique rassemble encore un électorat très faible \footnote{Noël Mamère, candidat des Verts, obtient 5.25\% des voies au premier tour de l'élection présidentielle la même année}.

\begin{quotation}
\Large
\centering
    \citit{L'humanité se trouve à un moment crucial de son histoire.}\\
\end{quotation}

C'est par ces mots que débute le texte de l'Agenda 21\footnote{\textcite{united_nations_conference_on_environment_and_development1992agenda}}, ratifié en 1992 par 120 chefs d'État à Rio de Janeiro. Ce document vise à fixer les objectifs d'un développement durable. Il souligne la nécessité d'intégrer les questions d'environnement et de développement, afin d'améliorer le sort de tous et de garantir un avenir aux prochaines générations. \\

Presque trois décennies plus tard, l'humanité se trouve toujours à un moment crucial de son histoire, mais semble avoir peu avancé. La prise de conscience des enjeux de la protection de la nature a gagné du terrain, mais les oppositions se font aussi plus fortes. L'écologie est aujourd'hui sujette à un débat extrêmement polarisé au niveau international. Les efforts à réaliser pour lutter contre le réchauffement climatique, l'accumulation des déchets sur la planète, l'épuisement des ressources naturelles et la disparition d'une partie de la biodiversité semblent vertigineux. C'est pourquoi certains - jusqu'à la présidence de l'État le plus riche de la planète - se réfugient dans un déni de la réalité, alors que d'autres envisagent dès à présent de coloniser de nouvelles planètes, espérant ainsi échapper au sort que l'humanité réserve à la sienne. Face à l'urgence, les réponses sont souvent légères, voire symboliques. Les démarches environnementales des entreprises tournent fréquemment au \citit{green-washing} et les engagements politiques peinent à se refléter dans les décisions prises. \\

Peut-être en raison du manque de contrôle de l'humanité sur cette question brûlante, celle-ci est laissée au second plan, au profit de sujets en apparence plus concrets tels que l’emploi ou la croissance. Alors que certains réclament une action collective et surtout politique, d’autres en appellent plutôt à des changements de mentalité au niveau individuel. Ce serait donc aux citoyens de se comporter de façon plus responsable et aux entreprises de limiter leur impact (avec une coercition relativement limitée). Si la sphère économique reste majoritairement ancrée dans une perspective de croissance et de maximisation des profits, ses acteurs perçoivent toutefois un intérêt à prendre en compte l’environnement dans leurs pratiques… et dans leur communication. Dans le cadre de la \rse, présenter un visage éco-responsable permet de répondre aux attentes de différentes parties prenantes et de maintenir sa légitimité. Dans une perspective institutionnelle, on observe également un effet de mimétisme : négliger l’environnement quand les concurrents agissent constitue en effet un risque d’image important. Certaines évolutions législatives ont également contraint les entreprises à évoluer, notamment dans les secteurs ayant un fort impact environnemental. Enfin, dans la lignée de l’hypothèse de Porter, les entreprises identifient des avantages économiques à adopter un positionnement écologique : réduction des coûts de production, développement de nouveaux produits ou encore accès à de nouveaux marchés. Ainsi, la plupart des grandes entreprises consacrent une part de leurs investissements à l’innovation environnementale. \\

L’\textbf{ Économie Sociale et Solidaire (ESS) } constitue un segment particulier de l’économie qui se réclame de l'intérêt général plutôt que de la recherche du profit individuel. Elle vise à apporter des solutions à des problèmes sociaux ou sociétaux. Cette économie s’inscrit en cohérence avec l’idée d'un \dd qui prend en compte les dimensions économique, sociale et environnementale de l’activité humaine. L'\ess bénéficie donc d’un présupposé positif quant à son impact, et notamment son impact environnemental. En outre, l'\ess est depuis longtemps positionnée sur les questions d'environnement \parencite{cretieneau2010economie}, de nombreux acteurs de l’\ess se consacrent entièrement à une mission écologique. Ils mettent en avant la contribution que cette économie peut apporter à la transition écologique. Pour autant, comment l'\ess s'inscrit t-elle dans \citit{le processus de transition} vers \citit{une économie écologique et sociale} décrit par \textcite{waridel2016economie} ? A nouveau, les faits sont résistants au discours. Seules de rares études se sont penchées en détail sur cette question et nous ne disposons ainsi que de peu de connaissances sur la prise en compte de l'environnement dans l'ESS. Les auteurs suggèrent que cette dimension n'est pas suffisamment considérée \parencite{edwards2013environmental,buchs2014role} alors même que le secteur aurait un impact non négligeable \parencite{dart2010green}. Au delà de certaines initiatives médiatisées, la prise en compte de l'environnement dans l'\ess reste en question, en particulier au sein des organisations dont l'activité n'est pas liée à cette thématique. \\

L'\ess est souvent méconnue voire ignorée. Pourtant, elle est fortement ancrée dans l'histoire et dans l'économie française comme dans de nombreux autres pays, quoique sous des formes différentes. Son poids dans l'économie est significatif et elle constitue un réel bassin d'emploi, dans de très nombreux secteurs d'activité. Selon la loi sur l'\ess de 2014, ce segment de l'économie regroupe les associations, coopératives, mutuelles, fondations et entreprises sociales. Fortement dépendante de ses rapports avec l'État \parencite{archambault2001historical}, l'\ess est aujourd'hui soumise à des critiques et une incitation croissante à se rapprocher du modèle capitaliste. Or ceci entre souvent en contradiction avec les valeurs et les principes de fonctionnement adoptés par les organisations. Le modèle de l'entreprise sociale, basée sur une activité économique rentable, mais tournée vers une mission sociale plutôt que vers le profit, tente de résoudre cette contradiction. Cependant, les entreprises sociales restent extrêmement minoritaires dans un secteur dominé par les associations. \\

L'expression \cit{économie sociale et solidaire} est très spécifique au contexte français. Dans de nombreux pays européens, on désigne la même réalité par le terme \cit{d'économie sociale}. L'approche anglo-saxonne est un peu différente, puisqu'elle distingue le secteur non lucratif (\citit{nonprofit}) de l'économie coopérative ou mutualiste. Dans les pays du Sud, on emploie plus volontiers le vocable \cit{d'économie populaire} qui désigne une économie informelle. Celle-ci n'est pas intégrée à la définition européenne de l'ESS. Afin de ne pas ajouter de confusion à un concept qui peine à faire consensus, nous suivons la définition institutionnelle de l'\ess en France, également partagée par le \cncres \parencite{cncres2015panorama}. Elle repose essentiellement sur les formes juridiques prises en compte dans le périmètre. Le parti pris du \cncres est de parler \cit{d'entreprises} plutôt que \cit{d'organisations} ou de \cit{structures}. En effet, la loi de 2014 décrit l'\ess comme un \cit{mode d'entreprendre}. Cependant, nous observons que certaines organisations, notamment des associations ou des fondations, ne se reconnaissent pas pleinement dans ce terme, voire le rejettent ouvertement. Ce débat est légitime, puisque le mot \cit{entreprise} n'a pas de définition juridique (la loi parle de sociétés de capitaux ou de sociétés de personnes). Ainsi, l'emploi exclusif de ce terme revient à prendre un parti et à adhérer à une vision capitaliste de l'économie sociale. Or, la vision dans laquelle s'inscrivent les organisations est susceptible d'influer sur leur approche de la protection de l'environnement. C'est pourquoi nous choisissons de garder une approche ouverte et plus neutre en utilisant alternativement les termes d'OESS, d'\eess ou de structures de l'\ess pour désigner les composantes de l'ESS. A l'inverse, le segment des organisations n'appartenant ni à l'ESS ni au secteur public est généralement désigné comme \textbf{\cit{l'économie classique}}. Nous utilisons également et sans distinction spécifique les termes \cit{économie de marché}, \cit{économie capitaliste}, \cit{secteur à but lucratif} ou, par abus de langage, \cit{secteur lucratif}. Nous parlons ainsi d'organisations ou de structures de l'économie classique, ou bien \cit{d'entreprises classiques} ou \cit{d'entreprises capitalistes}. \\

Nous expliquons dans le premier chapitre que l'\ess peut être conceptualisée (1) comme un secteur de l'économie, un ensemble d'organisations, ou bien (2) comme une vision alternative de l'économie, ou enfin (3) comme un capitalisme social. Le concept d'\textbf{identité organisationnelle}, soit la réponse à la question \cit{qui sommes nous en tant qu'organisation ?}, nous permet de nous extraire de ce débat. Il se concentre sur ce qui est réellement central dans les organisations et ne nécessite pas de préjuger d'une meilleure approche de l'ESS. La littérature relève le caractère souvent hybride des \oess qui concilient plusieurs identités, parfois proches de l'économie classique, parfois centrées sur des valeurs et des principes. \\

Parce qu'elles n'œuvrent pas dans la seule perspective de faire des profits, les \eess bénéficient d'un a priori positif et d'un capital de confiance \parencite{hansmann1980role}. C'est la raison pour laquelle elles sont souvent perçues comme vertueuses en matière d'environnement \parencite{cretieneau2010economie}. Puisqu'elles agissent dans le sens de l'intérêt général, il semble naturel qu'elles agissent en considération de leur impact sur la nature. En outre, l'\ess présente des points de convergence avec le \textbf{Développement Durable}, notamment sur le volet environnemental. Pourtant, l'\ess regroupe des secteurs d'activité ayant un réel impact sur l'environnement \parencite{dart2010green}. On peut dès lors se questionner sur la prise en compte de l'environnement naturel dans l'ESS. \\

Nous abordons cette question en deux temps. Dans un premier temps, nous adoptons l'angle de la communication. Bien que le discours ne présume pas des actes, la manière d'appréhender le sujet de la protection de l'environnement nous informe sur le positionnement des organisations vis-à-vis de ces questions. En outre, la préservation de l'environnement implique la diffusion des connaissances et la sensibilisation des individus. La première des deux études qui constituent le coeur de la thèse se penche sur les \textbf{stratégies rhétoriques} des \oess en matière d'environnement. Ce concept n'est pas récent, puisqu'il a fait l'objet d'écrits d'Aristote. Le terrain d'étude que nous avons choisi est toutefois bien éloigné de celui du philosophe grec : il s'agit d'un réseau social en ligne, Twitter. Dans cet espace numérique, les organisations interagissent entre elles, mais aussi avec les citoyens, ou encore les représentants des pouvoirs publics. Chacun peut s'y exprimer sur une large variété de thématiques, dont l'écologie et la protection de l'environnement. Cependant, la focalisation sur la communication comporte un risque : celui du \textit{green-washing}. Les procédés de communication qui visent à se donner une image responsable et écologique en minimisant les actions concrètes (afin de limiter les coûts), peuvent constituer un biais. C'est pourquoi la seconde étude menée dans le cadre de la thèse concerne \textbf{l'action environnementale}. \\

Celle-ci désigne l'ensemble des efforts réalisés par les entreprises pour limiter leur impact sur l'environnement. Les actions de protection de la nature répondent à \textbf{plusieurs motivations} : elles peuvent être d'ordre économique et viser à réduire les coûts ou à augmenter la valeur ajoutée perçue des biens et services. Elles répondent aussi à un enjeu de légitimité vis-à-vis des parties prenantes de l'entreprise ainsi que de la société dans son ensemble. Enfin, elles s'inscrivent  dans le cadre de la \rse et font suite à la perception en interne d'une exigence éthique. L'action environnementale est étroitement liée à l'innovation. Elle requiert en effet des changements de plus ou moins grande ampleur au sein de l'organisation. L'innovation ayant pour but de réduire l'impact environnemental des entreprises est qualifiée  \cit{d'innovation environnementale} ou \textbf{éco-innovation}. Elle peut être adoptée de manière pro-active par les organisations ou bien suivre une démarche adaptative.


\subsection*{Problématique}

Dans cette recherche exploratoire, nous cherchons à comprendre le positionnement de l’\ess vis-à-vis des questions environnementales. Nous posons donc la problématique suivante : \\

\begin{tcolorbox}
  \textbf{Quelles sont les stratégies d’action et de communication des entreprises de l’ESS face à l'enjeu de la protection de l'environnement ?}
\end{tcolorbox}

Plus précisément, nous cherchons d'abord à identifier les principales thématiques environnementales sur lesquelles se positionnent les associations, mutuelles, fondations, coopératives et entreprises sociales. Nous visons également à mettre en évidence différents modes de communication et différentes stratégies rhétoriques adoptés par ces acteurs. Ensuite, nous nous interrogeons sur les facteurs favorables ou défavorables à l’action environnementale dans ce segment particulier de l’économie. Nous nous efforçons finalement de déterminer s'il existe un lien entre l'identité des organisations et leur démarche d'action environnementale.

\subsection*{Plan de la thèse}

La thèse s'articule en trois parties et sept chapitres.\\

\textbf{La première partie} formule une synthèse de l'état de l'art et introduit le cadre théorique de la recherche. Le premier chapitre est consacré à la conceptualisation de l'\ess et en présente les différentes perspectives. Le second chapitre est focalisé sur l'action et la communication environnementale. Il traite également de la place de ces questions dans le cadre de l'ESS, en revenant sur les liens entre cette économie et le mouvement du développement durable. \\

\textbf{La seconde partie } développe le cadre scientifique de la thèse. Le chapitre \ref{chapitre:demarche} introduit une réflexion épistémologique mais aussi éthique sur la recherche menée. Le chapitre \ref{chapitre:methodes} porte sur les méthodes mobilisées à travers une étude quantitative et une étude qualitative, qui sont menées parallèlement, suivant un modèle de méthodes mixtes. \\

\textbf{La troisième partie} se consacre aux résultats de la recherche. Les deux études sont traitées dans des chapitres distincts (chapitres \ref{chapitre:twitter} et \ref{chapitre:casess}). Leurs résultats sont ensuite confrontés et discutés dans le dernier chapitre. Celui-ci propose également une prise de recul sur la validité, les limites et les perspectives de la thèse.
