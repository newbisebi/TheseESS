\label{annexe:guide}
\section*{Introduction}
    \textbf{Présentation du chercheur et du contexte de la recherche}
    \begin{itemize}
        \item Recherche menée à Aix-Marseille Université dans le cadre d’une thèse en gestion des organisations.  
        \item Travail sur la prise en compte de l’environnement dans l’ESS
        \item Retour sur les entretiens à l’issue de l’analyse et avant la soutenance de thèse (prévoir un délai d’un an) \\
    \end{itemize}
    	
    \textbf{Objectifs de l’entretien : }
        \begin{itemize}
            \item Comprendre le fonctionnement de votre entreprise.
            \item Comprendre ce qui motive ou freine vos décisions en matière environnementale. \\
        \end{itemize}
        

    \textbf{Les principes : } \\
        Posture d’un entretien de recherche  assez codifié : 
            \begin{itemize}
                \item Questions ouvertes, possible d’extrapoler. Les questions ne servent qu’à guider l’entretien.
                \item Ne pas hésiter à expliquer des choses qui peuvent paraître évidentes (pour avoir votre point de vue).
                \item Positionnement neutre : pas de jugement, pas de bonne ou mauvaise réponse. Pas de prise de position du chercheur pendant l’entretien, mais possible d’avoir un échange à la fin de l’entretien.  \\
            \end{itemize}
        L’entretien est enregistré, cependant l’anonymat est garanti. Il est possible d’interrompre l’enregistrement à votre demande si un sujet confidentiel est abordé. Le nom de personnes éventuellement citées pendant l’entretien sera changé dans la retranscription.

        
\section*{Présentation de l’entreprise et du répondant }
Pouvez-vous me parler de votre organisation/entreprise ? 
    \begin{itemize}
    	\item Quel est le statut juridique de votre organisation/entreprise ? 
    	\item Effectif, année de création, initiative de la création ? \\
	\end{itemize} 

Pouvez-vous me parler de votre position, votre rôle dans l’organisation/entreprise ? \\

Comment décririez-vous la mission de votre organisation/entreprise ?  
    \begin{itemize}
        \item Quelles sont ses activités principales ? A qui s’adressent-elles ? 
        \item Qui sont vos client / bénéficiaires ? Ont-ils un statut particulier (adhérent…)
        \item Quel type d’impact cherchez-vous à avoir ? (social, environnemental, économique) \\
    \end{itemize}
	
Quel est votre périmètre d’action ? (local / national / international) 

\section*{Caractéristique de l’organisation/entreprise}
Qui sont vos employés ? 
\begin{itemize}
    \item Bénévoles / salariés
    \item Type de contrats  \\
\end{itemize}

A qui appartient l’organisation/entreprise et qui la dirige ?
    \begin{itemize}
        \item Structure du groupe et lien mère-filiales ?
        \item Comment et par qui sont prises les décisions ? 
        \item Structure centralisée / décentralisée ? 
        \item Organisation verticale ou horizontale ? (hiérarchie)
        \item Qui est représenté au conseil d’administration ?  \\
    \end{itemize}

Vous appuyez vous sur des partenariats ? 
\begin{itemize}
    \item Qui sont vos partenaires clés ? 
    \item Sur quelles bases reposent ces partenariats ? 
    \item Quel intérêt avez-vous dans ces partenariats ? 
    \item Que leur apportez-vous ? 
    \item Quelle importance donnez-vous à ces partenariats ?  \\
\end{itemize}
	
Quelles sont les sources de revenus de votre organisation/entreprise ? (Dons d’entreprises ? de particuliers ? Mécénat ? Subventions ? Chiffre d’affaires ? Revenus publicitaires ?)
    \begin{itemize}
        \item Revenus réguliers ou irréguliers ? 
        \item D’où viennent les capitaux ? (Capital social, crédit, investissement solidaire…) \\
    \end{itemize}

Quels sont les principaux postes de dépense ? 

Êtes vous en concurrence avec d’autres organisations, et comment vous situez vous dans cette concurrence ? 
    \begin{itemize}
        \item En termes de prix ? 
        \item Parts de marché ? 
        \item Logique de domination de marché, de masse ?  \\
    \end{itemize}
	
Est-ce que vous évaluez les résultats de votre organisation/entreprise ? 
    \begin{itemize}
        \item Sur quels critères ? \\
    \end{itemize}

Comment décririez-vous les valeurs de votre organisation/entreprise ? 
    \begin{itemize}
        \item Qu’est ce qui est le plus important pour vous ? \\
    \end{itemize}

Quelles sont les perspectives et projets de votre organisation/entreprise ? 
    \begin{itemize}
        \item Croissance ? 
        \item Nouveaux marchés ? 
        \item Nouvelles actions ?
        \item Etc. \\
    \end{itemize}
	
Avez-vous le sentiment d’appartenir à l’ESS ? 
    \begin{itemize}
        \item Qu’est ce que ça veut dire pour vous ? 
        \item Est-ce que cela vous semble important ? 
        \item Qu’est ce qui vous y rattache le plus ? 
        \item Qu’est ce que cela change pour vous ? \\
    \end{itemize}

\section*{Innovation environnementale}
Pensez-vous que l’activité d’une organisation comme la vôtre a un impact environnemental (direct ou indirect) ?
    \begin{itemize}
        \item Quel type d’impact ? 
        \item Êtes-vous soumis à des normes environnementales ? 
        \item (Si applicable : Au-delà de vos activités, votre fonctionnement a-t-il un impact ?) \\
    \end{itemize}

Avez-vous une stratégie / une réflexion sur ces sujets ?  
    \begin{itemize}
        \item Cette réflexion est-elle prioritaire ? \\
    \end{itemize}
	

Avez-vous déjà mené des actions dans une perspective environnementale ? 
    \begin{itemize}
        \item Ou avez-vous comme projet de mener de telles actions dans le futur ?  \\
    \end{itemize}
	

Ces actions ont-elles nécessité de mettre en place des adaptations, des changements, des innovations ? \\

Sur quoi ont porté ces actions ? 
    \begin{itemize}
        \item Sur les produits et services ?
        \item Sur l’organisation ? 
        \item Rapport au consommateur ?  
        \item Systèmes de management ? Procédures ? Certifications ?  \\
    \end{itemize}


S’agissait-il plutôt d’adaptations de l’existant ou de l’introduction de quelque chose de complètement nouveau pour vous ? \\

Qu’est-ce qui vous a poussé à réaliser ces innovations avec un impact environnemental ? 
    \begin{itemize}
        \item Recherche de rentabilité
        \item Demande du client 
        \item Amélioration de l’offre
        \item Evolution légale 
        \item ... \\
    \end{itemize}

Avez-vous rencontré des obstacles à la mise en œuvre de ces innovations ? 
    \begin{itemize}
        \item Financiers
        \item Organisationnels
        \item Humains
        \item ... \\
    \end{itemize}
    
Au contraire, qu’est-ce qui a facilité la mise en œuvre de ces innovations ? 
    \begin{itemize}
        \item Organisation
        \item Partenariats
        \item Humains \\
    \end{itemize}

Quels effets attendiez-vous de ces innovations ?
    \begin{itemize}
        \item Avez-vous obtenu les résultats escomptés ? 
        \item Avez-vous évalué l’impact de ces innovations ? \\
    \end{itemize}
	
	

Qu’est-ce qui a guidé votre choix vers certaines innovations plutôt que d’autres ? \\ 

Pensez-vous qu’appartenir à l’ESS soit un atout pour mettre en œuvre des innovations en lien avec l’environnement ? \\

Avez-vous de nouveaux projets d’innovations en lien avec l’environnement ?  \\

Avez-vous des documents relatifs à votre organisation/entreprise (rapports annuels) et à votre action environnementale (rapports RSE…) \\

\section*{Conclusion}
Principaux points abordés \\

Souhaitez-vous ajouter quelque chose ? \\

Remerciements 
