Dans ce chapitre, nous avons présenté la démarche méthodologique de la thèse. Elle vise à répondre à la problématique : \textbf{Quelles sont les stratégies d’action et de communication des entreprises de l’ESS face à l'enjeu de la protection de l'environnement ?} Elle s'articule autour de deux études, permettant d'aborder cette question à travers l'angle de la communication organisationnelle d'une part, et celui des pratiques, notamment d'innovation, de l'autre. \\

La première étude repose sur une méthode encore peu utilisée en sciences de gestion, celle de l'exploration automatique de texte. L'exploration de texte mobilise les technologies récentes pour implémenter des algorithmes supervisés (c'est à dire assistés par le chercheur) ou non supervisés permettant d'extraire du sens d'un grand volume de texte. Dans une approche exploratoire, nous mobilisons un panel d'outils de l'exploration de texte et tirons partie de leur complémentarité.
Au cours des dernières années, un certain nombre de chercheurs ont utilisé les outils de programmation informatique pour collecter des données venant d'internet et les analyser de manière plus souple. En effet, la programmation, si elle a un coût d'entrée plus élevé, offre l'avantage de produire des traitements véritablement "sur mesure" et de donner un contrôle complet au chercheur sur la méthode qu'il met en oeuvre. Nous nous proposons ici de mettre en oeuvre cette méthode pour collecter et étudier des données de Twitter, un réseau social sur lequel communiquent de nombreuses entreprises, notamment de l'ESS. Cette étude vise à répondre à une première partie de la problématique. Comment les \eess communiquent-elles sur les questions environnementales ? À quels enjeux écologiques attachent-elles une plus grande importante ? Et enfin, y a-t-il, comme on peut l'attendre, différentes approches au sein de l'ESS, et quelles sont-elles ? \\

La seconde étude veut dépasser la vision purement rhétorique et s'appuie sur des éléments de terrain correspondant à ce qui est effectivement mis en place par les \eess. Elle mobilise une approche plus commune dans notre discipline, celle de l'étude de cas, à partir d'entretiens semi-directifs. Le choix est fait de concentrer l'étude sur un nombre réduit de cas. Une approche qualitative, en effet, ne peut rendre compte intégralement de la grande diversité des organisations de l'ESS. Cependant, elle permet de comprendre de manière précise ce qui motive l'action environnementale dans les entreprises étudiées et d'analyser ce qui fait leur identité. A partir d'un panel restreint, on peut ainsi faire ressortir des liens entre les dimensions qui caractérisent les \eess (au delà de leur simple statut), et les déterminants et leviers de l'action environnementale. \\

Dans la partie qui suit, nous présentons et discutons des résultats empiriques de la thèse, en commençant par l'étude de la communication sur Twitter (chapitre \ref{chapitre:twitter}), suivie de l'étude de cas (chapitre \ref{chapitre:casess}). Suivant un schéma de méthodes mixtes menées de manière parallèle, les résultats de chaque études sont analysés séparément, avant d'être confrontés dans une discussion générale (chapitre \ref{chapitre:discussion}).
