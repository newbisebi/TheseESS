Dans ce chapitre, nous avons présenté les résultats d'une étude qualitative ayant pour objectif de questionner les motivations à l'action environnementale dans l'ESS. La diversité du secteur ressort pleinement de l'étude, alors même que toutes les formes d'organisations n'y sont pas représentées. Plusieurs conclusions peuvent être tirées. \\

Tout d'abord, nous validons l'usage de quatre identités organisationnelles pour caractériser les \oess. Conformément à nos attentes, ces identités peuvent cohabiter, avec généralement une prédominance pour l'une d'elles. Alors que l'\ess est souvent définie par ses valeurs, la dimension normative n'est pas systématiquement au coeur de l'entreprise. Bien entendu, cela ne signifie pas que les \eess en question n'adhèrent pas aux valeurs de l'ESS, mais qu'elles ne se définissent pas et ne s'organisent pas autour de ces valeurs. Leur identité est parfois éloignée des éléments qui sont souvent mis en avant dans la communication promotionnelle de l'ESS. Inversement, la dimension utilitariste occupe dans certaines organisations une place centrale. La performance, la croissance et le management sont des termes parfaitement acceptables et peuvent faire partie intégrante de l'identité des \eess. Cette dimension est parfois volontaire, résultant de conviction que la mission de l'entreprise est mieux servie par une démarche utilitariste, mais aussi parfois subie, en raison de la pression financière qui pèse sur les organisations. En revanche, il est intéressant de noter qu'il existe une dimension \cit{anti-utilitariste}, à savoir des organisations qui rejettent frontalement l'approche capitaliste et managériale de l'entreprise, et adoptent des pratiques à l'opposé de celles du secteur lucratif. Pour certaines organisations, généralement protégées de la concurrence du marché, l'identité fonctionnelle est mise en avant. L'appartenance à l'\ess est statutaire plutôt que revendiquée. L'attention est donnée à la mission d'intérêt général ou collectif plutôt qu'aux modalités de sa mise en oeuvre. Enfin, des organisations affirment leur place dans l'\ess par leur identité collective. Celle-ci se manifeste par l'implication des salariés, voire leur engagement dans les organes de gouvernance, mais aussi par une approche privilégiant les partenariats à la concurrence ou aux rapports clients-fournisseurs classiques. Dans certains cas, l'ouverture se fait à des parties prenantes externes, via les instances de gouvernance ou via l'engagement bénévole ponctuel ou régulier. \\

L'étude valide ensuite la pertinence des modèles d'analyse des déterminants de l'action environnementale existant dans la littérature. L'approche de \textcite{bansal2000why}, qui retient les facteurs de compétitivité, de légitimité et de responsabilité environnementale peut être utilisée. Ce cadre d'analyse nécessite toutefois la prise en compte des spécificités de l'ESS. En particulier, le rôle du collectif, absent des principaux modèles,  est déterminant dans l'action environnementale des \oess. L'adoption d'une démarche collective est bénéfique en termes de ressources et de compétences et permet aux organisations de dépasser les barrières qu'elles rencontrent. De même, l'engagement personnel et collectif, l'adhésion à des valeurs fortes constituent un levier puissant d'action environnementale. Ces facteurs poussent les organisations à rechercher un véritable impact et à mener une action vraiment novatrice, là où les facteurs de compétitivité ou de légitimité sont plutôt vecteurs d'adaptation \textit{a minima}. \\

Enfin, l'action environnementale est rendue difficile par le contexte dans lequel évoluent les organisations. La réduction des financements et les pressions à adopter des \textit{business models} plus rentables sont évoquées par presque tous les répondants. Or les opportunités économiques liées à l'action environnementale ne semblent pas assez incitatives et les efforts nécessaires sont conséquents. Pourtant, les organisations activement engagées dans un processus d'innovation environnementale font état de réels bénéfices. \\
