Dans ce dernier chapitre de la thèse, nous avons cherché à mettre en perspective les résultats de deux études, au regard des enseignements de la littérature. A travers une approche pragmatique de l'ESS, nous montrons comment les organisations peuvent mettre en place une action environnementale et contribuer au processus nécessaire de transition écologique. \\

La discussion porte sur les différentes façons dont l'action environnementale est appréhendée dans l'ESS. Elle met l'accent sur la grande richesse que représente la variété des structures organisationnelles de cette économie et sur la complémentarité des approches qui en résultent. Les structures peuvent s'appuyer sur l'engagement et les convictions personnelles de leurs membres, mais aussi sur la force d'une démarche collective qui est au coeur des principes de l'ESS. \\

Cependant, la variété des approches présentes au sein des \oess, que nous abordons à l'aide du concept d'identités organisationnelles et d'hybridité des identités, ne suffisent pas à expliquer la mise en oeuvre d'une action environnementale. Ce ne sont pas les spécificités de l'\ess en soi qui font que les organisations agissent pour la protection de l'environnement. Elles confèrent toutefois à l'\ess une véritable capacité à contribuer à réorienter l'économie dans une direction plus écologique. De la même manière que pour la divulgation environnementale, il faut néanmoins distinguer ce qui déclenche l'action environnementale de ce qui facilite la mise en oeuvre. L'\ess présentant de nombreux atouts pour mieux protéger l'environnement, il faut agir sur les déclencheurs et encourager les organisations à s'engager dans une démarche d'action environnementale. En cela, le rôle des grandes organisations de l'ESS, ainsi que des organismes représentatifs, qui occupent une place centrale dans le réseau des \eess, est déterminant. \\

Un volet de ce chapitre est consacré à une réflexion sur les aspects académiques de la thèse. Nous soulignons les contributions apportées, à la fois sur le plan méthodologique, à travers la mise en oeuvre des techniques d'exploration automatique de texte, et sur le plan théorique, en complétant des approches conceptuelles présentes dans la littérature. Nous nous arrêtons sur les limites, inhérentes à toute recherche scientifique. Outre les questionnements liés aux méthodes, c'est la question de la généralisation des résultats qui est posée. La première étude s'inscrit dans le contexte particulier d'un réseau social en ligne, et la seconde retient une approche qualitative, limitée par la taille de l'échantillon. C'est pourquoi la thèse n'est pas \cit{fermée}, mais vise au contraire à ouvrir des perspectives de recherche. \\

La prise en compte de l'environnement dans l'ESS constitue un thème de recherche encore trop peu étudié. Il s'agit pourtant d'une question majeure, à la fois pour l'économie sociale elle-même, mais pour la société en général. Si des réponses aux inquiétudes liées à la dégradation de l'environnement doivent être trouvées, elles peuvent l'être dans ce secteur innovant et fort de nombreux atouts. Le rôle de la recherche est de mettre en lumière ces atouts et d'aider les organisations à les actionner.  