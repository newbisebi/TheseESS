Dans ce chapitre, nous avons présenté les résultats d'une analyse quantitative de contenu textuel visant à questionner la rhétorique des \eess à propos de l'environnement. Il a été montré que de nombreuses organisations se positionnent sur les questions d'écologie, y compris lorsqu'elles ne sont pas concernées directement par ces enjeux. Plusieurs stratégies rhétoriques sont identifiées, cependant celles-ci ne sont pas en soi déterminantes pour atteindre une audience. Le choix du ton du discours et du cadrage rhétorique à adopter doit avant tout correspondre au message que les organisations souhaitent faire passer. Un discours positif et plutôt factuel est ainsi adapté à une démarche entrepreneuriale et une volonté de saisir des opportunités économiques en lien avec les enjeux écologiques. Au contraire, un discours plus négatif et alarmiste est davantage mobilisé par des organisations militantes. Dans une perspective de changement social (vision transformatrice de l'\ess), le débat est recherché et peut mobiliser des arguments politiques. Au contraire, une approche économique favorise un discours qui va permettre de se légitimer et d'éviter les conflits. \\

    L'étude mobilise des méthodes nouvelles et originales. Nous avons insisté sur l'intérêt d'une telle approche et de l'usage de la programmation pour la recherche en sciences de gestion. Ces outils peuvent être mobilisés en substitution ou en complément de logiciels et méthodes déjà communément utilisés. Bien que l'approche retenue ici soit principalement quantitative, les méthodes qualitatives peuvent s'appuyer sur eux pour collecter des données originales. Ainsi, même un nombre important de tweets peut être étudié de manière qualitative. Une étude peut également porter sur un sous-échantillon aléatoire de tweets (par exemple \textcite{mariaux2017promouvoir}). Cependant, l'aspect \cit{technologique} et le recours aux statistiques ne doit pas donner l'illusion d'une objectivité parfaite. Ces méthodes, comme toutes celles auxquelles font appel les sciences humaines, ont des limites dont le chercheur doit être pleinement conscient. L'accès facilité à des données très riches, mais non parfaitement adaptées à l'objet de recherche, peut conduire à se laisser guider par les données, plutôt que de les mobiliser efficacement pour répondre à une question de recherche. Enfin, de nombreuses questions d'éthique sont soulevées par ces méthodes. Il est donc indispensable de se questionner sur la propriété des données, sur l'utilisation qui en est faite et sur l'impact éventuel pour les utilisateurs. L'étude présentée ici s'intéresse à des organisations et aucun compte individuel n'a été analysé. On ne choquera  ainsi personne en soutenant que Green Peace adopte un positionnement engagé sur les questions d'écologie. En revanche, de nombreux utilisateurs s'expriment sur Twitter (ou d'autres réseaux) en leur nom propre et sur des sujets parfois sensibles ou des questions très personnelles. Bien que les tweets soient publics, leur intention peut être de ne s'adresser qu'à leur propre réseau. \\

    Nous nous sommes intéressé en premier lieu au discours des \oess concernant les préoccupations environnementales. Dans le chapitre suivant, nous présentons une seconde étude qui porte au contraire sur les pratiques environnementales.



    % \todo[inline]{Si on a le temps : ajouter \textcite{ross2013common} : discuter de pourquoi on obtient des catégories thématiques différentes (méthode différente, contexte...)}

    % \todo[inline]{Aussi : si possible, quantifier chacun des 6 theme en comptant le nombre total d'occurrences des mots pertinents de chauqe catégorie}


    % \reynaud{L’étude des tweets est bien. La partie vraiment très bien est celle qui donne des exemples de tweets. Ca permet de mieux comprendre.  Je pense qu’il est important de finir votre étude sur Twitter en déterminant ce qui va être gardé pour la suite (soit pour l’étude 2 soit pour l’étude 3).}
