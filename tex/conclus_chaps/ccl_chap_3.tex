Ce bref chapitre ne prétend pas engager une grande discussion sur une conception donnée de la science, ni sur l'éthique scientifique. Plus modestement, il vise à donner un cadre général dans lequel inscrire la thèse. \\

L'adoption du réalisme scientifique comme posture de recherche a plusieurs implications pour la suite de ce travail. Tout d'abord, nous ne prétendons pas construire ou interpréter la réalité, mais au contraire étudier des phénomènes existants indépendamment de la recherche. La thèse est menée avec un certain recul sur le terrain, que nous ne cherchons pas à influencer. Il ne s'agit pas d'accompagner des entreprises dans leur démarche environnementale, mais bien de mettre en évidence des constantes dans les comportements écologiques (ou non) de l'ESS. Naturellement, ceci ne nous condamne pas à l'inaction et les résultats de la recherche ont pour ambition  de comprendre comment une économie autre que le capitalisme libéral peut contribuer à produire une société plus respectueuse de l'environnement. A travers cette thèse, modestement, nous espérons pouvoir y contribuer. Une seconde implication du réalisme scientifique porte sur la méthode. Bien qu'il n'interdise pas l'usage de méthodes qualitatives, il impose de les utiliser d'une manière qui limite le rôle et l'influence du chercheur. Cette posture invite également à approcher le sujet d'étude de différentes manières, ceci afin de réduire les biais inhérents à un instrument de mesure. Enfin, même si nous ne suivons pas une approche hypothético-déductive pour ce travail, nous nous conformons au principe de formulation d'énoncés vérifiables. \\

Comme nous le rappelons à plusieurs reprises, nous aurions tort d'associer naïvement \ess et éthique. Cependant, parler de l'une conduit naturellement à parler de l'autre, car une partie de l'identité de l'\ess repose sur un socle de valeurs qui la distingue, au moins conceptuellement, du capitalisme  désencastré, soucieux uniquement des profits. Bien que l'éthique ne soit pas le sujet de la thèse, il serait incohérent de l'oublier complètement. En outre, la recherche scientifique, y compris la recherche en gestion, pose \textit{de facto} des questions éthiques, parce qu'elle traite de sujets qui façonnent notre société et impactent évidemment les individus. Dans le cadre de cette thèse, nous n'avons pas relevé de problèmes éthiques majeurs. Toutefois, une attention a été portée aux modes de collecte des données et à l'information des personnes interrogées sur l'usage de leurs propos pour cette recherche. \\

Ayant posé le cadre éthique et épistémologique, nous pouvons rentrer, dans le chapitre suivant, dans la présentation des méthodes de recherches utilisées.
