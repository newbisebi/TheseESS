La thèse s'interroge sur la place de l'action environnementale dans l'ESS. Le chapitre \ref{chapitre:ei} retrace la littérature sur la place de l'écologie dans les organisations, et décrit les liens existants avec l'ESS. Les formes de communication sur les questions écologiques nous informent sur la manière dont l'environnement est appréhendé par les organisations. L'étude de la rhétorique montre que ces enjeux peuvent être abordés de plusieurs façons et que les entreprises cherchent à trouver un équilibre entre la poursuite d'activités rentables, la prise en compte des attentes des parties prenantes et leurs propres valeurs. Puisque l'environnement a été une préoccupation secondaire, voire inexistante, durant de nombreuses décennies, sa prise en compte nécessite des changements, des innovations. Celles-ci ont suscité un véritable intérêt des chercheurs au cours des dernières années. Les études ont permis de mettre en évidence les déterminants de l'innovation environnementale, mais aussi les obstacles rencontrés par les organisations. Elles ont aussi montré que l'\ei prend des formes diverses et ne se limite pas au développement et à l'adoption de nouvelles technologies. \\

Quelle attitude adoptent les \eess vis-à-vis de la gestion de leur impact environnemental ? La littérature fait apparaître des éléments prometteurs qui suggèrent une forte adéquation entre les valeurs de l'ESS, mais aussi ses modes de gestion, avec les enjeux écologiques. L'\ess semble constituer un cadre idéal, voire un laboratoire pour la mise en application des principes du développement durable. Cependant, la grande diversité de cette économie, la réalité des dynamiques de terrain et la tentation de faire entrer l'\ess dans le système capitaliste soulève des difficultés. L'écologie n'est pas toujours au coeur des préoccupations des \oess, souvent tiraillées entre une mission sociale pressante et une course forcée aux financements. La question des stratégies environnementales dans l'\ess n'est donc pas tranchée. \\

Pour y apporter des éléments de réponse, nous proposons d'étudier la problématique sous deux angles différents. Celui de la communication, d'une part, à travers une approche quantitative, et celui de la réalité des acteurs de terrain, à travers une approche qualitative de l'autre. Dans la partie \ref{partie:methodo}, nous détaillons les méthodes appliquées pour ces deux études, et expliquons comment le recours à des méthodes mixtes peut contribuer à faire avancer la connaissance dans ce domaine. 
