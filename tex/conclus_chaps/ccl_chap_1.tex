L'\ess n'est pas seulement le segment de l'économie qui regroupe les entreprises sociales, coopératives, associations, fondations et mutuelles. La littérature présentée dans ce chapitre dresse un tableau plus complexe, sans définition consensuelle, mais au contraire marqué par des disparités culturelles et idéologiques. Les auteurs l'appréhendent au sens large, à travers la notion de \cit{tiers secteur}, ou bien de manière plus restrictive en se limitant aux structures à but strictement non lucratif. Certains veulent en faire une composante sociale du système capitaliste, quand d'autres y voient une économie alternative. Elle est alternativement perçue comme porteuse d'une véritable dynamique entrepreneuriale, ou au contraire comme une voiture balai pour les laissés pour compte d'un système qui n'a pas une place pour chacun. L'émergence du concept d'entreprise sociale a bousculé l'ESS, et les tenants du système dominant accueillent positivement une économie sociale reposant sur des mécanismes de marché plutôt que sur la solidarité et la redistribution. Finalement réduite par la loi de 2014 à un \cit{mode d'entreprendre}, faisant face à une baisse massive des financements publics, mais aussi des dons privés, il se pose la question du rôle que l'\ess tiendra face aux enjeux de demain. \\

L'incertitude sur l'\ess interroge son identité. Les auteurs relèvent le caractère hybride de cette économie qui associe plusieurs \citit{vocations} organisationnelles, plusieurs identités. La thèse se propose d'analyser l'\ess sous le spectre de quatre identités. L'identité utilitariste renvoie à la recherche de rentabilité et à l'adoption de modalités de gestion managériales qui la rapproche de l'économie classique. Parfois critiquée pour le tableau idéaliste qu'elle dépeint, l'identité normative témoigne de la place de l'éthique, des valeurs et d'une démarche solidaire et humaniste dans l'ESS. La dimension fonctionnelle s'intéresse à la fin plutôt qu'aux moyens, et renvoie à sa mission sociale ou environnementale, et à la recherche d'impact concret sur les publics bénéficiaires de l'action des \oess. Enfin, l'identité collective renvoie à la démarche historique de mise en commun et de mutualisation des outils de production. Elle traduit la volonté de \cit{faire ensemble}, plutôt que d'agir comme des agents économiques désencastrés de la sphère sociale. Le recours à ce cadre théorique offre l'avantage d'échapper à une vision simpliste, ignorant les importantes variations parmi des organisations au même statut juridique. \\
