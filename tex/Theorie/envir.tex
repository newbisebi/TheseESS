% !TEX root = /Admin/main.tex

\section*{Introduction du chapitre}

La protection de l'environnement est devenue aujourd'hui un enjeu majeur - et pressant - pour la société. On voit se produire des changements organisationnels, comme l'introduction des rapports RSE parallèlement aux rapports annuels ; sociaux (tel que le mouvement des étudiants pour le climat en 2019) ; ou politiques, avec les bons succès électoraux de partis écologistes lors d'élections récentes. La communauté scientifique s'inscrit dans ce mouvement, comme le traduit une forte croissance du nombre de publications sur l'innovation environnementale depuis 2009 \parencite{diaz-garcia2015eco-innovation:}. Ce chapitre a pour objectif de montrer comment la littérature appréhende la prise en compte de l'écologie dans les organisations. De manière un peu surprenante, les recherches s'intéressent d'abord à la divulgation environnementale (à partir des années 1980, selon \textcite{ali2017determinants}), avant de porter davantage d'attention à l'innovation. \\

La divulgation environnementale désigne la communication des entreprises sur leur impact environnemental et sur les actions entreprises pour le réduire. Il s'agit d'une communication essentiellement formelle, s'inscrivant dans le cadre des documents de référence ou prenant la forme de communiqués de presse. Dans une première section de ce chapitre, nous présentons une synthèse de l'état de l'art, en soulignant notamment ses déterminants et ses effets. L'information donnée n'est jamais neutre : elle a pour objectif de répondre aux attentes et aux interrogations de différentes parties prenantes, et de convaincre de la pertinence et de la validité de l'activité de l'entreprise. Elle vise aussi à démontrer la conformité aux réglementations en vigueur. Or, les attentes et les réglementations sont multiples et peuvent paraître contraires aux intérêts des organisations. L'utilisation du langage, de la rhétorique, devient un outil pour répondre à ces attentes. Nous faisons le lien avec le chapitre \ref{chapitre:ess} en soulignant le rôle de l'identité organisationnelle dans les stratégies rhétoriques. La notion de cadrage rhétorique est également présentée. La typologie de \textcite{nisbet2010framing, nisbet2009communicating}, qui applique le cadrage au débat environnemental, est utilisée comme outil d'analyse pour la partie empirique de la thèse (chapitre \ref{chapitre:twitter}). \\

L'étude de la communication conduit à se questionner sur les actions réelles des organisations. La deuxième section du chapitre propose d'y répondre à travers le concept d'innovation environnementale, ou éco-innovation (EI). Depuis quelques années, les chercheurs s'évertuent à montrer ce qui motive le développement d'innovations prenant en compte l'environnement. Les motivations sont de trois ordres \parencite{bansal2000why}. Elles répondent d'abord à une recherche de compétitivité, indispensable à la survie des entreprises. Souvent présentée comme une contrainte pour les entreprises, l'écologie peut pourtant contribuer à les rendre plus efficientes. La recherche de légitimité, de conformité et de satisfaction des parties prenantes est un second facteur influençant l'éco-innovation. Enfin, les valeurs portées par les membres des organisations, en particulier le top-management, influencent les stratégies d'innovation environnementale. La littérature souligne aussi la variété des innovations, qui ne sont pas seulement des nouvelles technologies, mais résident aussi dans les changements d'organisation, de comportements individuels, ou dans l'organisation de nouvelles pratiques au niveau d'une industrie ou d'un réseau d'acteurs. \\

Enfin, le chapitre se termine sur le lien entre l'action environnementale et les caractéristiques des \oess. La littérature est peu abondante sur ce point. L'étude de l'éco-innovation ou de la divulgation concerne généralement les entreprises classiques, quand l'étude de l'\ess s'intéresse plus généralement à la gouvernance ou à l'action sociale. Pourtant, des auteurs soulignent l'importance d'interroger la place des questions environnementales dans l'\ess \parencite{dart2010green, buchs2014role, edwards2013environmental}. Si en premier lieu, ce segment de l'économie semble mieux à même de répondre à l'urgence environnementale que le secteur privé classique, il fait néanmoins face à de nombreux obstacles.

\section{La communication environnementale}

    Dans cette section, nous nous intéressons à la façon dont les entreprises abordent les questions environnementales dans le discours et la communication institutionnelle. Dans un premier temps, nous nous consacrons à la divulgation environnementale, c'est-à-dire aux éléments communiqués par les entreprises sur leur action. Dans un second temps, nous nous penchons sur les stratégies rhétoriques qui peuvent être mobilisées dans la communication.

    \subsection{La divulgation environnementale des entreprises}

        L'intérêt croissant pour la \rse s'est accompagné d'une communication plus importante des entreprises sur leur impact environnemental. Face à la montée des attentes des parties prenantes et à la crainte de voir émerger des régulations plus contraignantes, de nombreuses entreprises publient pro-activement des informations environnementales sur leur activité \parencite{cormier2007revisited, cormier1999corporate}. De nombreuses recherches visent à identifier précisément les facteurs qui motivent la divulgation volontaire et à en mesurer les effets. Les auteurs étudient notamment le lien entre la divulgation et la performance économique (croissance de la part de marché, du chiffre d'affaire ou de la rentabilité), la réaction des consommateurs et des actionnaires, considérés comme les parties prenantes les plus importantes, et l'effet en termes de légitimité et d'impact médiatique. Les résultats se révèlent toutefois inconsistants et parfois contradictoires \parencite{gray2001social}. En outre, des effets opposés sont parfois observés d'un pays à l'autre \parencite{bouten2012how, cormier2007revisited}.

        \subsubsection{Déterminants de la divulgation}
            Un certain nombre de recherches étudient les déterminants du niveau de divulgation environnementale des entreprises. Elles cherchent à identifier les caractéristiques des entreprises qui expliquent la quantité et la qualité de l'information communiquée sur l'impact environnemental, généralement sous la forme de communiqués ou dans les rapports annuels. \textcite{bouten2012how} soulignent l'importance de distinguer le choix de divulguer ou non, et le niveau de divulgation. Ainsi les facteurs qui décident une entreprise de communiquer sur les pratiques environnementales et du volume d'information à divulguer diffèrent. \\

            Un facteur régulièrement relevé est la taille de l'entreprise \parencite{cormier1999corporate}, qui impacte la décision de divulguer \parencite{bouten2012how}. De nombreux facteurs financiers sont également identifiés comme déterminants de la divulgation : la situation financière de l'entreprise \parencite{cormier1999corporate}, le levier (soit le ratio dette sur capital), la dispersion du capital , le niveau de risque et le ROE \parencite{bouten2012how}. \textcite{da_silva_monteiro2010determinants} soulignent également que le fait d'être cotées en bourse impacte significativement le niveau de divulgation des entreprises portugaises. Les auteurs relèvent aussi l'impact du coût de l'information \parencite{cormier2004impact} et de l'exposition médiatique \parencite{bouten2012how}. Les entreprises agissant dans un domaine sensible du point de vue environnemental divulguent davantage d'informations que les autres \parencite{aerts2008corporate, radhouane2018customer-related}.


            % \todo[inline]{On pourrait citer aussi Cho  and  Patten,  2007;  Clarkson  et  al.,  2008;  Liu  and  Anbumozhi,  2009}


        \subsubsection{Effets de la divulgation environnementale}

            La littérature s'intéresse à l'impact de la divulgation environnementale sur la performance économique, la légitimité organisationnelle et sur  la performance environnementale. \\

            L'effet de la divulgation environnementale sur la performance présente un intérêt évident pour les partie prenantes, en particulier pour les actionnaires, et fait naturellement l'objet de nombreuses études. \textcite{cormier2007revisited} mesurent un effet du reporting environnemental sur le coût du capital mais les différents éléments divulgués n'ont pas le même effet. En outre, le contexte de l'entreprise influence également l'effet de la divulgation. Plusieurs études trouvent un effet positif de la divulgation sur la valeur de marché \parencite{radhouane2018customer-related, cormier2015does, plumlee2015voluntary, aerts2008corporate}. Cependant, les actionnaires prennent également en compte le coût supporté dans leur évaluation de la divulgation. Ce coût est perçu comme plus élevé dans les industries sensibles sur le plan environnemental, car la divulgation peut entraîner une réaction de parties prenantes, voire conduire à de nouvelles législations contraignantes \parencite{radhouane2018customer-related}. \\

            Bien qu'elle apparaisse plus abstraite, la littérature abondante sur la légitimité organisationnelle semble valider l'assertion de \textcite{suchman1995managing} qui la présente comme une \citit{ressource opérationnelle}, essentielle à la pérennité d'une entreprise.
            Le niveau de légitimité est impacté positivement par une divulgation environnementale réactive, en réponse à une situation particulière, mais non par une communication pro-active. Le volume et la qualité des éléments économiques de la communication environnementale sont également déterminants pour la légitimité \parencite{aerts2008corporate}. Pour \textcite{brammer2008factors}, les grandes entreprises, ainsi que celles qui sont directement concernées par des questions environnementales présentent une information de meilleure qualité.

            Alors que la littérature tend à séparer les incitations financières et les incitations en termes de légitimité sur la divulgation environnementale, \textcite{cormier2015economic} soulignent leur interdépendance. Ainsi, les décisions relatives à la divulgation  doivent être prises en considération de multiples parties prenantes, incluant aussi bien les partenaires financiers que les médias et le public. Pour les auteurs, les efforts de communication sur les actions environnementales sont nécessaires car leur absence aboutit à \citit{une perte de crédibilité ou de réputation} (p. 17). La divulgation bénéficie aux parties prenantes financières mais permet aussi d'établir un meilleur dialogue avec la société civile et les pouvoirs publics. \\

            Outre les effets sur la performance économique et la légitimité des entreprises, la divulgation aboutit-elle à une meilleure performance environnementale ? \textcite{vazquez2008corporate} trouvent une forte relation entre le discours organisationnel et la performance environnementale. Toutefois, la littérature souligne l'importe d'une véritable culture d'entreprise pour améliorer la performance \parencite{crane2017rhetoric}. \textcite{al-tuwaijri2003relations} constatent que les entreprises qui ont la meilleure performance environnementale divulguent davantage d'informations environnementales. Évidemment, la divulgation est ici une conséquence de la performance, puisque l'incitation est forte à communiquer sur des \citit{bonnes nouvelles}. Cependant, les auteurs identifient aussi un lien de cause à effet entre la divulgation passée et la performance environnementale actuelle. La communication passée établit un seuil minimum de performance environnementale que l'organisation doit maintenir, au risque de décevoir les parties prenantes dont les attentes s'alignent sur les résultats passés.


    \subsection{Rhétorique et cadrage du discours}

        Le langage utilisé tient un rôle essentiel dans la communication environnementale des entreprises. La recherche en management s'appuie sur la rhétorique pour mettre en lumière les mécanismes mobilisées par les organisations. L'étude de la rhétorique remonte à l'antiquité Grecque, et aux écrits d'Aristote \parencite{aristotle1991rhetorique}, qui servent encore de base à son étude \parencite[par exemple][]{waldron2016how, brennan2014rhetoric, green2004rhetorical}. Pour plusieurs auteurs, la rhétorique est étroitement liée avec l'identité organisationnelle \parencite{sillince2009multiple, waldron2016how, frandsen2011rhetoric}. Dans cette section, nous discutons des mécanismes rhétoriques utilisés pour présenter l'action environnementale, en lien avec l'identité organisationnelle.


        \subsubsection{Définition et rôle de la rhétorique}
           \textcite{brennan2014rhetoric} définissent la rhétorique comme \citit{les ressources de communication utilisée pour atteindre des objectifs donnés}. Elle a cependant une intention de persuasion qui la distingue du simple \cit{discours} \parencite{higgins2012ethos}.

            La rhétorique est essentielle dans les organisations. Elle sert non seulement à échanger de l'information, mais aussi à construire une réalité sociale et organisationnelle \parencite[][]{heracleous2001organizational}. C'est ainsi par le discours qu'est créée la valeur perçue des innovations, ce qui en facilite la diffusion \parencite{green2004rhetorical}. \textcite{sillince2009multiple} démontrent que la rhétorique contribue à la création d'un avantage compétitif. En s'appuyant sur l'identité organisationnelle, elle donne aux ressources un caractère rare, non imitable et non substituable. Pour \textcite{heath2011external}, le discours des organisations a pour but de \citit{co-créer la réalité avec les publics extérieurs nécessaire pour obtenir un alignement d'intérêt plutôt que de souffrir de frictions nocives}. C'est aussi par la rhétorique que les organisations peuvent mobiliser les parties prenantes et obtenir leur adhésion, voire leur soutien \parencite{waldron2016how, brennan2014rhetoric}. Elle permet aussi de mobiliser l'engagement public face à des enjeux sociétaux et politiques, comme le réchauffement climatique \parencite{nisbet2009communicating}.

            Health souligne la distinction entre la rhétorique interne, instrument du lien entre l'organisation et les individus qui la composent, et la rhétorique externe, qui établit un pont avec la société.

        \subsubsection{Les preuves rhétoriques d'Aristote}
            Selon le philosophe, les rhéteurs s'appuient sur le \citit{logos}, c'est-à-dire la raison (apparente) et la démonstration scientifique, le \citit{pathos}, qui joue sur les émotions et les sentiments, et l'\citit{éthos}, qui met en avant le caractère moral et la crédibilité \parencite{higgins2012ethos}. Le \textit{logos} s'appuie sur l'argumentation, la logique, les justifications et preuves, les données ainsi que des exemples historiques. Le \textit{pathos} prend la forme de métaphores et d'identification à l'aide de références culturelles (sport, richesse, loyauté, amitié...). Enfin l'\textit{ethos} permet de construire la crédibilité en s'appuyant sur la similarité, la déférence, l'expertise, l'auto-critique, la volonté de réussir et la consistence \parencite[][p.198]{higgins2012ethos}. Pour ces auteurs, les trois registres sont mobilisés dans la communication environnementale des entreprises, cependant l'un prédomine généralement sur les autres. En outre, ils sont utilisés dans la construction de l'identité organisationnelle. Dans les cas étudiés par les auteurs, l'\textit{ethos} appuie l'identité d'une \citit{credible persona} (p.205), le \textit{pathos} souligne l'identité d'un \cit{bienfaiteur paternaliste} (p.201) et le logos, appuyé par les deux autres, confère l'identité d'une \cit{bonne entreprise citoyenne} (p. 204). \\

            \textcite{waldron2016how} font quant à eux le lien avec le différentiel d'identité entre l'organisation à l'origine du discours et celle qui en est destinataire. Lorsque ce différentiel est faible, des arguments rationnels, factuels, sont préférés (\textit{logos}), car ils vont vraisemblablement être compris par un interlocuteur proche de l'émetteur. Mais si le différentiel est fort et que les raisons factuelles de l'organisation risquent de ne pas être entendues par le destinataire du discours, l'organisation a recourt à des arguments basés sur le registre dramatique ou moral (\textit{pathos}). \\

            Le discours ne doit pas s'appuyer sur un seul de ces leviers, au risque de ne retenir l'attention que d'une partie de l'audience. \textcite[][p.14]{nisbet2009communicating} met en évidence cet effet dans le débat environnemental :
            \begin{quotation}
                \citit{De nombreux scientifiques et défenseurs [du climat] s'attendaient à ce que cette attention accrue des médias favorise une meilleure compréhension de la nature technique du problème par le grand public [...].  La communication est donc définie comme un processus de transmission, c'est-à-dire que les faits scientifiques sont censés parler d'eux-mêmes, leur pertinence et leur importance politique étant interprétées par tous les publics de la même manière. Malheureusement, une couverture médiatique de qualité n'atteindra probablement qu'un petit public de citoyens déjà informés et engagés.}
            \end{quotation}

        \subsubsection{Cadrage du discours}
            \citit{Les cadres sont des scénarios d'interprétation qui déclenchent un train de pensée précis, expliquant pourquoi une question pourrait soulever un problème, ce qui pourrait en être responsable, et ce qui devrait être fait pour y remédier} \parencite[][p.15]{nisbet2009communicating}. Le cadrage consiste ainsi à inscrire le discours dans un contexte qui va permettre d'en créer le sens. Tout discours, quel qu'il soit, s'inscrit nécessairement dans un cadre donné \parencite{nisbet2009communicating}. Le discours peut mobiliser différents cadres en fonction de l'objectif à atteindre. Par exemple, \citit{Poser un problème tel que la pollution dans le langage du discours dominant est un moyen puissant de présenter un argument et d'influencer l'opinion, puisque le discours dominant n'a pas besoin d'explications étendues ou de légitimation, car il est familier, reconnaissable, et accepté par une variété d'audiences} \parencite[][p.598]{mcgregor2004sustainable}. Pour \textcite{feinberg2013moral}, la polarisation du débat environnemental relève en grande partie du caractère moral qui lui est donné. Une échappatoire à cette polarisation consiste à inscrire ce débat dans une cadre différent, moins clivant.

            \textcite{waldron2016how} indiquent que l'identité guide l'utilisation des cadres rhétoriques. Plus précisément, c'est le différentiel d'identité perçu avec les entreprises visées par le discours qui est important pour les auteurs. Les entrepreneurs sociaux qui perçoivent un faible différentiel d'identité avec leur interlocuteur adoptent une stratégie isomorphique, mobilisant les cadres de référence de celui-ci. Mais en cas de fort différentiel d'identité, les entrepreneurs sociaux s'appuient sur des cadres différents, afin de pousser l'interlocuteur à faire évoluer sa position. \\

            Les questions environnementales, qui font l'objet d'un débat important dans la société, et se révèlent fortement polarisantes \parencite{nisbet2009communicating}, n'échappent pas à ce besoin de cadrage. \textcite{gamson1989media} étudient les cadres mobilisés lors du développement de l'énergie nucléaire. Les promoteurs de cette industrie faisaient passer le message d'une innovation très positive : \citit{Atoms for peace. Your friend, the atom. Electricity too cheap to meter.} \parencite[][p.1]{gamson1989media}. Plus généralement, les auteurs mettent en lumière sept cadrages du discours : le progrès, l'indépendance énergétique, le \cit{marchandage avec le diable} (\textit{Devil's bargain}), le \cit{sauve-qui-peut} (\textit{runaway}), la responsabilité publique, la non-rentabilité et les chemins alternatifs (\textit{soft paths}). Plus récemment, \textcite{nisbet2009communicating, nisbet2010framing} étend cette typologie et l'applique au cas du réchauffement climatique. Il aboutit à neuf cadres fréquemment utilisés dans le débat scientifique et sociétal (tableau \ref{table:typonisbet}).

            \begin{table}[h]
                \caption{Typologie de cadres applicables à la question du climat}
                \label{table:typonisbet}
                \begin{tabularx}{\linewidth}{|K{0.30\textwidth}|X|}

                    \hline
                    \textbf{Cadre} & \textbf{Les cadres définissent le problème posé comme...} \\ \hline
                    \hline
                    Progrès Social
                    & l'amélioration de la qualité de vie ou la résolution de problèmes. Alternativement, comme une harmonie avec la nature plutôt que sa maîtrise, la durabilité
                    \\ \hline

                    Développement économique et compétitivité
                    & les investissements économiques, les avantages ou les risques du marché ; la compétitivité à l'échelle locale, nationale ou globale.
                    \\ \hline

                    Moralité et éthique
                    & en termes de bien ou de mal ; respecter ou franchir des limites, des seuils ou des limites
                    \\ \hline

                    Incertitude scientifique et technique
                    & une question de compréhension par les experts ; ce qui est connu par opposition à ce qui est inconnu ; invoque ou au contraire met en doute le consensus des experts et fait appel à l'autorité des sciences dures, à la falsifiabilité ou à l'examen par les pairs
                    \\ \hline

                    Boîte de Pandore, Monstre de Frankenstein et \cit{Sauve-qui-peut}
                    & un appel à la précaution face à d'éventuels impacts ou catastrophes. Hors de contrôle, un monstre de Frankenstein, ou un fatalisme, c'est-à-dire que l'action est futile, le chemin est choisi, pas de retour en arrière possible.
                    \\ \hline

                    Responsabilité publique et gouvernance
                    & une recherche dans l'intérêt public ou au service d'intérêts privés ; une question de propriété, de contrôle ou de brevetage de la recherche, ou une utilisation responsable ou un abus de la science dans le processus décisionnel ; la \cit{politisation}
                    \\ \hline

                    Alternatives et compromis
                    & autour de la recherche d'une position de compromis possible, ou d'une troisième voie entre des points de vue ou des options conflictuels ou polarisés.
                    \\ \hline

                    Conflits et stratégies
                    & comme un jeu entre élites ; qui est en avance ou en retard pour gagner le débat ; bataille de personnalités ; ou de groupes ; (généralement l'interprétation est dirigée par les journalistes)
                    \\ \hline
                \end{tabularx}
                \footnotesize{Source : \textcite{nisbet2010framing}}
            \end{table}


        \subsubsection{Le discours du \cit{juste milieu}}
            \textcite{higgins2012ethos} montrent comment la rhétorique est mobilisée par les entreprises pour mettre en place une stratégie de \citit{juste milieu} (\textit{middle ground}) qui consiste à l'imbrication des questions environnementales dans les aspects économiques. Ce rapprochement est fait sans égards pour les frictions qui en résultent et sans mention de l'incapacité du marché à répondre aux enjeux environnementaux \parencite{livesey2002discourse}. Ainsi, \textcite{laine2005meanings} estime que le langage de la divulgation environnementale vise à créer la vision d'une démarche \cit{gagnant-gagnant}, associant performance environnementale et progrès économique. Ceci permet aux entreprises de se maintenir dans le discours dominant, celui du libre-marché, tout en répondant aux injonctions des défenseurs de l'environnement. Le discours mobilise le cadrage de l'effet positif du développement économique sur les questions sociales et environnementales \parencite{livesey2002discourse}. Des thématiques devenues récurrentes et communément acceptées sont ainsi au service de ce discours, comme la notion de \citit{triple bottom line} \parencite{higgins2012ethos}. \textcite{milne2006creating} soulignent aussi la prépondérance de la métaphore du \cit{voyage} vers des pratiques durables. Elle permet de donner l'impression d'une démarche active et confère une vision optimiste de l'avenir, permettant d'obtenir l'approbation du public. Elle est aussi utilisée pour \citit{redéfinir le développement durable dans des termes qui ne menacent pas le 'business as usual'}. \\

            Cette recherche d'un juste milieu a en réalité pour objectif de maintenir les pratiques économiques habituelles et de réduire les pressions exercées par les partisans de l'écologie. Elle est à rapprocher du management des impressions (\citit{impression management}) \parencite{cho2010language, frandsen2011rhetoric, higgins2012ethos} qui désigne \citit{le processus par lequel les individus s'efforcent de contrôler les impressions que les autres forment sur eux} \parencite[][p.34]{leary1990impression}. En gestion des organisations, il est utilisé pour obtenir des bénéfices matériels et sociaux, et créer une identité désirable \parencite{merkl-davies2007discretionary}. D'après \textcite{cho2010language}, la divulgation environnementale constitue elle-même un outil de management des impressions.

            \transition

            Les stratégies des entreprises pour répondre aux attentes environnementales tout en maintenant leur pratique et leur rentabilité soulèvent des questions intéressantes dès lors que l'on s'intéresse à l'ESS. En effet, nous avons vu que dans sa définition cette économie donne la priorité aux conséquences sociales et environnementales de leur action. Cependant, la tendance entrepreneuriale du secteur peut aussi les conduire à adopter ces mêmes pratiques de communication et de construction des impressions.


\transition

    Dans cette section, nous avons discuté de la communication environnementale des entreprises. Au delà des écrits et des paroles, ce sont les actions concrètes qui peuvent avoir un impact sur l'environnement. Or, avec la progression importante des pratiques de \rse, il peut s'avérer difficile de déterminer si une entreprise agit réellement de manière éco-responsable ou si elle se contente de pratiquer un \citit{green-washing} \parencite{parguel2011how}. Ainsi, \textcite{cho2010language} vérifient bien que les entreprises les moins performantes du point de vue écologique divulguent l'information environnementale de manière plus optimiste et plus assurée que les plus performantes. En s'appuyant sur la théorie institutionnelle, des auteurs argumentent toutefois que les entreprises sont susceptibles de devenir demain ce qu'elles prétendent être aujourd'hui, et donc d'appliquer réellement les bonnes pratiques environnementales qu'elles mettent en avant dans la communication \parencite{frandsen2011rhetoric}.

    Dans la section suivante, nous laissons de côté les discours, pour nous intéresser plutôt aux pratiques, à travers le concept d'\ei.


\section{L'éco-innovation : une démarche éthique ou pragmatique ?}



    L'innovation est depuis longtemps un élément important de la littérature économique et managériale. Selon \textcite{schumpeter2006capitalism}, le capitalisme est par essence un système dynamique, évolutif. La capacité des entreprises à changer pour s’adapter à ces évolutions, à travers un processus de \citit{destruction créatrice}, est donc une clé de leur réussite. Pour certains auteurs, l’innovation est un moyen d’obtenir des avantages compétitifs leur permettant de faire durablement face à la concurrence (McDonald, 2007). Il n’est donc pas étonnant qu’elle soit au cœur de nombreuses recherches. Une définition communément retenue est celle de l’OCDE \parencite{oecd2005oslo} :
    \begin{quotation}
        \citit{The implementation of a new or significantly improved product (good or service), or process, a new marketing method, or a new organisational method in business practices, workplace organisation or external relations. }
    \end{quotation}

    L’innovation est distincte de l’invention, qui n’est que l’émergence d’une idée ou d’une technologie nouvelle. Celle-ci ne peut être qualifiée d’innovation que dès lors qu’elle est introduite sur un marché. \textcite{baregheh2009towards} comparent 60 définitions distinctes dans des revues en management, innovation, organisations, technologies ou marketing. Ces définitions insistent sur la nature de l’innovation (la nouveauté), ses typologies, ses objectifs, le contexte social dans lequel elle s’applique, les moyens nécessaires à sa mise en place et les étapes respectives. A partir de la synthèse de ces différentes approches, les auteurs proposent la définition suivante :
        \begin{quotation}
            \citit{Innovation is the multi-stage process whereby organizations transform ideas into new/improved products, service or processes, in order to advance, compete and differentiate themselves successfully in their marketplace.}
        \end{quotation}

    L’essence de l’innovation repose dans son caractère nouveau, dans le changement qu’elle induit par rapport à une situation précédente. Pour \textcite{johannessen2001innovation}, l’innovation peut être appréhendée par un \citit{dénominateur commun} : la nouveauté. Leurs résultats montrent que l’innovation est un concept unidimensionnel caractérisé uniquement par le degré de radicalité. On distingue donc les innovations incrémentales des innovations radicales. Les innovations incrémentales concernent des modifications continues et graduelles qui ajoutent de la valeur tout en maintenant le processus de production existant, alors que les innovations radicales créent une discontinuité et visent à remplacer l’existant \parencite{carrillo-hermosilla2010diversity}. Il faut toutefois souligner que la nouveauté, et donc l’innovation, sont des perceptions et dépendent du point de vue duquel on se place, ou de \citit{l’unité d’adoption pertinente} \parencite{johannessen2001innovation}. Selon l'\textcite{oecd2015innovation}, \citit{une innovation peut être nouvelle pour l’entreprise, pour le marché, ou pour le monde}. Ainsi, ce qui peut constituer une innovation importante pour l’organisation n’est pas nécessairement perçu comme tel par les consommateurs.  Il est donc important, lorsqu’on cherche à étudier l’innovation, de spécifier le cadre dans lequel on l’analyse. \\



    Le terme d’innovation évoque souvent l’émergence de nouvelles technologies. Pourtant, le concept englobe un champ beaucoup plus large. Déjà en 1942, dans son ouvrage de référence \cit{Capitalisme, Socialisme et démocratie}, \textcite[][p.83]{schumpeter2006capitalism} intègre les nouveaux produits, les nouvelles méthodes de production ou de transport, les nouveaux marchés, les nouvelles formes d’organisation industrielle créées par l’entreprise capitaliste (son propos portant spécifiquement sur ce système). L’OCDE distingue quatre principales catégories d’innovation : l’innovation produit, process, marketing ou organisationnelle (cf. encadré \ref{encadre:typoinnov}). \\

    \begin{encadre}
        \begin{tcolorbox}
        \caption{Typologie d'innovations}
        \label{encadre:typoinnov}
            \textbf{  Innovation produit} : l’introduction d’un bien ou service nouveau ou significativement amélioré dans ses caractéristiques ou son usage attendu. Ceci inclue les améliorations signataires des spécifications techniques, composants et matériaux, logiciels incorporés, interface utilisateur ou autres fonctions techniques. \\

            \textbf{ Innovation process} : l’implémentation d’une méthode de production ou de délivrance nouvelle ou significativement améliorée. Ceci inclue des changements significatifs en matière de techniques, d’équipements et/ou de logiciels. \\

            \textbf{ Innovation marketing} : l’implémentation d’une nouvelle méthode marketing impliquant un changement significatif dans le design ou le packaging du produit, le placement du produit ou la promotion et le prix du produit. \\

            \textbf{  Innovation organisationnelle} : l’implémentation d’une nouvelle méthode organisationnelle dans les pratiques commerciales de l’entreprise, l’organisation du travail ou les relations externes.
        \end{tcolorbox}
    \end{encadre}

    Le concept d'éco-innovation s'intègre à cette littérature en se focalisant sur la dimension environnementale de l'innovation. Il s'agit \citit{de toutes les mesures d'acteurs pertinents (entreprises, politiciens, syndicats, associations, Églises et foyers) qui : développent des idées, comportements, produits et processus nouveaux, les appliquent ou les introduisent ; et qui contribuent à réduire les charges pesant sur l'environnement ou à atteindre des objectifs de durabilité écologiquement précis.} \parencite{rennings2000redefining}. \textcite[][]{kemp2001survey}, repris par \textcite{horbach2008determinants}, donnent une définition similaire et précisent que les éco-innovations peuvent prendre la forme de \citit{techniques, systèmes et produits nouveaux ou améliorés afin de réduire les dommages environnementaux}. \\

    A l'heure actuelle, la définition la plus communément retenue de l'éco-innovation est sans doute celle proposée par le projet MEI (\textit{Measuring Eco-Innovation}) de la Commission Européenne : \textit{\cit{La production, assimilation ou exploitation d'un produit, processus de production, service ou méthode de management qui est nouvelle pour l'organisation et qui résulte, au cours de son cycle de vie, à une réduction du risque environnemental, de la pollution ou d'autres impacts négatifs liés à l'utilisation des ressources, relativement à d'autres alternatives.}} Cette définition retient plusieurs points importants. Tout d'abord, l'éco-innovation n'est pas limitée aux technologies innovantes, mais intègre aussi les changements d'organisation et de modes de production. On distingue ainsi plusieurs types d'éco-innovations. L'éco-innovation technologique, l'éco-innovation organisationnelle (qui peut se matérialiser par des certifications comme l'ISO 9001), l'éco-innovation sociale qui s'appuie sur des changements de comportements individuels et l'éco-innovation institutionnelle qui porte sur un changement plus systémique, induisant des effets sur de nombreuses entreprises voire sur l'ensemble d'une filière.  La définition souligne aussi que l'innovation est considérée au niveau micro-économique. L'implémentation d'une éco-innovation déjà mise en place ailleurs est bien considérée comme une éco-innovation, dès lors qu'elle a un impact environnemental moindre par rapport aux alternatives existantes \parencite{arundel2009measuring, kemp2010eco-innovation:}.


    \subsection{Déterminants}

        La littérature porte une attention particulière aux déterminants de l'éco-innovation. Plusieurs typologies émergent de la littérature, dont les catégories se recoupent comme les capacités technologiques, la demande du marché et l'influence des réglementations \parencite{rennings2000redefining, horbach2008determinants, triguero2013drivers}. Ces deux derniers facteurs intègrent un enjeu de légitimation \parencite{bansal2000why}. L'éco-innovation est également la conséquence de déterminants normatifs, éthiques \parencite{bansal2000why, mathieu2015les, reynaud2008les, paulraj2009environmental}, s'intégrant au cadre de la \rse. Les déterminants sont présentés ici selon le modèle de \textcite{bansal2000why}, basé sur trois catégories : la recherche de compétitivité, le besoin de légitimation et la responsabilité écologique. Les barrières rencontrées par les éco-innovateurs sont également évoquées.

        \subsubsection{La recherche de compétitivité}

            \textcite{porter1995toward} soutiennent qu'au lieu de s'opposer, écologie et développement économique peuvent aller de pair et, à travers l'innovation, devenir source d'avantages concurrentiels. Dans cette perspective, l'éco-innovation est source de \citit{valeur partagée} \parencite{porter2011creating}. La recherche de compétitivité et de réalisation de profits à long terme à travers l'innovation environnementale est le premier déterminant du modèle de \textcite{bansal2000why}. Cette démarche pragmatique d'optimisation économique repose sur des innovations à la fois sur les produits et sur les processus \parencite{triguero2013drivers}, dans l'optique de créer (et capter) une valeur supplémentaire et de réduire les coûts. Ceci repose sur le développement technologique, mais aussi sur les capacités internes \parencite[\textit{technology-push,}][]{rennings2000redefining}. Le gain de compétitivité passe à la fois par les produits et les processus. La dimension écologique joue un rôle marketing, permettant de développer de nouveaux produits, d'accéder à de nouveaux marchés et d'augmenter la valeur perçue par les consommateurs. Plus généralement, elle contribue à l'amélioration de l'image de l'organisation \parencite{triguero2013drivers, rennings2000redefining}. A cet effet, \textcite{pujari2006eco-innovation} souligne l'importance de la coordination, de l'intégration des fournisseurs et de l'équilibre entre progrès scientifique et orientation au marché pour atteindre un avantage compétitif à partir des éco-innovations produits. L'évolution régulière des technologies permet aussi de réduire le gaspillage, ce qui permet un usage plus optimal des ressources. Les auteurs soulignent notamment le gain possible en termes d'économies d'énergie \parencite{porter2011creating, dangelico2010mainstreaming}, qui conduisent à une réduction des coûts.

        \subsubsection{Le besoin de légitimation}

            Outre la compétitivité, \textcite{bansal2000why} observent que l'éco-innovation répond à un fort besoin de légitimation. L'idée de Milton Friedman selon laquelle l'entreprise est dépourvue de responsabilité sociale et doit opérer dans le seul intérêt de ses actionnaires \parencite{porter2011creating} est dépassée. Une certaine satisfaction des parties prenantes constitue un enjeu de survie pour les entreprises qui ont besoin d'être acceptées par la société. Elles s'assurent également d'éviter de causer des dommages écologiques qui constitueraient une mauvaise publicité et une perte de confiance du public.

            La légitimité passe aussi par la conformité avec les législations écologiques, afin d'éviter de possibles sanctions \parencite{bansal2000why}. Les réglementations constituent un déterminant important pour les innovations environnementales qui ne sont pas appliquées spontanément \parencite{rennings2000redefining}. La pression institutionnelle est donc nécessaire, en particulier pour les innovations concernant les processus. La littérature se réfère régulièrement à \cit{l'hypothèse de Porter}, qui soutient que la régulation est nécessaire pour pousser à la mise en place de politiques environnementales dans les organisations. Pour autant, ceci ne doit pas être perçu seulement comme une contrainte, puisqu'elle est, comme nous l'avons décrit, génératrice d'avantages concurrentiels. Des études empiriques soutiennent cette hypothèse. \textcite{al-tuwaijri2003relations} vérifient que les entreprises adoptant un \citit{bon management} et qui reconnaissent leur responsabilité sociale contribuent à créer conjointement de la valeur économique et environnementale.

        \subsubsection{La responsabilité écologique}
            L'approche de \textcite{bansal2000why} met l'accent sur des déterminants internes, liés aux valeurs et convictions des personnes dans les entreprises, et en particulier des dirigeants. Le facteur \cit{responsabilité écologique} désigne \citit{les valeurs et obligations sociales} (p.728) de l'entreprise, le sentiment d'une responsabilité vis à vis de l'environnement. Il est guidé par une démarche éthique et non utilitariste. Cette dimension est aussi marquée par un sentiment de satisfaction, voire de fierté de \cit{bien agir}. \\

            Une étude menée dans 22 pays montre en effet que la sensibilité sociale ou environnementale est liée aux valeurs de collectivisme et d'universalisme des individus \parencite{reynaud2008les}. \textcite{mathieu2015les} se penchent en détail sur les facteurs internes de l'éco-innovation et mettent en lumière que différentes approches stratégiques de l'éco-innovation peuvent être liées aux référentiels gestionnaires des organisations. Ceux-ci désignent \citit{les cadres référents, c'est-à-dire des ensembles cohérents de valeurs organisationnelles et de principes d’action auxquels les entreprises se réfèrent, pour justifier de leur positionnement vis-à-vis de la durabilité et interagir avec leur environnement} \parencite{martinet2004entreprise}. En particulier, les entreprises s'inscrivent dans un référentiel plutôt financier, ou plutôt durable. Le référentiel financier renvoie à l'analyse classique de l'entreprise, dans laquelle la rentabilité financière prime sur la prise en compte du contexte social et environnemental. A l'inverse, le référentiel durable suppose que l'entreprise agisse au regard de son environnement : l'activité économique n'est pas dissociable du développement durable. Les auteurs montrent que la prédominance de logiques durables conduit généralement à des stratégies plus proactives et à un plus grand nombre d'éco-innovations. Au contraire, les entreprises répondant à une approche financière adoptent plus souvent une stratégie adaptative, répondant à la demande des clients ou à la réglementation. \\


        \subsubsection{Barrières à l'éco-innovation}

            Les auteurs étudient également les facteurs faisant obstacle à l'innovation environnementale. \textcite{ashford1993understanding}, cité notamment par \textcite{arundel2009measuring}, les classe en sept catégories : (1) les barrières technologiques, qui incluent l'indisponibilité des technologies ou des ressources alternatives moins polluantes, mais aussi le scepticisme vis-à-vis des technologies existantes et le manque de flexibilité des processus existants ; (2) les barrières financières, liées aux coûts de R\&D, d'acquisition des technologies, ainsi qu'au coût du risque et à une recherche de rentabilité à court terme ; (3) les barrières liées à la force de travail, résultant du manque de compétences techniques ou managériales au sein de l'organisation, mais aussi aux résistances humaines face au changement ; (4) les barrières liées aux régulations, dues notamment à des réglementations imparfaites et contre-productives, mais aussi à l'incertitude quant à l'évolution des réglementations ; (5) les barrières liées aux consommateurs, qui peuvent mal recevoir les changements ou imposer un cahier des charges trop contraignant ; (6) les barrières liées aux fournisseurs, conséquence d'un manque d'implication et de support et ; (7) les barrières managériales provenant d'un manque d'implication des dirigeants, du manque de coopération, d'expertise, ou du refus du changement. \\

            Par ailleurs, plusieurs auteurs soulignent une particularité de l'éco-innovation, celle de la \cit{double externalité} \parencite{faucheux2011it, rennings2000redefining, ozusaglam2012environmental}. Selon \textcite{ozusaglam2012environmental}, elle conduit les entreprises à sous-investir dans l'éco-innovation. Cet effet provient du fait que l'\ei produit des externalités positives durant la phase de recherche et développement, comme toute autre innovation, mais également lors des phases d'adoption et de diffusion, à travers l'impact environnemental positif. En termes économiques, cet impact est un bien public qui bénéficie davantage à la société qu'à l'entreprise elle même. C'est cet effet qui justifie l'intervention des pouvoirs publics vis-à-vis de l'action environnementale des entreprises. Il est intéressant de questionner cette analyse du point de vue de l'ESS. En effet, \textcite{santos2012positive} soutient que les entreprises sociales (par extension, les \oess) sont des entreprises qui cherchent à créer de la valeur plutôt qu'à capter la valeur créée (à l'opposé, par exemple, de sociétés d'investissements dont le but est de capter de la valeur qu'elles n'ont pas créée elles-mêmes). Si l'\ess est vraiment une économie visant à contribuer au progrès social et environnemental, la double externalité devient alors souhaitable et positive, au lieu d'être un obstacle. Cependant, comme mis en évidence dans le chapitre précédent, les \oess défendent aussi leurs intérêts individuels.

    \subsection{Différents types d'innovations}
        La littérature distingue les éco-innovations de type technologiques, organisationnelles, sociales et institutionnelles \parencite{rennings2000redefining}.

        Les éco-innovations technologiques concernent les changements opérés sur les produits eux mêmes, afin de les rendre plus écologiques, ainsi que sur les processus de production. Le développement de nouvelles technologies permet ainsi de limiter les déchets et les consommations d'énergie. Le rapport MEI\footnote{\textit{Measuring eco-Innovation}} de la commission européenne intègre à cette catégorie le contrôle de la pollution et notamment les rejets dans l'eau ou dans l'environnement, les processus \citit{plus propres}, les équipements de management des déchets, les instruments de suivi de l'action environnementale, les technologies vertes, la gestion des ressources en eau et le contrôle du bruit et des vibrations \parencite[][p.10]{kemp2007final}.

        L'\ei organisationnelle désigne les changements dans les modes opératoires et \citit{les instruments de management au niveau de l'entreprise tels que les audits environnementaux}. \parencite{rennings2000redefining}. Plus précisément, il s'agit de rendre plus efficients les processus de production, de mettre en place des systèmes de management et d'audit s'appuyant sur des mesures et des reportings, ainsi que des certifications (par exemple ISO 14001), et enfin la coopération entre les entreprises pour optimiser la gestion des déchets tout au long de la chaîne de valeur (\citit{du berceau à la tombe}) \parencite[][p.10]{kemp2007final}.

        L'\ei sociale regroupe les changements dans les comportements individuels, notamment des consommateurs, afin qu'ils adoptent des modes de vie plus respectueux de l'environnement \parencite{rennings2000redefining}.

        Pour finir, l'\ei dépasse parfois le seul cadre des innovations instaurées au sein des entreprises et s'appuie sur un ensemble d'acteurs. On parle alors d'\ei institutionnelles. Elles peuvent s'appuyer par exemple sur \citit{un réseau d'ONG, de scientifiques, d'entreprises et d'autorités publiques} \parencite[][p.324]{rennings2000redefining}. Ils proposent des changements à un niveau plus macro-économique, pour rechercher un impact environnemental plus important. \\

        Les différentes formes d'éco-innovations ne doivent pas être complètement distinctes et avec des frontières hermétiques. Au contraire, l'éco-innovation passe généralement par l'imbrication de ces différentes catégories \parencite{rennings2000redefining}.


\transition
    Dans l'approche classique de l'entreprise, la poursuite d'objectifs environnementaux n'est pas pertinente, puisque sa mission est de servir les intérêts des actionnaires et non de la société. Ainsi, l'éco-innovation devrait être impulsée par les pouvoirs publics à travers des régulations. Cependant, \textcite{porter1995toward} soutiennent que l'action environnementale n'est pas qu'une contrainte, mais qu'elle est en réalité vertueuse pour les entreprises. C'est pourquoi la littérature cherche à établir le lien entre l'éco-innovation et la performance économique. La question se pose différemment dans le cas de l'économie sociale : qu'elle soit envisagée dans la perspective d'un \cit{capitalisme social} \parencite{yunus2010building}, une \cit{économie sociale et écologique} \parencite{waridel2016economie} ou comme une \cit{économie transformatrice}, elle doit servir une autre mission que la recherche de profit économique. L'action environnementale, comme la communication qui l'accompagne, peut donc constituer une finalité ou s'intégrer aux valeurs de l'entreprise. Cependant, les entreprises de l'\ess sont aussi intéressées par la performance économique ou le gain de légitimité. Dans la section suivante, nous présentons un volet de la littérature qui cherche à comprendre si l'\ess est intrinsèquement liée au volet environnemental du développement durable, ou si les motivations de la communication et de l'action environnementales sont en réalité similaires à celles des entreprises du secteur lucratif. \\

    \begin{quotation}
        \small
        \textit{Note :} Par la suite, nous distinguons \cit{l'innovation environnementale} de \cit{l'action environnementale}. Comme le montre l'étude de la littérature, l'action environnementale peut être considérée comme innovante, même lorsqu'elle prend la forme de petites évolutions successives (innovation incrémentale), ou lorsqu'il s'agit d'imiter des pratiques déjà anciennes sur le marché. Cependant, le maintien de \cit{bonnes pratiques} environnementales ne constitue pas en soi une innovation, puisqu'il n'apporte aucun élément nouveau, même quand il nécessite une démarche pro-active. Ce type d'actions est pertinent pour notre étude, c'est pourquoi nous gardons une perspective large en parlant d'action environnementale plutôt que d'éco-innovation. Nous postulons  que la grille d'analyse des motivations à l'éco-innovation peut s'appliquer plus largement à l'action environnementale.
    \end{quotation}


\section{L'ESS et le développement durable}

    Né à la fin des années 1980, le \dd \citit{imagine  la possibilité  d’un  développement  rendant compatible croissance économique, protection de l’environnement et prise en compte des exigences sociales} \parencite[][p.118]{reynaud2004developpement}. Le \dd est plus généralement défini comme une approche permettant de répondre aux besoins présents de la société, sans compromettre ceux des générations futures. C’est donc un mode de développement qui refuse de reporter à demain les coûts économiques et écologiques de la vie humaine. Il s’organise autour de trois volets : économique, social et environnemental. Comme l’ESS, le développement durable a pour objectif de mettre en place un mode de fonctionnement différent de la société.

    Dans cette partie, nous discutons du lien entre \ess et \dd, souvent présenté comme \cit{évident} \parencite{cretieneau2010economie} mais qui fait pourtant l’objet de peu d’investigations \parencite{dart2010green,edwards2013environmental} .

    \subsection{Convergence entre ESS et développement durable}

        \subsubsection{Des finalités comparables}

            Comme le soulignent \textcite{gendron2011developpement}, le \dd et l’\ess sont deux concepts distincts. Cependant, leurs finalités se recoupent. Les deux rejettent un mode de développement économique reposant sur la destruction de l’environnement et créateur d’injustices sociales. Ils visent à promouvoir des modèles alternatifs au service de l’intérêt général et adoptent des principes similaires \parencite{cretieneau2010economie,gendron2011developpement}. \\

            \ess et \dd  s’inscrivent dans une perspective de long terme. Un principe essentiel de l’\ess est de maintenir la continuité de l’activité au service de ses membres plutôt que de rechercher un enrichissement rapide mais potentiellement néfaste pour l’entreprise. Ce principe est garanti par la limite dans la redistribution des excédents et dans le caractère impartageable des réserves, qui rendent impossibles des pratiques spéculatives observées dans le secteur privé classique. L’enjeu est davantage le maintien de l’activité et sa transmission que la prise de bénéfice par une génération donnée. Le développement durable s’inscrit dans une démarche comparable en assurant aux générations futures qu’elles n’auront pas à supporter le coût des décisions prises des années auparavant, comme une dette laissée à sa descendance. Sur le plan économique, ce mécanisme se retrouve dans le système des dettes publiques, qui s’accumulent et se transmettent. Sur le volet écologique, le réchauffement climatique et l’épuisement de certaines ressources auquel le monde va devoir faire face sont le résultat des fortes croissances obtenues au détriment de l’environnement. \\

            Les promoteurs du \dd et les \aess défendent l’idée qu’il est possible de répondre aux besoins de tous, tout en respectant l’homme et l’environnement. \textcite{cretieneau2010economie} souligne ainsi que l’ESS, au contraire de l’économie capitaliste, n’est pas \cit{désencastrée}, c'est-à-dire qu’elle agit en lien avec la société, au lieu d’œuvrer comme si elle était indépendante de tout contexte social. La finalité de l’ESS, comme celle du développement durable, n’est pas l’enrichissement de certains, mais la satisfaction des besoins du plus grand nombre et l’atteinte d’objectifs sociétaux \parencite{gendron2011developpement}. \\

            Ainsi, l’ESS, dans sa finalité, dans sa mission de promotion de l’intérêt général, se positionne comme un segment de l’économie particulièrement pertinent pour mettre en place un développement durable.

        \subsubsection{Des caractéristiques organisationnelles favorisant la prise en compte du développement durable}

            Au-delà de ses finalités, la structure des entreprises de l’\ess peut jouer en faveur de l’intégration des principes du développement durable, notamment sur le volet environnemental. S’intéressant particulièrement aux structures coopératives, \textcite{bocquet2010economie} montrent que la non-lucrativité ainsi que le fonctionnement partenarial de l’\ess favorisent le développement de projets en faveur de l’environnement. La conscience d’appartenir à l’\ess est également un facteur qui les pousse à agir de façon éthique. Tout d’abord, détachées de la seule logique de recherche de profits, les coopératives peuvent entreprendre des projets plus responsables sur le plan écologique, même si les retombées économiques ne sont pas à la hauteur des ressources investies. Contrairement à des entreprises cotées ou financées par des investisseurs externes, elles n’ont pas à craindre une baisse de la valeur de l’entreprise à court terme. Il faut toutefois que les projets entrepris garantissent a minima une rentabilité permettant de couvrir les coûts engagés. Ensuite, la dimension collective des coopératives, et notamment l’importance des partenariats, les conduit à prendre en compte des intérêts plus larges, plus divers, et non seulement ceux des investisseurs. La variété des parties prenantes engagées dans la gouvernance conduit à mieux prendre en compte la dimension écologique. Enfin, l’appartenance à l’\ess conduit les entreprises à s’interroger sur leurs valeurs et à s’efforcer d’en appliquer les principes.


        \subsubsection{Une démarche environnementale affirmée }

            Dans un document publié en 2015, l’Atelier Ile de France, association de soutien aux porteurs de projets de l’ESS, présente cette économie comme une force motrice de la transition écologique. Il rappelle la concordance entre les valeurs de démocratie, citoyenneté et solidarité de l’\ess et les principes du développement durable \parencite{latelier_ile_de_france2015economie}. Les entreprises de l’\ess sont présentées comme des pionnières dans le développement de nouveaux modes de consommation, comme l’agriculture biologique et les circuits courts, la réduction et la valorisation des déchets ou encore la mise en œuvre de modes de vie moins consommateurs d’énergie. \textcite{waridel2016economie} souligne ainsi que de nombreuses entreprises visant à avoir un impact environnemental positif choisissent l’ESS. Selon \textcite{cretieneau2010economie}, l’\ess n’a pas attendu l’émergence du concept de développement durable pour mettre en œuvre des idées allant dans le sens d’un développement économique différent, ayant pour but de mieux répondre aux besoins des sociétés. Toutefois, pour l’auteure, c’est davantage le \dd qui constitue une opportunité pour l’\ess que l’inverse. Cette approche récente du développement économique est en cohérence avec le projet de l’ESS, à condition qu’elle s’en saisisse pleinement. Le \dd ouvre en effet à des nouveaux modes de consommation et de production, face auxquels l’\ess est particulièrement bien placée pour agir (systèmes d’échanges locaux, agriculture paysanne…). En outre, l’\ess est souvent l’outil des politiques publiques menées par les États. Elle bénéficie donc déjà d’un soutien pour intervenir dans les domaines privilégiés du \dd. Le \dd nécessite aussi une forme d’éducation populaire, qui ne peut s’exercer qu’au plus près des populations : or l’\ess est fortement implantée dans les territoires.

    \subsection{Limites à l’application du développement durable dans l’ESS}

        Malgré l’intérêt pour l’\ess de s’investir dans le développement durable, \textcite{bocquet2010economie} rappellent que la prise en compte de l’environnement répond à une démarche volontaire des entreprises. Selon \textcite{dart2010green}, le secteur non lucratif est présent dans des secteurs d’activité dont l’impact environnemental peut être important. Pourtant, cet impact est rarement étudié. \textcite{edwards2013environmental} constatent que les \eess mettent moins souvent en place des systèmes de management environnementaux que les entreprises du secteur privé classique. Ainsi, alors que le tiers secteur a de plus en plus recours à des audits de leur performance sociale, le volet écologique est fréquemment oublié.

        \subsubsection{La non-conscience de la responsabilité environnementale}

            Certains auteurs ont souligné une limite importante à la prise en compte des aspects environnementaux dans l’activité des \oess : nombre d’entre elles n’ont pas conscience de leur responsabilité en la matière, et du rôle qu’elles peuvent jouer. Pour Favreau (cité par Waridel, 2016), l’écologie n’est \cit{pas dans l’ADN de l’\ess}. \textcite{dart2010green} font le même constat : les dirigeants des \oess considèrent souvent que leur rôle est de se concentrer sur leur mission sociale.

        \subsubsection{La faiblesse des attentes vis-à-vis de l’ESS}

            Les auteurs estiment que les attentes en matière de responsabilité environnementale sont moins fortes vis-à-vis de l’\ess que du secteur capitaliste \parencite{dart2010green,edwards2013environmental}. En effet, la responsabilité et l’éthique professionnelle sont, d’une certaine manière, tenues pour acquises dans une économie qui se définit par ses valeurs. \textcite{dart2010green} affirment que l’action environnementale dans l’économie classique a souvent pour but de contrebalancer les effets négatifs de l’activité des entreprises dans la perception du public. Or, les entreprises de l’\ess ne sont pas (ou moins) confrontées à de telles critiques, leur action étant déjà perçue comme plus responsable, et sont donc moins poussées à se soucier de leur image. La forte implantation de l’\ess dans le secteur des services, dont l’impact écologique est perçu comme plus faible vient renforcer cet effet. Ainsi, alors que la responsabilité environnementale constitue un véritable enjeu de légitimité pour certaines entreprises, cela est moins le cas dans l’ESS. \\

            Outre que les pressions externes vis-à-vis de l’\ess sont moindres, les organismes fédérateurs de cette économie ont également tardé à porter eux-mêmes l’agenda environnemental et à le promouvoir auprès de leurs membres \parencite{edwards2013environmental}. Ainsi, l’environnement est souvent absent des publications diffusées auprès des entreprises de l’ESS. Or, des incitations venant de ces entreprises de référence pourrait inciter les entreprises à mettre en place des systèmes pour prendre en compte leur impact environnemental.


        \subsubsection{Le manque de moyens et d’outils}
            L’étude de terrain menée par \textcite{edwards2013environmental} met en évidence un manque évident de moyens pour mettre en place des processus de gestion environnementale, y compris parmi les entreprises les plus sensibilisées. Ce manque se traduit à la fois dans la disponibilité d’outils de mesure où de management et dans le coût de leur mise en place. Tout d’abord, les auteurs soulignent la diversité des outils de management de la performance environnementale \parencite{edwards2010mainstreaming}. Ils identifient cinq outils spécifiquement dédiés à l’environnement et huit comprenant au moins une partie consacrée à l’environnement. Cependant, ces outils ne sont pas nécessairement adaptés aux besoins de l’ESS et sont souvent trop coûteux.

    \subsection{Synthèse}

        Souvent perçue comme plus éthique, plus responsable, l’\ess  fait face à des incitations ambivalentes concernant la prise en compte de l’environnement. D’un côté, certaines de ses caractéristiques semblent faciliter le développement d’innovations environnementales. De l’autre, et de façon assez paradoxale, elles font face à moins de pressions pour être écologiquement responsables que les entreprises classiques.

        Le tableau \ref{table:syntheseessenvir} synthétise les différents facteurs (positifs ou négatifs) identifiés dans la littérature qui impactent la prise en compte de la dimension environnementale par les entreprises de l’ESS.

        \begin{footnotesize}
         \begin{landscape}
         \begin{longtable}{
             |>{\setlength{\baselineskip}{0.75\baselineskip}}K{0.07\linewidth}
             |>{\setlength{\baselineskip}{0.75\baselineskip}}K{0.1\linewidth}
             |>{\setlength{\baselineskip}{0.75\baselineskip}}K{0.37\linewidth}
             |>{\setlength{\baselineskip}{0.75\baselineskip}}K{0.37\linewidth}
             |}

             \caption{Facteurs influençant la prise en compte de l'environnement dans l'ESS}
             \label{table:syntheseessenvir} \\ \hline
              \textbf{Source}	& \textbf{Organisations considérées}	& \textbf{Facteurs positifs}	& \textbf{Facteurs négatifs} \\ \hline

              \endfirsthead         \hline
              \textbf{Source}	& \textbf{Organisations considérées}	& \textbf{Facteurs positifs}	& \textbf{Facteurs négatifs} \\ \hline
              \endhead

                \textcite{handy2001advocacy}
               & Environmental nonprofit organisations
               & Nonprofits are the best suited orga. for envir. advocacy
               &
              \\ \hline

              \textcite{gendron2002economie}
               & Économie Sociale et mouvements écologistes
               & - \og Ecologistes et acteurs de l’économie sociale peuvent se rejoindre pour contester le modèle dominant et proposer un modèle alternatif. \fg{} \newline - Mouvements rapprochés par le lien entre l’environnement et la pauvreté. \newline - L’\ess fournit un terrain d’action aux mouvements écologistes qui apportent plutôt les idées
               & Tous les mouvements écologistes n’adoptent pas une perspective d’économie sociale et tous les mouvements d’économie sociale n’adoptent pas une perspective écologiste.
              \\ \hline

              \textcite{bocquet2010economie}
              & \multirow{3}{=}{Coopératives et mutuelles}
              & - Statuts font que l’entreprise peut supporter une baisse de rentabilité à court terme liée contrairement à une entreprise classique qui ne peut se permettre une baisse du CA vis-à-vis de ses actionnaires
              \newline - Logique partenariale qui permet d’obtenir des avantages concurrentiels
              \newline - ancrage local et territorial
              &Prise en compte prioritaire des intérêts des PP impliquées dans la gouvernance ; l’action environnementale relève d’une démarche volontaire
              \\ \hline

              \textcite{dart2010green}
              &Nonprofits
              &- Il existe un bénéfice concurrentiel à être \og plus verte \fg{}\newline - Rôle des fondateurs pour insuffler une démarche écologique ?
              &- Non conscience des dirigeants et PP de l’impact environnemental
              \newline- Environnement perçu comme en dehors du champ d’action de l’entreprise : important focus sur le volet social qui masque la dimension environnementale
              \newline- Non tournées vers la performance, les organisations ne perçoivent pas le bénéfice potentiel de l’innovation environnementale ; le bénéfice en termes de réduction des coûts nécessite un investissement initial trop élevé
              \newline- Attentes sociales en termes de performances environnementales plus faibles que pour les entreprises classiques
              \newline- Evidence perçue de l’éthique dans les organisations conduit à ne pas mener une réflexion sur cet aspect
              \\ \hline

             \textcite{cretieneau2010economie}
             &ESS
             &- L’\ess n’est pas désencastrée contrairement à l’économie capitaliste qui agit sans lien avec son environnement \newline
             - Les acteurs de l’\ess s’inscrivent majoritairement dans une logique de développement durable\newline
             - Principes de l’\ess cohérents avec le DD
             &
             \\ \hline

             \textcite{gendron2011developpement}
             &Économie sociale (Quebec)
             &- \dd et \ess og se rejoignent à plusieurs égards qui peuvent se résumer par la reconnaissance d'une dimension sociale, le souci de l'intérêt général et l'idée d'un développement \og autrement \fg{} porteur d'objectifs sociétaux. \fg{}
              \newline(1) Ils s'appuient sur des principes semblables c'est-à-dire l'autonomie, un développement centré sur la satisfaction des besoins, la résilience et la démocratie \newline(2) Ils suggèrent des modes alternatifs de satisfaction des besoins sociaux \newline (3) Ils interrogent en profondeur la définition du bien commun, du bien-être social collectif, et plus largement la question de l'intérêt général.
              \newline \ \newline
              \underline{On distingue 4 articulations :}
               \begin{itemize}
                 \item	l’environnement et le \dd  comme révélateurs de la dimension socialement construite de l’économie,
                 \item	L’ES et le \dd  partageant l’interface du social,
                 \item	L’ES comme opérationnalisation du \dd,
                 \item	L’ES et le \dd  comme contributeurs mutuels.
               \end{itemize}

             & -La prise en compte des dimensions sociales ne signifie pas nécessairement que l'on fait du développement durable.
             \newline- Parallèlement, toutes les entreprises d'économie sociale n'ont pas nécessairement des comportements qui favorisent l’environnement.
             \newline- La réalisation des objectifs du développement durable déborde ceux des groupes d’économie sociale parce qu'ils ne visent pas à faire du développement durable, mais à mettre leur énergie de travail ensemble pour se doter de services et répondre à des besoins qui ne pourraient pas être satisfaits autrement.
             \\ \hline

             \textcite{favreau2011planete}
             & ESS
             & In the agricultural sector, cooperatives play an important role in the implementation of a CSR approach by their members.
            =>	Les coopératives jouent un rôle de diffusion : elles font intervenir des experts, organisent des formations ; si ressources dispo elles peuvent recruter des spécialistes pour promouvoir la RSE auprès de leurs membres

             &
             \\ \hline

              \textcite{taddei2012role}
             & Coopératives agricoles
             & - L’\ess met en place un \og new deal écologique et social à l’échelle de la planète. \fg{} Elle s’est déjà engagée dans la \og bataille \fg{} pour l’écologie.
             \newline - L’\ess permet de démocratiser l’économie, ce qui est nécessaire à la transition écologique
             & Facteurs politiques : l’\ess est mise en cause par le système capitaliste qui la contraint à adopter des principes de marché, au risque de délaisser ses valeurs et son rôle politique originel.
             \\ \hline

             \textcite{edwards2013environmental}
             &UK third sector organisations
             & “There is often a tacit assumption that the social purpose of most TSOs will ensure that the environment is considered” (Pearce, 2003, page 33).
             &- Le tiers secteur est composé en grande partie de PME, qui mettent généralement en place moins de Systèmes de Management Environnemental que les grandes entreprises.
             \newline- Les acteurs clés manquent de connaissance sur la façon de mettre en place des systèmes de management environnemental et ne savent pas où trouver cette information.
             \newline- Les acteurs sont limités par les moyens
             \newline- Manque de pressions institutionnelles
             \newline- Manque d’outils adaptés : il y a des doutes sur l’efficacité d’outils tels que le SROI en matière environnementale
             \\ \hline

             \textcite{mojo2015social}
             &Coopératives
             & Therefore, according to the definition, principles, and values, it is reasonable to align cooperatives as a right organizational form for sustainable development.

            Collective action is therefore believed to improve people’s engagement with environmental protection and natural resource management activities that need collective efforts.

             &Cooperative membership negatively associated with envir. perf : cooperatives are successful in diffusing practice, but they promote intensive agriculture
             \\ \hline

             \multirow{2}{=}{ \textcite{waridel2016economie}}
             & ESS
             & - Conscience des acteurs de terrain de la nécessité d’être exemplaires sur le plan social ET environnemental
             \newline- La majorité des organisations à but environnemental font partie de l’ESS, même sans mettre avant cette appartenance
             & L’\ess est plutôt caractérisée par sa capacité à associer des objectifs sociaux et économiques que par sa dimension écologique : l’écologie n’est pas dans l’ADN de l’\ess (Favreau).
             \newline
             Freins à l’atteinte d’objectifs environnementaux : \begin{itemize}
             \item	Les coûts plus élevés associés à la responsabilité environnementale
             \item	Le manque d’information
             \item	Le manque de réglementation
             \item	Le manque de volonté à l’interne
             \item	L’absence de choix plus écologiques
             \end{itemize}
             \\ \hline

            \textcite{draperi2018activites, draperi2018quand}
             & ESS
             & La prise en compte de l’environnement devrait devenir « une condition de la réalisation du projet coopératif »
             &Temporalité différente : la coopération est plus ancienne que le DD : ses principes ont été fixés alors que la question environnementale ne se posait pas.
            \\ \hline

            \textcite{musson2018les}
             & Coopératives (cas des maraîchers)
             & Les coopératives renforcent la confiance, variable décisive dans l’adoption d’innovations notamment environnementales.
             & Si le concept de développement durable est globalement maîtrisé, l’environnement n’est pas spontanément abordé dans les propos
            \\ \hline


         \end{longtable}
        \end{landscape}
        \end{footnotesize}
