\section*{Introduction du chapitre}

    \citit{Méthodologie sans épistémologie n’est que ruine de réflexion}. Par cet avertissement, \textcite{avenier2011mixer} nous invitent à inscrire notre recherche dans une réflexion plus large sur le sens que l'on donne à la connaissance scientifique que l'on veut produire. Celle-ci est rendue nécessaire par le constat que les sciences de gestion, bien que récentes, ont atteint une certaine maturité. Toutefois, il demeure une certaine confusion entre les questions de méthode, de méthodologie et d'épistémologie \parencite{martinet2013epistemologie}. Dans la première section, nous expliquons que le positionnement adopté est celui du réalisme scientifique, qui fait partie du courant post-positiviste. Nous considérons qu'une connaissance du réel est possible, mais qu'elle n'est pas sans faille. La démarche scientifique s'appuie donc sur des énoncés vérifiables ou infirmables.  \\
    
    \citit{Science sans conscience n'est que ruine de l'âme}. Plus connu, cet adage de Rabelais nous appelle à faire preuve de sagesse, lorsque l'on se veut scientifique. Il nous semble en effet nécessaire de prendre un certain recul sur la recherche et de nous interroger sur ses implications. Bien que la thèse ne pose pas a priori de difficulté majeure dans sa problématique, ni dans sa méthodologie, il ne nous semble pas inutile de nous poser la question de son positionnement éthique. C'est l'objet de la seconde partie de ce chapitre. Elle insiste notamment sur la question des données, de leur collecte et de leur usage, qui devient une question majeure, alors qu'elles sont parfois qualifiées de \cit{nouvel or noir}. Précisément, la méthode présentée dans le chapitre \ref{chapitre:methodes} s'appuie sur la collecte automatisée d'une grande quantité de données et leur traitement statistique. \\
    
    En donnant un cadre épistémologique et éthique à la recherche, ce chapitre a pour objectif de permettre au lecteur de mieux appréhender le contexte dans lequel s'inscrit cette thèse.

\section{Epistémologie}

    
    Comme toute recherche scientifique, ce travail a pour objectif la production de connaissances. Cette partie clarifie le positionnement de la thèse vis-à-vis de la nature des connaissances et de ce qui fait leur scientificité. \\
    
    Au niveau ontologique, la thèse s’inscrit pleinement dans une perspective réaliste. Il est admis qu’il existe un réel qui est indépendant de l’esprit humain et de l’attention qui lui est portée. Le réel est unique : les objets observés, les expériences vécues ou les émotions ressenties font partie d’une même réalité. Naturellement, nos sens peuvent nous tromper et nous conduire à tirer des conclusions fausses sur le réel. Pour autant, celui-ci n’est pas remis en question. Le positivisme logique limite le réel à l’observable, considérant que tout ce qui va au-delà relève de la métaphysique et conduit inévitablement à des écueils. Il exclut par conséquent le non-observable du champ de la recherche scientifique Nous considérons au contraire que le réel est composé d’entités observables (tel que les objets physiques) ou non observables (tel que les désirs ou les relations humaines). Le réel étant unique, ces entités bénéficient du même statut ontologique et font partie d’une seule et même réalité. Selon \textcite{hunt1992for}, \citit{Most of the entities postulated in the physical and biological theories are, at least in principle, 'tangible,' whereas many, but not all, of the entities postulated by theories in marketing and the social sciences are 'intangible' or 'unobservable in principle}. Ceci est applicable à l’étude de la stratégie des organisations, qui porte essentiellement sur un réel non-observable, intangible. Les organisations elles-mêmes ne sont pas du domaine de l’observable et on ne peut en voir que les femmes et hommes qui les constituent ou les locaux qu’ils occupent. Les concepts mobilisés pour cette étude sont également de l’ordre du non-observable. Pour autant, nous les traitons comme réels (bien qu’abstraits).\\
    
    Quelle connaissance peut-on avoir de tels objets ou concepts ? Quelle certitude peut-on avoir sur quelque chose qui échappe à nos sens, quelque chose d’aussi abstrait qu’une émotion ? La posture épistémologique adoptée repose sur deux hypothèses : (1) cette connaissance est possible, (2) cette connaissance n’est pas infaillible. \\
    
    Tout d’abord, il est admis que le réel est unique et que le chercheur fait partie intégrante de ce réel. La science porte donc sur le réel, et non sur \cit{des manifestations} ou sur \cit{un vécu du réel}. Elle vise à développer des connaissances correspondant à la réalité du monde. On cherche à connaître le réel \cit{tel qu’il est vraiment} (\citit{real-as-is}, Avenier \& Thomas, 2015, p.71). Cette hypothèse est corroborée par les succès passés de la science : \citit{the long term success of a scientific theory gives reason to believe that something like the entities and structure postulated by the theory actually exists} (Hunt, 1992, p.95). Toutefois, la nature complexe de la réalité et la nécessité de l’étudier à travers des instruments de mesure conduisent à reconnaître la faillibilité des connaissances. Celles-ci sont en outre influencées par nos sens et nos perceptions \parencite{hunt1992for}. Nous ne pouvons prétendre que la science nous donne une connaissance parfaite, exacte du monde. Les réalistes considèrent qu’un tel savoir est de l’ordre théologique et non scientifique, comme si le monde était perçu \cit{à travers l’œil de Dieu}. Toutefois, cette incertitude n’est pas un obstacle : il nous suffit de savoir que la connaissance est \cit{une bonne approximation de la réalité} voire qu’elle correspond peut-être à la réalité, sans que nous puissions en être certains \parencite{hunt2011philosophical}.  Il en découle que la démarche scientifique doit chercher à questionner et améliorer les théories existantes, pour donner la vision la plus exacte possible de la réalité. La science s’inscrit dans un processus continu de remise en question et d’amélioration. \\ 
    
    Les hypothèses posées sur le plan ontologique et épistémiques nous rapprochent du paradigme du \emph{réalisme scientifique}. Il constitue l’un des courants majeurs du post-positivisme \parencite{avenier2012inscrire} et correspond à la position la plus communément adoptée (souvent implicitement) par la sphère scientifique \parencite{bunge1993realism}. Son application en sciences de gestion a notamment été décrite et encouragée par Hunt. Nous décrivons maintenant les implications de cette posture épistémologique pour la thèse. \\
    
    Une spécificité du réalisme scientifique est de ne pas considérer la vérité comme une entité en tant que telle, mais comme un attribut. Ce n’est pas la Vérité qui est recherchée, mais le caractère vrai ou faux des théories. D’un point de vue sémantique, il en résulte que les énoncés doivent être formulés « comme ayant une valeur de vérité », peut importe qu’ils soient en fait vrais ou faux \parencite{chakravartty2015scientific}. La recherche s’attelle ensuite à déterminer si ces énoncés sont vrais, c'est-à-dire si le monde est tel que l’énoncé dit qu’il est. Le post-positivisme poppérien conteste la possibilité d’affirmer qu’un énoncé est vrai. En effet, pour Popper, l’induction, c'est-à-dire la formation de connaissances à partir de l’observation du terrain, ne permet pas de générer des connaissances scientifiques. Mille observations ne permettent pas de prédire le résultat de la mille et unième. Pour lui, la science a pour objectif de se rapprocher de la vérité en mettant à jour de nouvelles preuves permettant de réfuter les théories existantes et de formuler de nouvelles théories. La validité d’une théorie tient donc uniquement dans sa capacité à résister à la réfutation. \\
    
    Le réalisme scientifique accepte le principe de réfutation. Toutefois, il considère qu’une théorie est également renforcée par les résultats positifs des tests. Si une théorie a été vérifiée de nombreuses fois, on a de bonnes raisons de croire qu’elle correspond à la réalité \parencite{hunt2011philosophical}. Pour cela, il est par contre souhaitable d’avoir recours à différentes mesures pour s’assurer de la convergence des résultats \parencite{chakravartty2015scientific}. La première étape de la démarche réaliste est la formulation d’hypothèses, c’est-à-dire des énoncés réfutables ayant « valeur de vérité ». Celles-ci reposent sur les connaissances actuelles des entreprises sociales et de l’éco-innovation. Cependant, comme nous l'avons montré dans les premiers chapitres, nous disposons de trop peu de recherches sur l'action environnementale dans les \eess pour nous appuyer sur une théorie solide. Nous nous positionnons donc en amont du processus de recherche et tentons de donner des éléments pour bâtir une théorie et formuler des hypothèses. \\
    
    Les positivistes logiques comme les post-positivistes soulignent l’importance de la neutralité et de l’objectivité. Nous considérons qu’aucune méthode n’est parfaitement neutre. La neutralité doit être recherchée par le recours à des méthodes ayant le moins d’impact sur le sujet ou le phénomène observé. Toutefois, même une observation ‘lointaine’ a nécessairement un impact sur le phénomène observé, en particulier en sciences humaines. Il ne s’agit pas de dire que l’observation modifie le réel, mais simplement que le phénomène en situation d’observation n’est pas identique au phénomène ‘en temps normal’. De la même manière c’est notre conviction que l’être humain ne peut se détacher complètement de son histoire, de ses croyances et de ses valeurs. Le chercheur ne peut donc porter un regard parfaitement objectif sur le sujet qu’il étudie. Cependant, il peut (et doit) se dissocier complètement du sujet de l’étude.  Pour \textcite[][p.209]{bunge1993realism}, la description d’un fait est objective dès lors \citit{qu’elle ne fait pas référence à l’observateur, et qu’elle est raisonnablement vraie (ou constitue une approximation suffisante de la réalité)}. Le ressenti, les impressions ou les opinions du chercheur sont par conséquents exclus du périmètre de la recherche et il n’en est pas fait état dans la présentation des résultats. En aucun cas nous n’affirmons que ces perceptions n’existent pas ou n’ont pas de valeur, mais elles n’ont pas de validité scientifique. \\
    
    Nonobstant les limites à la neutralité et à l’objectivité, la création de connaissances vraies (au sens réaliste) est possible, puisque le réalisme scientifique met en avant la notion de faillibilité. Celle-ci implique la confrontation des résultats à la communauté scientifique, en lui donnant tous les moyens de vérifier ou de réfuter les résultats. Dans cette optique, chaque étape de la démarche empirique est décrite avec toute la précision possible. Les protocoles de recherche sont détaillés en annexe du document (sources de données, logiciels utilisés, lignes de code…) et les biais inhérents à chaque méthode sont systématiquement discutés. 


\transition
    Nous avons présenté la posture épistémologique adoptée pour cette recherche. Dans la section suivante, nous discutons des enjeux éthiques qu'elle présente. 