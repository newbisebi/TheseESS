\section{Ethique de la recherche}

Les données constituent aujourd’hui une véritable richesse économique, un nouvel \cit{or noir}. Les entreprises, mais aussi certains mouvements politiques, investissent des sommes considérables dans la collecte des données et leur usage à des fins marketings ou électorales. La question des données est particulièrement sensible lorsqu’elle touche à des variables individuelles et qu’elles questionnent le droit des individus à disposer des informations qui les concernent personnellement. Certains scandales récents nous alertent sur la nécessité de questionner nos pratiques et la façon dont les données sont collectées et utilisées. On citera par exemple l’affaire \cit{Facebook-Cambridge Analytics} dans laquelle des données collectées sur le réseau social ont été utilisées pour influencer la campagne électorale aux États-Unis. \\

La thèse mobilise plusieurs sources de données et il nous semble important, dans la mesure où l’éthique est une question centrale du sujet traité, de préciser comment elle a été prise en compte pour cette recherche. Une étude s’appuie sur des entretiens individuels et des documents relatifs à l’activité des entreprises (diffusés publiquement ou fournis par les organisations). Cette approche assez classique et communément utilisée dans la recherche en sciences de gestion ne pose pas de problèmes éthiques majeurs. Les données sont en effet collectées avec le consentement des répondants. Quelques principes ont toutefois été posés. (1) Les personnes interrogées ont été informées du contexte de la recherche et de l’utilisation faite des données, à savoir la production de travaux académiques. (2) Des garanties ont été données concernant la confidentialité des données, en particulier des entretiens enregistrés. Les répondants ont été informés que la diffusion des enregistrements se limiterait aux personnes concernées par la recherche (Directrice de thèse, jury de thèse, éventuellement co-auteurs pour la publication d’articles utilisant ces données). (3) Les répondants ont donné leur accord pour que leur nom, leur fonction et le nom de l’organisation soient cités dans les travaux de recherche. (4) Dans la mesure où la recherche est financée par des fonds publics et vise à contribuer au développement de la société, il a été proposé à tous les répondants d’être informés de toutes les publications résultant de ces études. \\

Une autre étude s’inscrit dans une perspective \cit{big-data}, s’appuyant sur des volumes importants de données collectées sur les réseaux sociaux à l’aide d’un programme informatique. Ce type de méthode est au cœur des préoccupations éthiques actuelles car elles questionnent la propriété des contenus publiés sur internet et des données rattachées aux émetteurs. Les différences entre un usage commercial du big-data et une utilisation académique des données peuvent sembler évidentes ; cependant il nous semble tout de même utile de le rappeler, cette méthode de recherche étant encore assez nouvelle dans le champ de la gestion. Il faut tout d’abord souligner que la portée de l’étude n’est pas comparable à celle que peut avoir un usage du big-data par des entreprises privées. Si le volume collecté est important et requiert l’automatisation à l’aide d’outils informatiques, il n’est nullement comparable à la masse des données utilisées par des entreprises privées. A titre de comparaison, nous traitons 950 000 messages diffusés par 1 110 entreprises, quand Cambridge Analytics a pu collecter l’intégralité des données Facebook de 50 millions d’utilisateurs. En outre, les données collectées sont exclusivement des contenus publics : il s’agit de messages diffusés sur un média (Twitter) ayant précisément vocation à toucher une audience large. Les utilisateurs de ce réseau ont la possibilité de restreindre l’accès aux contenus publiés. Dans ce cas, nous n'avons pas cherché à contourner cette limitation, et les utilisateurs ont simplement été retirés du panel et les contenus correspondants n’ont pas été collectés. Le véritable enjeu éthique réside toutefois dans l’utilisation des données qui se fait parfois au détriment de ceux à qui elles appartiennent initialement. Fréquemment utilisées à des fins commerciales, les données peuvent également être mobilisées dans une optique de manipulation, par exemple pour pousser un consommateur vers un produit ou un électeur vers une posture politique. Dans le cadre de cette thèse, naturellement, la finalité est scientifique et la seule utilisation des données est la réalisation de travaux de recherche. Les données collectées n’ont pas vocation à être vendues ou utilisées dans une démarche commerciale. La thèse s’inscrit dans une perspective post-positiviste dont un des principes est l’objectivité et la volonté de ne pas modifier (ou le moins possible) le phénomène observé.
