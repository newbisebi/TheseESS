% !TEX root = /Admin/main.tex

\section*{Introduction du chapitre}

  Dans les chapitres \ref{chapitre:twitter} et \ref{chapitre:casess}, nous analysons les résultats de deux études et les discutons indépendamment. L'objectif de cette partie est de confronter les conclusions de ces deux études et de présenter une analyse globale de l'ensemble de la thèse. Ce chapitre vise également à prendre du recul sur le travail réalisé et à s'interroger sur ses implications, mais aussi ses limites. \\

  Dans une première section, nous revenons sur les principaux résultats obtenus et en proposons une synthèse. Nous en tirons des implications, non seulement pour les \eess, mais aussi pour la société en général. En effet, nous considérons que la recherche en Sciences de Gestion, en particulier la recherche publique, vise certes à apporter des outils et des connaissances aux entreprises, mais qu'elle est également au service de la société. Il nous semble donc pertinent d'avoir une réflexion sur la façon dont la recherche nous éclaire sur l'organisation de la société et du système économique. \\

  La deuxième section de ce chapitre revient plus en détail sur les aspects académiques de la recherche. Nous nous interrogeons sur la validité interne des études qui sont conduites, et discutons de la pertinence de la généralisation des résultats. Nous nous arrêtons par ailleurs sur les limites de la recherche. Les méthodes utilisés, ainsi que les concepts mobilisés, présentent inévitablement des contraintes qui impliquent une nécessaire prise de précautions dans la lecture des résultats. Nous terminons en ouvrant sur des perspectives d'approfondissement de cette recherche. Nous restons convaincu de l'importance d'étudier l'action environnementale dans l'ESS et encourageons de futures études sur cette thématique encore peu explorée dans notre discipline. Nous détaillons certains éléments de réflexion s'appuyant sur les cadres théoriques de la résistance au changement d'une part, et du business model d'autre part, qui peuvent constituer une base de travail pour de futurs travaux.

\section{Communication et action environnementale dans l'ESS}

  Dans cette première section, nous rappelons les principaux résultats de l'étude et discutons des conséquences qui peuvent en être tirées pour les \oess, mais aussi plus largement pour la société.

    \subsection{Principaux résultats}

        Les deux études qui constituent cette thèse convergent pour montrer que la protection de l'environnement est bien prise en compte dans l'ESS, y compris par des organisations dont ce n'est pas le rôle premier. La majorité des \oess communiquent sur Twitter sur des sujets en lien avec l'environnement et plusieurs cas étudiés font ressortir de réelles préoccupations environnementales. En matière de communication, il semble que les coopératives et les entreprises sociales, qui sont, structurellement, les plus proches de l'économie de marché classique, sont celles qui consacrent la part la plus importante du discours à l'environnement. Cependant, les deux études permettent de constater que leur action environnementale est motivée essentiellement par les opportunités économiques et de développement qu'elle présente ou par le gain de légitimité qu'elle apporte. Elles adoptent alors une stratégie plutôt adaptative (voir tableau \ref{table:strategiesenvir}). Au contraire, les organisations qui se distinguent fortement du secteur lucratif ont un positionnement à la fois plus engagé, prenant part à un débat politique et alertant sur les risques de la non prise en compte de l'environnement. Ces organisations, en particulier celles qui mettent au premier plan leur engagement écologique, vont plus loin dans la démarche et adoptent une stratégie proactive. En outre, malgré les difficultés (notamment économiques) qu'elles rencontrent, elles disposent de nombreuses ressources pour mettre en oeuvre de nouvelles technologies ou de nouvelles pratiques. Les organisations de ce type trouvent un écho important et utilisent des stratégies qui fonctionnent sur les réseaux sociaux. \\
        % \todo[inline]{\textcite{bidet2003insoutenable} ==> isomorphismes, économie plurielle: ESS tiraillée entre isomorphismes et opposition au marché}

        Un autre enseignement de l'étude est celui de l'impact positif de l'identité collective des \oess sur l'action environnementale. Celui-ci ressort en particulier de l'étude de cas : les partenariats ainsi que l'intervention de parties prenantes multiples aux activités et à la gouvernance des organisations renforcent la capacité d'innovation, notamment en matière environnementale. Elle permet de contourner les principales barrières rencontrées par les entreprises dans leur action environnementale. Ce résultat est conforme à ce que trouvent \textcite{mojo2015social} dans le contexte des coopératives. Les auteurs indiquent en effet que l'action collective améliore l'engagement des producteurs dans la protection de la nature. Pour \textcite{musson2018les}, l'effet positif de la dimension collective des coopératives sur l'innovation passe par le renforcement de la confiance. Bien que ce dernier paramètre ne fasse pas partie du cadre de notre étude, il peut effectivement être observé dans plusieurs de cas de l'analyse qualitative. \\


        Cependant, il apparaît que le contexte général dans lequel évoluent les \oess rend difficile l'action environnementale. Elles perçoivent peu d'opportunités de développement en matière d'écologie et souffrent d'un manque important de ressources. Les efforts à mettre en oeuvre pour mener et maintenir une action environnementale efficace sont importants, et nécessitent un engagement humain et financier conséquent. Cependant, celles qui s'y engagent semblent en tirer de réels bénéfices.


        \begin{table}[]
          \centering
          \caption{Stratégies environnementales des organisations de l'\ess}
          \label{table:strategiesenvir}
              \begin{tabularx}{\linewidth}{|K{0.205\linewidth}|X|X|}
                  \hline
                   & \textbf{Attitude opportuniste} & \textbf{Attitude engagée} \\ \hline

                  \textbf{Communication}
                      & Rhétorique positive \newline Discours sur les opportunités, perspectives économiques
                      & Rhétorique négative \newline Discours sur les valeurs et la responsabilité \\ \hline

                  \textbf{Motivations}
                      & Compétitivité, légitimité
                      & Responsabilité (engagement) \\ \hline

                  \textbf{Portée de l'action environnementale }
                      & Niveau organisationnel \newline Adoption d'innovations
                      & Niveau organisationnel et démarche de diffusion \newline Développement d'innovations\\ \hline

              \end{tabularx}
          \end{table}

    \subsection{Implications pour l'\ess et la société}
        \subsubsection{Apports de la pluralité de l'ESS}
        Les différentes approches théoriques de l'\ess révèlent des disparités entre une vision purement sectorielle, la vision d'une économie qui doit s'intégrer au système dominant, ou celle d'une économie alternative. Ces différentes perspectives se retrouvent sur le terrain, et les acteurs de l'\ess perçoivent eux même leur rôle différemment. Les deux études mettent en évidence des organisations qui s'inscrivent dans une perspective économique, pleinement compatible avec le système capitaliste. Elles valorisent une activité économique rentable et un fonctionnement managérial similaire à celui du secteur lucratif. Elles se rattachent à l'\ess par leur mission sociale ou environnementale et par certaines spécificités, par exemple leur caractère coopératif. Inversement, d'autres acteurs se positionnent de manière distincte, voire en opposition, avec le secteur capitaliste. La mission sociale mais aussi les modalités de sa mise en oeuvre sont prioritaires et la dimension économique apparaît comme une contrainte qu'il faut gérer plutôt comme une finalité. \\

        La diversité d'approches dans l'ESS explique en partie les difficultés de visibilité et de légitimation du secteur. Si le positionnement de l'\ess semble manquer de clarté, c'est qu'il doit en réalité composer avec des réalités multiples et parfois contradictoires. Le risque d'une vision unique de l'ESS, institutionnalisée par des lois comme celle de 2014, ou par des rapprochements entre les organismes représentatifs, est d'effacer ou de minimiser certaines approches, certaines façons de voir et de pratiquer l'économie solidaire et sociale. Pourtant, il existe une complémentarité dans ces différentes visions. Un modèle profitable d'entreprises a le potentiel pour redonner du poids à cette économie et pour gagner en efficacité dans le traitement des enjeux sociaux qui manquent encore de réponse. Il contribue aussi à re-légitimer une économie décriée face à une vision largement partagée de l'entreprise, dont le but perçu est avant tout la croissance économique. Mais à l'inverse, il est évident que le système dominant trouve aujourd'hui ses limites, qui se manifestent dans les déséquilibres sociaux et dans la crise environnementale dont les effets se font maintenant ressentir de façon brûlante. Contraindre l'\ess à reproduire les pratiques d'un modèle en crise paraît alors absurde, en particulier quand ses organisations sont capables de produire des alternatives sérieuses et de répondre efficacement à certains enjeux. La recherche va donc dans le sens de Karl Polanyi qui défend une économie plurielle \parencite{laville2003avec}, permettant à plusieurs modèles de s'exprimer et de bénéficier ainsi de leurs avantages. \\

        Au regard de l'action environnementale, la pluralité de l'ESS constitue un véritable atout. Les entreprises sont capables d'appréhender ces enjeux de multiples façons et de mobiliser différents leviers en fonction de leur positionnement. Au niveau de la communication, l'ESS adopte plusieurs approches. Certaines organisations jouent un rôle d'information et sensibilisent le public aux enjeux environnementaux, en lien avec les questions sociales. D'autres, plus engagées, se font le porte-voix d'une partie de la société en alertant sur l'urgence d'agir. Enfin, des entreprises montrent la possibilité d'avancer sur de nouvelles voies de développement, en soulignant que la prise en compte de l'environnement n'est pas opposée au progrès, mais peut être créatrice de valeur. Sur le plan de l'action environnementale, les EESS qui souhaitent s'inscrire dans une démarche écologique bénéficient aussi de leviers intéressants. Les modèles du secteur non lucratif permettent de mener des actions qui ne sont pas justifiées uniquement par les profits : elles peuvent par conséquent mener des projets à caractère environnemental qui ne sont pas particulièrement rentables ou qui n'ont pas d'impact marketing. L'impact environnemental est en lui même une finalité. Pour autant, les OESS ne sont pas aveugles aux opportunités économiques associées à l'action environnementale. Elles présentent en outre des atouts pour capter ces opportunités. \\

        \subsubsection{Soutenir l'action environnementale dans l'ESS}
        Malgré les atouts que présente l'ESS sur le plan de l'écologie, la place des questions environnementales dans l'ESS pose un réel questionnement. En effet, pourquoi l'écologie, qui constitue un enjeu crucial pour l'humanité, n'occupe t-elle pas une place plus centrale dans des organisations qui agissent pour l'intérêt général ? On comprend aisément que certains enjeux sociaux apparaissent plus urgents, plus préoccupants que la protection de l'environnement. Cependant, il existe une forte interdépendance entre les questions sociales et environnementales \parencite[voir par exemple][]{bodin2019improving, reuveny2007climate}. La protection de l'environnement nécessite de repenser l'organisation de la société et les rapports sociaux. C'est donc un enjeu qui concerne l'ensemble de l'ESS, et pas seulement les organisations qui ont une mission environnementale. \\

        Notre recherche apporte certaines indications sur les facteurs qui peuvent soutenir l'action environnementale dans l'ESS. Tout d'abord, l'étude menée sur Twitter met en lumière la place centrale des organismes de représentation, ainsi que celle de certaines grandes organisations, notamment des ONG. Ces acteurs bénéficient donc d'une posture privilégiée et d'une réelle capacité d'influence pour orienter l'ESS vers les questions environnementales. C'est donc en premier lieu à ces organisations de porter un discours écologique. Celui-ci peut viser à éclairer les EESS sur les atouts de l'action environnementale. La recherche de compétitivité est un déterminant de l'action environnementale, mais les organisations peinent à déceler des opportunités. Il est donc possible de jouer sur ce levier en répertoriant les actions environnementales bénéfiques pour les organisations et en leur montrant l'intérêt de porter leur attention sur ces sujets. L'engagement apparaît également comme un puissant levier d'action dans les OESS : les personnes engagées dans l'ESS sont souvent portées par des convictions et s'attachent aux valeurs de l'ESS. Toutefois, l'écologie n'est pas systématiquement identifiée comme un élément central de l'ESS. En cela, les pouvoirs publics ont un rôle à jouer dans la manière dont l'ESS est institutionnalisée. L'article 3 de la loi sur l'ESS prévoit six catégories de \cit{bonnes pratiques} (cf. encadré \ref{encadre:pratiquesess}) dont aucune n'inclut la prise en compte de l'environnement naturel. Cet oubli semble en décalage avec les enjeux actuels et les futures réflexions sur l'ESS pourraient davantage tenir compte des enjeux environnementaux. Enfin, un facteur clé de l'action environnementale est la dimension collective des organisations. Plutôt que d'orienter l'ESS vers des logiques concurrentielles, il convient de valoriser les statuts qui font une place importantes aux parties prenantes. Le modèle des SCIC, dont le nombre augmente rapidement \parencite{observatoire_national_de_leconomie_sociale_et_solidaire_france2017atlas}, pourrait être davantage mis en avant. Il est en effet conçu pour permettre la participation de parties prenantes au sens très large. Les SCIC peuvent également être utilisées pour constituer des groupes d'entreprises (sans liens capitalistiques) ou encore des PTCE (cf. section \ref{ptce}). \\

        \begin{encadre}
        \begin{tcolorbox}
          \caption{Extrait de l'article 3 de la loi de 2014 sur l'ESS}
          \label{encadre:pratiquesess}
          \cit{
            Ces bonnes pratiques concernent notamment : \\
            1 - Les modalités effectives de gouvernance démocratique ;\\
            2 - La concertation dans l'élaboration de la stratégie de l'entreprise ;\\
            3 - La territorialisation de l'activité économique et des emplois ;\\
            4 - La politique salariale et l'exemplarité sociale, la formation professionnelle, les négociations annuelles obligatoires, la santé et la sécurité au travail et la qualité des emplois ;\\
            5 - Le lien avec les usagers et la réponse aux besoins non couverts des populations ;\\
            6 - La situation de l'entreprise en matière de diversité, de lutte contre les discriminations et d'égalité réelle entre les femmes et les hommes, en matière d'égalité professionnelle et de présence dans les instances dirigeantes élues.
          }\\ \parencite{noauthor2014loi} \\
          \end{tcolorbox}
        \end{encadre}

        A un niveau plus micro-économique, la thèse apporte des éléments pour les entreprises de l'ESS qui souhaitent communiquer sur les enjeux écologiques ou mener une action environnementale. Plusieurs stratégies  s'avèrent pertinentes pour communiquer sur les réseaux sociaux, en fonction du positionnement souhaité. Deux approches opposées aboutissent à une bonne performance des tweets. Les organisations peuvent se positionner dans une démarche d'information et de mise en garde sur l'urgence écologique. Dans ce cas, elles adopteront un discours plutôt négatif et n'hésiteront pas à prendre position dans les débats politiques. Inversement, une approche plus économique est possible, valorisant davantage les opportunités de développement. La communication peut souligner la capacité de l'ESS à générer des innovations environnementales et ainsi participer à la création de richesses du pays. Si notre étude s'arrête à étudier la performance des stratégies dans l'absolu, on peut faire l'hypothèse que ces deux positionnements permettent de s'adresser à des publics différents. Un discours négatif, alarmiste et très politique va probablement obtenir l'adhésion de publics militants, déjà sensibilisés et ayant la volonté de s'engager. Inversement, un discours positif, accentuant les alternatives et les voies d'un développement écologique pourra atteindre un public qui s'intéresse aux solutions plutôt qu'à la critique et qui préfère des réponses individuelles aux réponses collectives et politiques. Finalement, du point de vue de l'action environnementale, nous adressons deux recommandations aux entreprises de l'ESS. D'une part, les convictions portées par les membres de l'organisation jouent un rôle important dans l'action environnementale. Leur implication dans les projets d'innovation environnementale est donc à recommander. Non seulement les salariés et les bénévoles peuvent être moteurs dans ces projets, leur participation peut également aider à accepter les changements induits. Dans bien des cas, l'action environnementale est aussi valorisante pour ces personnes. D'autre part, nous suggérons de valoriser l'action collective aussi bien en interne qu'en externe, en multipliant notamment les partenariats. Les échanges entre organisations facilitent l'accès aux ressources nécessaires pour innover et permettent de bénéficier de la complémentarité entre différents types de structures.

        % Pouvoirs publics : https://www.entreprises.gouv.fr/files/files/directions_services/etudes-et-statistiques/prospective/PIPAME-circuits-courts-alimentaires.pdf


\section{Apports, limites et perspectives}

    Dans la section précédente, nous discutons des résultats et des apports de la recherche pour la société. Dans cette section, nous discutons des aspects académiques de la thèse. Nous soulignons ses contributions conceptuelles et méthodologiques, mais aussi les questions de validité, et soulignons les limites inhérentes à la recherche. Nous terminons par la proposition de voies de recherche potentielles.

    \subsection{Principaux apports pour la recherche}

    Sur le plan théorique, cette recherche a permis de tester certains modèles dans le cadre particulier des organisations de l'ESS. En raison de leurs spécificités, notamment le caractère non lucratif de nombreuses organisations, la propriété ou la gestion collective et le recours à des modes de fonctionnement particuliers, on peut s'interroger sur la pertinence de théories et modèles généralement basés sur l'étude d'entreprises classiques. Les chercheurs qui se consacrent à l'entreprise sociale ne s'accordent pas sur la nécessité de développer des théories spécifiques à ce domaine \parencite{valentinov2008economics, dacin2010social, weerawardena2006investigating, santos2012positive}. Ainsi, alors que \textcite{santos2005organizational} proposent un modèle basé sur la création de valeur, ou \textcite{weerawardena2006investigating} un autre reposant sur l'innovation, la proactivité et le management des risques, \textcite{dacin2010social} contestent la nécessité de nouvelles théories. Pour \textcite{yunus2010building-1}, le concept de business model peut être utilisé à condition d'y intégrer une nouvelle composante : le profit social (\citit{social profit equation}). \\

    De la même façon, nous testons un modèle des déterminants de l'éco-innovation et montrons qu'il est adapté pour décrire le comportement des organisations de l'ESS. S'il nécessite quelques reformulations et fait apparaître des éléments spécifiques à l'ESS, le modèle de \textcite{bansal2000why} s'applique bien à ce cadre spécifique. Toutefois, nous le complétons en ajoutant la dimension collective comme déterminant de l'action environnementale dans l'ESS. En effet, l'étude de cas met en évidence un impact fort de la coopération, qu'elle se fasse au niveau de la gouvernance ou à travers des partenariats, voire de l'engagement ponctuel de parties prenantes diverses. Cet élément, central dans l'histoire de l'économie sociale, est essentiel pour acquérir les ressources nécessaires à l'action environnementale. Le volet \cit{responsabilité} du modèle de Bansal et Roth peut également être reformulé. L'action environnementale n'est pas seulement quelque chose que l'on doit faire parce que l'on se sent \cit{responsable} ou parce que cela semble juste. La responsabilité environnementale prend ici la dimension d'un réel engagement, d'une volonté d'agir basée sur des convictions fortes. \\

    Le concept d'identité organisationnelle est déjà mobilisé dans la littérature sur l'ESS, notamment par \textcite{chedotel2012linfluence} ou \textcite{young2000alternative, young2001organizational, young2001organizational-1}. Appliqué à notre recherche, il se révèle efficace pour appréhender l'\ess dans toute sa diversité. Les organisations ayant une forte identité utilitariste sont motivées par des enjeux de compétitivité et se légitiment par leur ressemblance avec le secteur classique. Inversement, celles qui s'appuient sur une identité normative et sont portées par des valeurs peuvent aller plus loin dans l'action environnementale. Pour certaines organisations, la pérennité est presque secondaire, puisqu'une action en contradiction avec les engagements et convictions n'aurait pas de sens. Le recours au concept d'identité permet d'échapper à certaines simplifications, comme l'opposition \cit{marché VS valeurs}. En effet, conformément à \textcite{foreman2002members}, nous observons que les identités cohabitent et que le recours à des pratiques de marché ne s'opposent pas nécessairement à l'engagement. En outre, nous montrons ainsi que l'appartenance à l'\ess a plusieurs formes : elle peut résider dans les valeurs elles-mêmes, mais aussi dans la poursuite d'une mission, indépendamment des modalités d'action, ou dans le caractère collectif de l'activité. Nous complétons donc les approches précédentes en ajoutant l'identité fonctionnelle, déjà identifiée dans nos précédents travaux \parencite{mariaux2018leconomie}, et l'identité collective. Ces deux identités revêtent bien un caractère \citit{central, durable et distinctif} dans l'ESS. La thèse démontre que si certaines organisations valorisent des principes moraux (identité normative) ou plutôt leur rôle économique (identité utilitariste), d'autres se concentrent en premier lieu sur la contribution de leur activité et de leur capacité à trouver des solutions à des problèmes sociaux ou environnementaux (dimension fonctionnelle). Enfin, pour certaines organisations, l'identité collective est incontournable et constitue une caractéristique réellement différenciante de l'ESS. C'est pourquoi elle nous semble incontournable pour rendre compte de ce que représente l'économie sociale.

 \subsection{Validité scientifique de la recherche}

        \subsubsection{Validité interne et externe}

        \textbf{Validité interne} \\
        Nous avons discuté dans le chapitre \ref{chapitre:twitter} des avantages et des inconvénients de la collecte et de l'exploration automatique de texte. Elle permet une bonne reproductibilité de l'étude et limite l'interprétation du chercheur. Le panel d'entreprises est constitué de manière à limiter les biais de sélection : une première sélection est faite à partir de listes institutionnelles d'\eess et d'une recherche par mots clés dans le réseau social lui-même afin de donner un caractère aléatoire. Les analyses visent à réduire l'intervention du chercheur. Cependant, leur caractère quantitatif ne doit pas donner l'illusion d'une parfaite objectivité : l'application des concepts comme les cadres rhétoriques, le pré-codage des données d'entrainement et la détermination du caractère environnemental des tweets impliquent inévitablement des choix et des interprétations. \\

        L'étude qualitative est également conduite de manière rigoureuse afin de permettre des comparaisons pertinentes entre les cas. Tous les entretiens s'appuient sur un même guide et les sujets traités sont donc similaires. Le codage à l'aide de Nvivo permet d'homogénéiser l'analyse des données et d'objectiver dans une certaine mesure l'observation des identités. Nous restons toutefois prudent quant à la quantification des éléments codés qui sont fortement marqués par l'importance donnée à un thème par le répondant et par sa propension à élaborer sur ces sujets. L'apport de l'étude documentaire vient solidifier l'analyse, en apportant des éléments sur l'identité organisationnelle et l'action environnementale qui ne résultent pas de l'interprétation du seul répondant. \\

        \textbf{Validité externe} \\
        Dans quelle mesure les résultats de la thèse peuvent-il être généralisés ? Le premier élément sur lequel on peut s'interroger est celui de la pertinence du contexte français pour étudier l'ESS. Cette économie est historiquement très ancrée dans notre pays. Bien qu'elle soit parfois contestée et, comme nous l'avons vu, poussée à évoluer, elle est omniprésente dans la vie économique et sociale de la France. Nous pouvons donc estimer que la France est un bon terrain d'étude pour s'intéresser à l'ESS. En outre, elle est structurée de manière similaire dans de nombreux pays, en particulier européens, ainsi qu'au Canada. Cependant, l'approche anglo-saxonne est un peu différente, plus centrée sur le secteur non lucratif. Mais de réelles différences peuvent être observés dans \cit{les suds}, selon l'expression de \textcite{laville2016economie}, notamment en Afrique ou en Amérique latine, où l'\ess est prédominée par une économie populaire. Celle-ci s'appuie sur des organisations et des échanges informels, hors de tout cadre institutionnel. Une telle économie existe en France, mais de manière bien moindre, et elle échappe au cadre de notre étude. Par conséquent, on peut s'attendre à des résultats très éloignés en matière d'action environnementale dans de tels contextes. L'économie populaire est souvent une économie de la survie : on peut donc s'attendre à ce que les préoccupations environnementales, plus éloignées du quotidien des individus, soient mises au second plan, voire complètement éludées. \\

        La question se pose également de la généralisation des résultats des études au sein même du contexte de l'\ess dans les pays européens. L'étude du discours sur Twitter est fortement contextualisée. Le réseau social a ses propres codes, ses propres isomorphismes et ses contraintes (notamment en termes de nombre de caractères). L'expression des organisations sur ce réseau est donc simplifiée, synthétisée à travers des messages courts. En outre, la communication sur les réseaux est non vérifiée et non engageante. Il existe cependant un contrôle informel par les utilisateurs eux mêmes : les retours sur Twitter sont immédiats et peu tempérés, ce qui peut inciter les organisations à adopter un discours édulcoré et plus consensuel. Il est donc possible que le discours porté sur Twitter soit différent de celui véhiculé dans d'autres médias et que les résultats en matière de rhétorique environnementale diffèrent. Cependant, la force de l'étude est de s'appuyer sur un échantillon conséquent d'organisations et de porter sur un très grand nombre de messages. Ceci confère aux résultats une validité statistique, confirmée par les indicateurs calculés à chaque étape de l'analyse. \\

        L'étude qualitative, enfin, est la plus difficilement généralisable ; ce n'est d'ailleurs pas sa vocation. Le panel est réduit à sept cas, ce qui ne suffit pas à représenter la diversité de l'ESS, et encore moins à procéder à une généralisation statistique. C'est pourquoi elle aboutit sur la formulation de propositions plutôt que sur des résultats définitifs. Elle a pour objet de révéler des liens entre identité organisationnelle et action environnementale, mais ne permet nullement de les mesurer. Cette étude est donc un appel à davantage de recherches et à l'élaboration d'outils permettant de tester statistiquement la validité des propositions émises.

        \subsubsection{Principales limites}

        Le paragraphe précédent met en exergue certaines limites à notre travail et certaines précautions à adopter pour interpréter nos résultats. La principale limite est la difficulté de la généralisation des résultats liée aux modalités d'étude (une étude menée dans un contexte très spécifique, l'autre portant sur un petit échantillon). D'autres limites doivent être soulignées. \\

        Une d'entre elles porte sur la diversité de l'ESS. Au sein même du secteur associatif, \textcite{valeau2003differentes} souligne les grandes variations qui rendent difficiles les comparaisons. Cette difficulté est d'autant plus forte que nous traitons ici d'un périmètre plus étendu. Le concept d'identité organisationnelle contribue efficacement à rendre compte de cette hétérogénéité. Cependant il se heurte des difficultés d'opérationnalisation, en particulier dans une démarche quantitative. En effet, si nous sommes parvenu à déterminer les identités des \oess à partir des entretiens, l'exercice est moins aisé à mettre en oeuvre à partir d'un questionnaire. A plus forte raison, les données extraites de Twitter, non structurées et non guidées, ne permettent pas d'exploiter réellement ce concept. On est ainsi contraint d'appuyer l'étude sur les statuts organisationnels qui peuvent s'avérer trompeurs et ne reflètent pas effectivement l'identité de l'entreprise. \\

        Une autre limite importante est le recours à des éléments indirects pour évaluer l'action environnementale des organisations. Ceci renvoie à la difficulté de la mesure de l'éco-innovation \parencite{arundel2009measuring, kemp2007final}. En effet, la première étude mesure la prise en compte de l'environnement par le nombre de tweets consacrés à l'environnement. Or, si cela peut rendre compte de l'intérêt de l'association pour les problématiques écologiques, ça n'est pas une mesure de leur engagement concret dans des actions environnementales. Les résultats rendent compte d'une intention davantage que d'une action. En outre, les biais de communication sont importants et l'effet \citit{greenwashing} ne saurait être négligé. Les organisations qui communiquent fortement sur l'environnement sont-elles sensibles à la protection de la nature ou bien... à leur propre image ? La pertinence de l'étude réside donc dans l'interprétation qui est faite de ces résultats : ainsi, on s'attachera moins à la quantité de contenu relatif à l'environnement qu'aux stratégies rhétoriques adoptées. Celles-ci sont plus à même de nous renseigner sur la façon dont l'écologie est perçue par les organisations (risque ou opportunité ?). \\

        Par ailleurs, l'action environnementale doit être distinguée de la performance environnementale. En effet, nous nous basons sur les propos des répondants et sur la documentation pour déterminer ce qui est fait en matière environnementale. Toutefois, nous ne disposons pas d'outils permettant de juger de l'impact réel de ces actions et de juger si elles sont symboliques ou réellement efficaces. Cette limite est d'autant plus importante que la différence entre intention et effet réel est cruciale en matière d'environnement. Les effets cachés sont nombreux et il est peu aisé de choisir le meilleur comportement en matière de protection de la nature. Deux exemples assez courants permettent d'illustrer ce phénomène. Le recours au coton bio pour les vêtements est une pratique d'apparence écologique, mais la production du coton bio est très consommatrice d'eau et a donc un impact sur l'environnement. De même, la voiture électrique permet d'émettre moins de gaz à effet de serre : toutefois sa production nécessite l'extraction de métaux rares et est très coûteuse en énergie. Toute action environnementale nécessite donc une réflexion sur ses effets réels et doit aboutir au meilleur compromis. La thèse s'arrête à l'étape de l'observation des actions menées et ne va pas jusqu'à l'évaluation des effets réels.

    \subsection{Approfondissements et voies de recherche}
        Comme le montre l'analyse de la littérature faite dans les premiers chapitres, la recherche sur la prise en compte de l'environnement dans l'ESS reste à un stade embryonnaire. La thèse apporte de nouveaux éléments, mais soulève également un besoin d'approfondissement. Dans ce paragraphe, nous proposons des pistes de recherche et des théories pouvant porter la lumière sur ces questions.

        \subsubsection{Extension de la thèse}
            Dans la section précédente, nous évoquons les limites de la thèse. Celles-ci soulèvent la nécessité de compléter la connaissance en utilisant d'autres méthodologies. En particulier, il semble indispensable de disposer de données chiffrées plus précises sur l'action environnementale dans l'ESS. Les propositions formulées pourraient notamment être testées de manière statistique. Une telle recherche soulèverait toutefois deux difficultés. D'une part, celle de l'opérationnalisation du concept d'identité, afin d'en permettre la mesure, et celle de l'hétérogénéité du terrain étudié. Nous avons été confronté à ces difficultés dans le cadre d'une étude menée en parallèle de la thèse. Un premier questionnaire, validé par des représentants de fédérations d'\oess, s'est révélé inapproprié pour certaines organisations. A l'inverse, l'usage de questionnaires spécifiques aux différents types d'organisations ne permettrait pas de rendre compte de l'\ess dans son ensemble. \\

            On pourrait par ailleurs s'interroger sur l'effet de l'évolution du contexte général sur les \oess. Au cours des dernières années, on observe une attention grandissante sur le sujet du réchauffement climatique, mais aussi une forte polarisation des opinions sur la question. Peut-on encore ignorer l'écologie lorsque l'on agit au sein de l'\ess ? La pression des parties prenantes sur les questions environnementales est-elle plus forte suite à ces changements et de quelle manière les organisations peuvent-elles y répondre ?  \\

            Le caractère non lucratif (ou à lucrativité limitée) de l'\ess ouvre aussi des voies d'exploration intéressantes quant à son action environnementale. Dans l'économie classique, le rôle de l'entreprise est de gagner de l'argent pour les actionnaires. Dès lors, l'action environnementale est portée par le retour sur investissement espéré (qu'il s'agisse d'un gain financier ou d'un bénéfice en termes d'image et de légitimité). A l'inverse, les \oess peuvent mener des actions sans attendre de retour sur investissement. Là encore, on se demandera quelles sont les attentes des parties prenantes et dans quelle mesure elles peuvent soutenir de telles actions. La théorie des parties prenantes peut également éclairer sur les jeux de pouvoir, l'influence des différents groupes et les effets sur l'action environnementale. \\

            Enfin, en mobilisant la théorie institutionnelle, la recherche pourrait se pencher sur l'impact des isomorphismes sur l'action environnementale. Les modèles de l'\ess sont-ils encore pertinents pour mettre en place une action environnementale innovante ? Ou bien, poussées à ressembler de plus en plus aux organisations du secteur lucratif, les \oess sont-elles vouées à reproduire les mêmes pratiques ? Nos résultats suggèrent que des stratégies proactives sont plutôt menées par des organisations qui, d'une part, s'éloignent d'un pur fonctionnement de marché et, d'autre part, dont les membres sont animés par une forte conscience écologique. \\

            Le cadre théorique du changement organisationnel et celui du \citit{Business Model} peuvent être utilisés pour approfondir ces recherches.

        \subsubsection{Théories mobilisables}
        \textbf{Le changement organisationnel} \\
            L'engagement personnel et collectif constitue un déterminant puissant de l'action environnementale dans les OESS. Cependant, nous voyons qu'il existe aussi des résistances humaines aux changements qu'elle implique. Alors que les organisations doivent s'adapter et se transformer pour répondre aux évolutions du contexte dans lequel elles évoluent, les changements font généralement face à des oppositions de la part des personnes qui subissent le changement \parencite{mariaux2011impact, thomas2011reframing, lines2004influence}.  Si la résistance au changement peut être perçue comme un obstacle à une démarche nécessaire, elle peut aussi être interprétée comme un facteur de succès \parencite{thomas2011reframing}. En effet, pour \textcite{thomas2011managing}, dans un contexte de dialogue, la résistance donne de la matière pour avancer et produire de nouvelles connaissances, et donc finalement faciliter le changement. Plus encore que la communication, la participation active des personnes impactées par un changement dans le processus de mise en oeuvre constitue un facteur de succès des projets organisationnels \parencite{lines2004influence}. \\

            Cette approche soulève une piste d'étude intéressante dans le cadre de l'ESS, où l'implication des salariés, mais plus largement des parties prenantes, tient une position centrale. Or, le changement organisationnel peut être vu comme un ensemble de compromis entre différents acteurs plutôt que comme une stratégie \cit{gagnant-gagnant} \parencite{staw2000what}. Dans cette perspective, le rôle déterminant du collectif dans la mise en oeuvre de projets environnementaux pourrait être expliqué par sa capacité à favoriser le changement. \\

            La réussite des changements organisationnels est également expliquée par un alignement entre les valeurs des membres de l'organisation et l'objectif que l'on cherche à atteindre \parencite{burnes2011success}. Notre étude de cas va dans le même sens et les valeurs partagées au sein de l'organisation jouent un rôle favorable à l'action environnementale. Il serait intéressant d'étudier plus en détail les valeurs individuelles des personnes travaillant dans les \eess et de s'interroger sur l'alignement avec les objectifs environnementaux. Si on peut sans trop de risque parier sur la sensibilité écologique d'une personne engagée dans une ONG de protection de l'environnement, qu'en est-il d'un employé d'une mutuelle, d'une coopérative agricole ou d'une association humanitaire ? \\

        \textbf{Le business model durable} \\
            D'abord exploité sur le terrain, notamment porté par le regain d'intérêt pour l'entrepreneuriat, le concept de Business Model a fait l'objet de nombreuses publications au cours des dernières années. Ayant été mis en pratique avant d’être clairement défini théoriquement, il a souffert d’une certaine confusion sur le plan conceptuel. Il peut être vu comme \citit{des histories qui expliquent comment les entreprises fonctionnent} \parencite{magretta2002why} ou, plus pragmatiquement, comme \citit{les choix qu’une entreprise effectue pour générer des revenus} \parencite{lecocq2006business}. Malgré des divergences sur la définition et les composantes du Business Model, les auteurs s’accordent sur sa finalité : expliquer comment les organisations créent de la valeur et s’en approprient une partie. Si l'on écarte la dimension d'appropriation de la valeur, ce concept est parfaitement applicable à l'ESS, y compris au secteur non lucratif. En outre, les considérations sociales et environnementales s'intègrent parfaitement à ce cadre d'étude \parencite{birkin2009new, bocken2014literature, boons2013business, upward2016ontology}.\\

            L'angle du Business Model présente un vrai potentiel pour comprendre comment les \oess s'adaptent à l'évolution du contexte et répondent aux injonctions à se rapprocher des entreprises capitalistes. Il permet d'avoir une approche globale de l'organisation et de mettre en évidence les schémas par lesquels les organisations peuvent continuer à créer de la valeur sociale et environnementale tout en assurant leur modèle économique. Une telle étude peut aboutir à la constitution d'une typologie d'\oess en fonction du modèle adopté. Il est alors envisageable de questionner la pertinence des différentes catégories au regard des objectifs à atteindre. De manière intuitive, la thèse permet d'identifier plusieurs business models dans l'\ess (cf. tableau \ref{table:bm_ess}). \\


            \begin{table}
              \caption{Une classification des business models de l'\ess}
              \label{table:bm_ess}
              \begin{tabularx}{ \linewidth }{ |K{0.17\textwidth}
                                              |K{0.17\textwidth}
                                              |K{0.17\textwidth}
                                              |K{0.17\textwidth}
                                              |K{0.17\textwidth}
                                              |
                                            }
                \hline

                \multirow{2}{*}{}
                &	\multicolumn{2}{|C|}{\textbf{Modèles basés sur le marché}}
                & \multicolumn{2}{|C|}{\textbf{Modèles non basés sur le marché}}
                \\ \cline{2-5}

                &\textbf{Business Model entrepreneurial}
                & \textbf{Business Model coopératif}
                & \textbf{Business Model militant}
                & \textbf{Business Model associatif} \\ \hline

                \textbf{Finalité} & Economique et sociale & Economique et sociale  & Sociale & Sociale \\ \hline
                \textbf{Initiative} & Individuelle & Collective & Collective & Collective \\ \hline
                \textbf{Prix} & Prix de marché & Prix de marché & Prix non basés sur le marché & Prix non basés sur le marché \\ \hline
                \textbf{Coûts financés par les revenus de l'activité} & Oui & Généralement oui (mais subventions possibles) & Généralement oui (mais subventions possibles)  & Non (financement par des dons, adhésions et subventions)\\ \hline
                \textbf{Capture de la valeur (recherche de profit par les investisseurs)} & Limitée & Limitée & Nulle & Nulle \\ \hline
                \textbf{Ressources humaines} & Salariées & Salariées & Principalement salariées & Salariées et/ou bénévoles \\ \hline

              \end{tabularx}
            \end{table}

          Cette classification embryonnaire ne mobilise pas toutes les composantes du business model, mais celles qui sont le plus souvent utilisées. Le modèle entrepreneurial se développe avec le concept d'entreprise sociale. L'idée est de répondre à un besoin social ou environnemental à travers une activité économique rentable, adoptant les \cit{bonnes pratiques} managériales. Seule la finalité et une moindre recherche de profits différencient ce modèle de l'entrepreneuriat classique. Le modèle coopératif est logiquement celui des coopératives. La finalité économique est bien présente, mais la grande différence avec le précédent est l'entrepreneuriat collectif. Le modèle militant se caractérise également par un entrepreneuriat collectif, mais se distingue par un modèle économique hors-marché. Les organisations peuvent proposer des prix plus élevés - mais considérés comme plus justes - ou au contraire des prix plus faibles pour donner accés à des biens et services à des populations défavorisées. Enfin, le modèle associatif (qui mériterait certainement une sous-catégorisation) se caractérise par la non lucrativité, un modèle économique ne dépendant pas strictement des ressources issues de l'activité et le possible recours au bénévolat. \\

          Alors que les acteurs de l'\ess prétendent jouer un rôle important dans la lutte contre le réchauffement climatique et plus largement dans la protection de l'environnement, les connaissances spécificiques à l'action environnementale dans ses organisations restent encore limitées. Nous espérons encourager de futures recherches sur un sujet déterminant pour les prochaines générations.
